\documentclass[../../main.tex]{subfiles}

\begin{document}
\providecommand{\local}{{L^{1}_{loc}}}
\providecommand{\lf}{{\mathcal{L}_f}}

\problem{3.17}
\begin{wts}
Let the maximal function of any measurable $f\in\borel_\realn$ be denoted by $Hf(x)$, more precisely,
\[
Hf(x) = \sup_{r>0}A_r|f|(x) = \sup_{r>0}\dfrac{1}{m(B(r,x))}\int_{B(r,x)}f(y)dy
\]
where $A_r|f|$ is the average of $|f|$ on a ball with radius $r>0$ centered at $x\in\realn$. In symbols,
\[A_r|f| = \dfrac{1}{m(B(r,x))}\int_{B(r,x)}f(y)dy\]
The maximal theorem makes two claims:
\begin{enumerate}
    \item $(Hf)^{-1}((\alpha,+\infty))=\bigcup\limits_{r>0}(A_r|f|)^{-1}((\alpha,+\infty))$, and $Hf$ is measurable for every $f\in\local$. 
    \item There exists a $C>0$, for every $f\in L^1$ \[
        m(\{Hf(x)>\alpha\})\leq \dfrac{C}{\alpha}\norm{f}_1
    \]
    for every $\alpha>0$.
\end{enumerate}
\end{wts}
\begin{proof}
    Let $\alpha>0$ and fix $z\in (Hf)^{-1}((\alpha,+\infty))$, so $Hf(z)>\alpha$ and 
    \[\sup_{r>0}A_r|f|(z)>\alpha\]
    and with $Hf(z)-\alpha>0$, we get some $r_0>0$ \[
    Hf(z)-(Hf(z)-\alpha)=\alpha < A_{r_0}|f|(z)\implies z\in \bigcup\limits_{r>0}(A_r|f|)^{-1}((\alpha,+\infty))
    \]
    Next, let $z\in\bigcup\limits_{r>0}(A_r|f|)^{-1}((\alpha,+\infty))$, it is clear that
    \[
    Hf(z)\geq A_{r_0}|f|(z)>\alpha
    \]
    for some $r_0>0$. Since $A_r|f|$ (a function indexed by $r>0$) is continuous in $x\in\realn$, $(A_r|f|)^{-1}((\alpha,+\infty))$ is open, and $Hf$ is measurable.\\

    The second claim is slightly more intricate than the first. Define 
    \[
    E_\alpha = \bigset{Hf>\alpha}=\bigcup\limits_{r>0}\{A_r|f|>\alpha\}
    \]
    Let $x\in E_\alpha$, this induces a $r_x>0$ where $x\in\bigset{A_{r_x}|f|>\alpha}$. Rearranging gives
    \[
    \biggl(\dfrac{1}{\alpha}\int\limits_{B(r,x)}|f|dz\biggr)<m(B(r,x))
    \]
    We wish to apply Theorem 3.15 to this family of open balls. Notice
    \begin{itemize}
        \item Each $x\in E_\alpha\induces r_x>0\induces A_{r_x}|f|$,
        \item If $U = \bigcup\limits_{x\in E_\alpha} B(r_x,x)$, then $E_\alpha\subseteq U$,
        \item Choose $c<m(E_\alpha)\leq m(U)$ (by monotonicity) arbitrarily, 
        \item By Theorem 3.15, there exists a finite disjoint subcollection of points indexed by 
        \[
        x_1,\ldots,x_N \in E_\alpha
        \]
        so that $\bigsqcup_{j\leq N}B(r_{x_j},x_j)=U\supseteq E_\alpha$, and $c<3^n\sum_{j\leq k}m(B_j)$
        \item Define $B_j= B(r_{x_j}, x_j)$ for all $j\leq k$, and 
        \[
        m(B_j)< \dfrac{1}{\alpha}\cdot\int_{B_j}|f|dz
        \]
        by finite additivity,
        \[
        c3^{-n}<\sum_{j\leq k}m(B_j)<\dfrac{1}{\alpha}\cdot\sum_{j\leq k}\int_{B_j}|f|dz
        \]
        and finally
        \[
        c<\dfrac{3^n}{\alpha}\sum_{j\leq k}\int_{B_j}|f|dz\leq \dfrac{3^n}{\alpha}\norm{f}_1
        \]
        \item By inner regularity, of $m$ on $\borel$, since 
        \[
        m(E_\alpha) = \sup\bigset{m(K),\: K\in\cpt_{\realn}, \: K\subseteq E_\alpha}
        \]
        for any $K\in\cpt_{\realn}$, $K\subseteq E_\alpha$, we have $m(K)<+\infty$, $m(K)\leq m(E_\alpha)$ and 
        \[
        m(K)=c<\dfrac{3^n}{\alpha}\norm{f}_1\implies m(E_\alpha)\leq \dfrac{3^n}{\alpha}\norm{f}_1
        \]
    \end{itemize}
    \begin{remark}
        We used the properties of a Radon Measure here, without relying on the phrase `sending $c\to E_\alpha$`, which would require us to deal with two cases $m(E_\alpha)<+\infty$ and $m(E_\alpha)=+\infty$.
    \end{remark}
\end{proof}

\end{document}