\documentclass[../../main.tex]{subfiles}

\begin{document}
\providecommand{\fn}{\{F_n\}}
\begin{wts}
    Let $\nu$ be a signed measure on $(\xx,\mcal)$. If $\{E_j\}$ is an increasing sequence in $\mcal$, $\lim_{n\to +\infty}\nu(E_j)=\nu\qty(\bigcup E_j)$. If $\{E_j\}$ is a decreasing sequence in $\mcal$, $\lim_{n\to +\infty}\nu(E_j)=\nu\qty(\bigcap E_j)$ provided $\nu(E_1)$ is of finite measure.
\end{wts}
\begin{proof}
    Let $\nu$ be a signed measure, and fix any increasing sequence $E_j\nearrow E=\bigcup E_{j\geq 1}$ of sets. This induces a disjoint sequence in $\{F_n\}$. Define $F_1 = E_1$, and if $n\geq 2$,
    \[F_n = E_n\setminus\bigcup E_{j\leq n-1}\]
    Use $\sigma$-additivity of $\nu$, where the sum is 'defined' to be non-ambiguous.\\

    For the second part of the proof, notice if $A\subseteq B$ are measurable sets, if $\nu(A)=\pm\infty$, then $\nu(B)=\pm\infty$, because of the second property of $\nu$. Indeed,
    \[
        \nu(B) = \nu(A)+\nu(B\setminus A) = \pm\infty+ c
    \]
    where $c\in\real\cup\{\pm\infty\}$. Therefore $\nu(B)=\nu(A)$. By assumption $\nu(E_1)\in\real$, the contrapositive of the previous argument shows that the intersection $\cap E_j$ is of finite measure as well. We can produce an increasing sequence $G_n = E_1\setminus E_n$ for $n\in\nat^+$. Then
    \[
        \bigcup G_n = \bigcup E_1\setminus E_n = E_1\cap\qty[\bigcup E_n^c]=\qty[\bigcap E_j]^c
    \]
    We then write
    \[
        E_1 = \qty[\bigcup G_n] + \qty[\bigcap E_n]
    \]
    The finiteness of $\nu(E_1)$ on the left hand side implies all the terms in the union converge absolutely. Therefore
    \begin{align*}
            \nu(E_1) - \nu\qty(\bigcap E_n) &= \lim_{n\to+\infty} \nu(G_n)    \\
            &= \lim_{n\to+\infty}\nu(E_1) - \nu(E_n)\\
            &= \nu(E_1) - \lim_{n\to+\infty}\nu(E_n)
    \end{align*}
    Cancelling terms finishes the proof.
\end{proof}

\end{document}