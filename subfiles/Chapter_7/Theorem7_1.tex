\documentclass[../../main.tex]{subfiles}

\begin{document}
\problem{7.1}
\begin{wts}
    If $I$ is a linear functional on $\cc{X}$, then for every compact $K\subseteq X$, there exists some $C_k\geq 0$ with
    \[
    |I(f)|\leq C_K\cdot \norm{f}_u
    \]
\end{wts}
\begin{proof}
    Since $\supp{f}$ is compact, by Urysohn's Lemma (Theorem 4.32), there exists a $\phi\in\cc{X,[0,1]}$ such that $\phi=1$ on $K$ and vanishes outside some compact $\cl{V}\subseteq X$. Then at every $x$,
    \[
    -\norm{f}_u\leq f(x)\leq +\norm{f}_u
    \]
    Implies that
    \[
    (-\norm{f}_u)\phi \leq f(x)\leq (+\norm{f}_u)\phi
    \]
    So that $f+\norm{f}_u\phi\geq 0$ and $+\norm{f}_u-f\geq 0$, and by linearity,
    \[
    (-\norm{f}_u)I(\phi)\leq I(f)\leq (+\norm{f}_u)I(\phi)
    \]
    Therefore $|I(f)|\leq I(\phi)\norm{f}_u$, and taking $C_K = I(\phi)$ will suffice.
\end{proof}
\end{document}