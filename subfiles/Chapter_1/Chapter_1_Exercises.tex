\documentclass[../../main.tex]{subfiles}

\begin{document}

\providecommand{\mustar}{\mu^*}
\providecommand{\calm}{\mathcal{M}}
\providecommand{\caln}{\mathcal{N}}
\subsection*{Problem 1.4}
\begin{wts}
    An algebra $\mathcal{A}$ is a $\sigma$-algebra $\iff$ it is closed under countable increasing unions.
\end{wts}
\begin{proof}
    $\impliedby$ is trivial. And it suffices to show that $\mathcal{A}$ is closed under countable disjoint unions. Indeed, if $\{E_j\}_{j\geq 1}\subseteq \mathcal{A}$ is a countable disjoint sequence of sets, write
    \[
    F_n=\bigcup E_{j\leq n}
    \]
    Clearly, $F_j$ is increasing, and denote $F=\cup E_{j\geq 1}$, which is a member of $\mathcal{A}$. We claim that
    \[
    \bigcup F_{n\geq 1}=\bigcup E_{j\geq 1}
    \]
    Fix any $x\in\bigcup E_{j\geq 1}$, then $x$ belongs in some $E_j\subseteq F_j$, and $\supseteq$ is proven. Also, if $x\in\bigcup F_{n\geq 1}$, then there exists some $F_n$ for which $x$ is a member of. For this particular $F_n$, means that $x\in E_j$ where $j\leq n$ and $x\in \bigcup E_{j\geq 1}$.
\end{proof}

\newpage\subsection*{Problem 1.5}
\begin{wts}
    Let $\calm(\Epsilon)$ be the $\sigma$-algebra generated by $\Epsilon\subseteq X$, and 
    \[
    \caln =\biggl\{\calm (\mathcal{F}),\, \mathcal{F}\subseteq \Epsilon,\, \mathcal{F} \text{ is countable}\biggr\}
    \]
    Show that $\calm(\Epsilon) = \caln$.
\end{wts}
\begin{proof}
    The outline of the proof is as follows,
    \begin{enumerate}
        \item Prove that $\caln \subseteq \calm(\Epsilon)$,
        \item Show that $\caln$ is a $\sigma$-algebra,
        \item Show that $\caln$ contains $\Epsilon$ as a subset, and hence $\calm(\Epsilon)\subseteq \caln$.
    \end{enumerate}
    First, for any $\mathcal{F}\subseteq \Epsilon$, where $\mathcal{F}$ is countable, it follows from Lemma 1.1 that $\calm(\mathcal{F})\subseteq\calm{\Epsilon}$. Taking the union over all of such $\mathcal{F}$, we get $\bigcup \mathcal{M}(\mathcal{F})=\caln\subseteq \mathcal{M}(\Epsilon)$.\\
    
    To show that $\caln$ is a $\sigma$-algebra, fix any $A\in \caln$, and $A$ belongs to $\calm(\mathcal{F})$, therefore $A^c\in\calm(\mathcal{F})\subseteq \caln$. To show closure under countable unions, fix a sequence $\{E_j\}\subseteq\caln$, then each of these $E_j$ belongs to a corresponding $\calm(\mathcal{F}_j)$, for $j\in \{1,2,\ldots\}$. Now define 
    \[
    \overline{\mathcal{F}} = \bigcup \mathcal{F}_{j\geq 1}\subseteq \Epsilon
    \]
    and $\overline{\mathcal{F}}$ is obviously countable. Hence for every $j\geq 1$, $\calm(\mathcal{F}_j)\subseteq \mathcal{M}(\overline{\mathcal{F}})$ and taking the union yields
    \[
    \bigcup \calm(\mathcal{F}_{j\geq 1})\subseteq \calm (\overline{\mathcal{F}})\subseteq \caln
    \]
    It is also clear that our sequence $\{E_j\}$ is contained in $\calm(\overline{\mathcal{F}})$, and $E=\bigcup E_j$ belongs to $\calm(\overline{\mathcal{F}})\subseteq\caln$ as an element. Therefore $\caln$ is a $\sigma$-algebra.\\
    
    Let $\alpha\in A$ index the family of sets in $\Epsilon$, (so that $E_\alpha\in\Epsilon$) and the singleton set of a set $\{E_\alpha\}$ is a countable subset of $\Epsilon$. For every $\alpha\in A$, we have
    \[
    E_\alpha\in\calm(\{E_\alpha\})\subseteq\caln\implies \Epsilon\subseteq\caln
    \]
    And one final application of Lemma 1.1 finishes the proof.
\end{proof}

\newpage\subsection*{Problem 1.7}
\begin{wts}
    If $\mu_1,\ldots,\mu_n$ are measures on $(X,\calm)$, and $a_1,\ldots, a_n\in[0,+\infty)$, then $\mu = \sum^n_1\mu_j$ is a measure on $(X,\calm)$.
\end{wts}
\begin{proof}
    If $\{E_j\}$ is a disjoint sequence in $\calm$, and denote $E = \bigcup (E_j)$. If for each $k\leq n$, $\mu_k(E)<+\infty$, 
    \[
    \mu_k(E)=\sum \mu_k(E_j)\implies a_k\mu_k(E)=\sum a_k\mu_k(E_j)
    \]
    Then,
    \[
    \mu(E)=\sum_{k\leq n}a_k\mu_k(E)= \sum_{k\leq n}\sum_{j\geq 1}a_k\mu_k(E_j)=\sum_{j\geq 1}\sum_{k\leq n}a_k\mu_k(E_j)=\sum_{j\geq 1}\mu(E_j)
    \]
    If there exists some $mu_k$ such that $\mu_k(E) = +\infty$, then 
    \[
    \mu(E) = \sum_{k\leq n}\sum_{j\geq 1}a_k\mu_k(E_j)
    \]
    Now if there exists some $\mu_{k'}$ with $\mu_{k'}(E)=+\infty$, then $\mu(E)=\sum_{k\leq n}\mu_k(E)=+\infty$, and 
    \[
    \sum_{j\geq 1}\mu(E_j)=\sup_N\sum_{j\leq N} \sum_{k\leq n}a_k\mu_k(E_j)\geq \mu_{k'}(E)
    \]
    Therefore $\mu(E)=\sum_{j\geq 1}\mu(E_j)$, and $\mu$ is a measure.
\end{proof}

\newpage\subsection*{Problem 1.8}
\begin{wts}
    If $(X,\calm,\mu)$ is a measure space, and $\{E_j\}\subseteq \calm$, then $\mu(\liminf E_j)\leq \liminf \mu(E_j)$. Also, $\mu(\limsup E_j)\geq \limsup \mu(E_j)$ provided that $\mu(\bigcup E_{j\geq 1})<+\infty$
\end{wts}
\begin{proof}
If $\{E_j\}_{j\geq 1}$ is a sequence in $\calm$, and define $F_m = \bigcap_{j\geq m}E_j$
\[
\liminf E_j = \bigcup_{m\geq 1}\bigcap_{j\geq m}E_j = \bigcup_{m\geq 1} F_m
\]
Also, for every $m\geq 1$, $F_m\subseteq E_m$, and $F_m$ is an increasing sequence, because
\[
[m,+\infty)\supseteq [m+1,+\infty)\implies F_m\subseteq F_{m+1}
\]
Using continuity above, and writing $F = \bigcup F_{m\geq 1}=\liminf E_j$, we have
\begin{align*}
    \mu(\liminf E_j) &= \mu(F)\\
    &= \liminf \mu(F_m)\\
    &\leq \liminf \mu(E_m)
\end{align*}.

The second part of the proof is similar, if $G_m = \bigcup_{j\geq m} E_j$, then
\[
\limsup E_j = \bigcap_{m\geq 1}\bigcup_{j\geq m} E_j = \bigcap_{m\geq 1}G_m
\]
Similarly, $G_m$ is a decreasing sequence, and since $\mu(\bigcup E_{j\geq 1}) = \mu(G_1)$ is finite, we can use continuity from above in the same manner, and the proof is complete.
\end{proof}
\newpage


\subsection*{Problem 1.12}
\begin{wts}
Let $(X,\calm,\mu)$ be a finite measure space,
\begin{itemize}
    \item If $E,F\in\calm$, and $\mu(E\Delta F)=0$, then $\mu(E)=\mu(F)$,
    \item Say that $E\sim F$ if $\mu(E\Delta F)=0$, then $\sim$ is an equivalence relation on $\calm$,
    \item For every $E,F\in\calm$, define $\rho(E,F)=\mu(E\Delta F)$. Show that $\rho$ defines a metric on the space of $\calm/\sim$ equivalence classes.
\end{itemize}
\end{wts}
\begin{proof}[Proof of Part A]
    Use the fact that $\mu(F)=\mu(E\cap F)+\mu(F\cap E^c)$, and by monotonicity,
    \[
    \mu(F\cap E^c)\leq \mu(E\Delta F)=0
    \]
    And $\mu(F)=\mu(E\cap F)=\mu(E)$, the last equality follows after a simple modification.
\end{proof}
\begin{proof}[Proof of Part B]
    Suppose that $\mu(E\Delta F)=\mu(F\Delta G)=0$, then
    \begin{itemize}
        \item $\mu(E\cap F^c)=\mu(F\cap E^c)\leq \mu(E\Delta F)=0$ by monotonicity,
        \item Similarly, we have $\mu(F\cap G^c)\mu(G\cap F^c)=0$, and
        \item By subadditivity, 
        \begin{itemize}
            \item $\mu(E\cap G^c)=\mu(E\cap F^c\cap G^c) + \mu(E\cap F\cap G^c)\leq 0$, and $\mu(E\cap G^c)=0$, and
            \item $\mu(G\cap E^c)=0$
        \end{itemize}
        \item Therefore $\mu(E\Delta G)=\mu(E\cap G^c) + \mu(G\cap E^c)=0$
    \end{itemize}
    It is clear that the relation is reflexive, since $E\Delta E =\varnothing$, and symmetry is trivial.
\end{proof}
\begin{proof}[Proof of Part C]
    Since $\rho(E,F)=\rho(F,E)$, and $\rho(E,F)\geq 0$ for every $E,F\in\calm$, and $\rho(E,F)=0\iff E\sim F$. We only have to prove the Triangle Inequality. Notice that
    \begin{align*}
    \mu(E\setminus F) &= \mu(E\cap F^c\cap G) + \mu(E\cap F^c\cap G^c)\\
    &\leq \mu(F^c\cap G) + \mu(E\cap F^c)
    \end{align*}
    and in the same fashion,
    \[
    \mu(F\setminus E) \leq \mu(F\cap G^c) + \mu(E^c\cap F)
    \]
    Combining the two inequalities, and applying additivity finishes the proof.
\end{proof}

\newpage\problem{1.13}
\begin{wts}
Every $\sigma$-finite measure is semi-finite
\end{wts}
\begin{proof}
    Suppose $\mu$ is $\sigma$-finite then there exists an increasing sequence of sets $E_j\nearrow X$ with $\mu(E_j)<+\infty$. Now for every $W\in\mathcal{M}$, if $\mu(W)=+\infty$ then $\mu(W)=\lim_{j\to\infty}\mu(E_j\cap W)=+\infty$. Since this real-valued limit converges to its supremum $+\infty$, there exists a non-null subset $E_j\cap W$ of finite measure.
\end{proof}

\newpage

\problem{1.14}
\begin{wts}
If $\mu$ is a semi-finite measure, and if $\mu(E)=+\infty$, for every $C>0$, there exists an $F\subseteq E$ with $0<\mu(F)<+\infty$.
\end{wts}
\begin{proof}
    Suppose by contradiction that there exists a $C>0$ so for every $F\subseteq E$, if $F$ is of finite measure, then $0\leq \mu(F)\leq C$. Let $s = \sup\{\mu(F),\,F\subseteq E,\, 0<\mu(F)<+\infty\}$, and for any $n^{-1}>0$, this induces a $F_n$ with measure\[\mu(F_n)>s-n^{-1}\] and take $A_n = \bigcup_{j\leq n}F_j$. A simple induction will show that $\mu(A_n)\leq \sum_{j\leq n} \mu(F_j)<+\infty$, therefore $\mu(A_n)\leq s$ for every $n\geq 1$. By continuity from below\[\lim_{n\to\infty}\mu(A_n)=\mu\biggl(\bigcup_{j\geq 1}F_j\biggr)\leq s\]
    Next, by monotonicity, denoting the union over $A_n$ by $A$, for every $n^{-1}>0$\[s-n^{-1}\leq\mu(A_n)\leq\mu(A)\leq s\implies \mu(A)=s\]
    Now, $E\setminus A$ is a set of infinite measure, and by semi-finiteness. Find a set $B\subseteq E\setminus A$ with strictly positive measure, so that \[\mu(A\cup B)=\mu(A)+\mu(B)>s\]
And this finishes the proof.
\end{proof}
\newpage
\problem{1.15}
\providecommand{\xx}{\mathbf{X}}
\providecommand{\calm}{\mathcal{M}}
\begin{wts}
    Given a measure $\mu$ on $(\xx,\calm)$, and define $\mu_0 = \sup\{\mu(F),\, F\subseteq E,\, \mu(F)<+\infty\}$. Show $\mu_0$ is semi-finite. Then, show that if $\mu$ is semi-finite, $\mu=\mu_0$. Lastly, there exists a measure $\nu$ on $(\xx,\calm)$, with $\mu = \nu + \mu_0$, where $\nu$ only assumes the values $0$ or $+\infty$.
\end{wts}
\begin{proof}
First, a small Lemma. We claim that $\mu_0 = \mu$ on finite sets. Let $E\in\calm$, and $\mu(E)<+\infty$, since \[\mu(E)\in\{\mu(F),\, F\subseteq E,\,\mu(E)<+\infty\}\implies \mu(E)\leq \mu_0(E)\]
Next, for every $W\subseteq E$, $\mu(W)\leq \mu(E)$, so $\mu_0(E)\leq \mu(E)$. This proves the equality.\\

If $E$ is any measurable subset of $\xx$, and suppose also $\mu_0(E)=+\infty$,  one can easily find subsets of $E$, $\{E_n\}_{n\geq 1}$ with\[n\geq \mu(E_n)<+\infty\]
But $E_n$ is a subset of finite measure, so $0<\mu(E_n)=\mu_0(E_n)<+\infty$. This proves the semi-finiteness of $\mu_0$.\\

Next, suppose $\mu$ is semi-finite, and fix any measurable set $E$. If $E$ is if finite measure, then $\mu(E)=\mu_0(E)$, and if $\mu(E)=+\infty$, apply Exercise 14, so there exists a sequence of subsets of finite measure $E_n\subseteq E$ for every $n\geq 1$, with $\mu(E_n)\to \mu(E)$. Therefore $\mu_0(E)=\mu(E)$.\\

For the last part of the proof, let $\mu$ be an arbitrary measure. And let $E\in\calm$. If $\mu(E)<+\infty$, then $\nu(E)=0$ would suffice (this proves the first property of the measure). If $\mu(E)=+\infty$, and if $\mu(E)$ is not semi-finite, then set $\nu(E)=+\infty$. So that $\mu_0(E)+\nu(E)=0 + \infty = \infty = \mu(E)$. The additivity of $\nu$ is immediate, since $\nu$ can only assume two values. This finishes the proof.
\end{proof}





\newpage\subsection*{Problem 1.24}
\begin{wts}
    If $\mu$ is a finite measure on $(X,\mathcal{M})$, and let $\mu^*$ be the outer measure. Suppose that $E\subseteq X$ satisfies $\mustar(E)=\mustar(X)$ (but $E\notin \mathcal{M}$ necessarily). Show that
    \begin{enumalpha}
        \item For any $A,B\in\mathcal{M}$, and $A\cap E = B\cap E$, then $\mu(A)=\mu(B)$.
        \item Let $\mathcal{M}_E=\{A\cap E,\,A\in\mathcal{M}\}$, and define $\nu$ on $\mathcal{M}$ with $\nu(A\cap E)=\mu(A)$. Then $\mathcal{M}_E$ is a $\sigma$-algebra, and $\nu$ is a measure on $\mathcal{M}_E$.
    \end{enumalpha}
\end{wts}
\begin{proof}[Proof of Part A]
% First, notice that $\mathcal{M}$ is an algebra, and $\mu$ is a pre-measure. Therefore $X\in\mathcal{M}$ is $\mustar$-measurable, and
% \[
% \mustar(X)=\mustar(E)=\mustar(X\cap E)+\mustar(X\setminus )
% \]
% How to actually show this?
\[
\mustar(E)=\mustar(X)\implies \mustar(X\setminus E)=0
\]
This is a simple consequence of the $\mustar$-measurability of $X$, since $X\in\mathcal{M}$, and the $\mu$ is a pre-measure on $\mathcal{M}$, b
And by monotonicity, 
\[
\begin{cases}A\cap (X\setminus E)\subseteq (X\setminus E)\\ B\cap (X\setminus E)\subseteq (X\setminus E)\end{cases}\implies\begin{cases}\mustar(A\cap (X\setminus E))=0\\
\mustar(B\cap (X\setminus E))=0\end{cases}
\]
Write $A\cap X = (A\cap E)\cup (A\cap X\setminus E)$, and by subadditivity of $\mustar$,
\begin{align*}
\mu(A)&=\mustar(A\cap X)\\
&\leq \mustar(A\cap E)+\mustar(X\setminus E)\\
&=\mustar(B\cap E)\\
&\leq \mustar(B\cap X)\\
&=\mu(B)
\end{align*}
Therefore $\mu(A)\leq \mu(B)$, and $\mu(B)\leq \mu(A)$ is trivial. 

\end{proof}

\begin{proof}[Proof of Part B]
    We want to show $\mathcal{M}_E$ is a $\sigma$-algebra.
    \begin{itemize}
        \item Closure under complements,
        \[
        \forall A\cap E\in\mathcal{M}_E, A\in\mathcal{M}\implies (E\setminus A^c)\in\mathcal{M}
        \]
        Therefore $(E\setminus A^c)\cap E\in\mathcal{M}_E$. Note that the question mentions that $\mathcal{M}_E$ is a $\sigma$-algebra on $E$, therefore we take complements relative to $E$.
        \item Closure under countable unions. Fix any countable sequence $\{A_j\cap E\}\subseteq\mathcal{M}_E$ where $\{A_j\}\subseteq \mathcal{M}$. It is obvious that $A=\cup A_j\in\mathcal{M}$, therefore $\cup (A_j\cap E)=E\cap A\in\mathcal{M}_E$ as well.
    \end{itemize}
    Since $\nu(\varnothing)=\mu(\varnothing\cap E)=0$, and for countable additivity, fix any disjoint sequence $\{A_j\cap E\}_{j\geq 1}\subseteq \mathcal{M}_E$, where $\{A_j\}_{j\geq 1}\subseteq\mathcal{M}$, and let $A=\bigcup A_{j\geq 1}$
    \begin{align*}
        \nu(A\cap E)&=\mu(A)\\
        &=\sum \mu(A_{j\geq 1})\\
        &=\sum \nu(A_{j\geq 1}\cap E)\\
    \end{align*}
\end{proof}
\end{document}