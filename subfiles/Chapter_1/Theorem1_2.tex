\documentclass[../../main.tex]{subfiles}

\begin{document}
\subsection{Theorem 1.2}
\newcommand{\borel}{\mathbb{B}_{\mathbb{R}}}
\newcommand{\gen}[1]{\mathcal{M}(\mathcal{E}_{#1})}
\begin{wts}
    The Borel $\sigma$-algebra of $\mathbb{R}$, $\borel$ is generated by the following
    \begin{itemize}
        \item The family of open intervals $\cal{E}_1=\{(a,b),\:a<b\}$,
        \item The family of closed intervals $\cal{E}_2=\{[a,b],\:a<b\}$,
        \item The family of half-open intervals $\cal{E}_3=\{(a,b],\:a<b\}$ or $\cal{E}_4=\{[a,b),\:a<b\}$
        \item The open rays $\cal{E}_5=\{(a,+\infty),\:a\in\mathbb{R}\}$ or $\cal{E}_6=\{(-\infty,a),\:a\in\mathbb{R}\}$
        \item The closed rays
        $\cal{E}_7=\{[a,+\infty),\:a\in\mathbb{R}\}$ or
        $\cal{E}_8=\{(-\infty,a],\:a\in\mathbb{R}\}$
    \end{itemize}
\end{wts}
\begin{proof}
    By definition, $\borel$ is generated by the family of all open sets in $\mathbb{R}$, but every open set is a countable union of open intervals. Therefore
    \[
    \Tau_\mathbb{R}\subseteq \cal{M}(\cal{E}_1) \implies \borel\subseteq\cal{M}(\cal{E}_1)
    \]
    Conversely, every open interval is an open set, hence
    \[
    \cal{E}_1\subseteq \Tau_\mathbb{R}\subseteq\borel\implies\cal{M}(\cal{E}_1)\subseteq\borel
    \]
    Every closed interval can also be written as a countable intersection of open intervals, for every $[a,b]$, with $a<b$, we have
    \begin{equation}\label{closed interval as countable intersection of open intervals}
    [a,b] = \bigcap_{n\geq 1}(a-n^{-1},b+n^{-1})
    \end{equation}
    Indeed, fix any $x\in[a,b]$ then for every $n\geq 1$, 
    \[
    a-n^{-1}<a\leq x\leq b<b+n^{-1}
    \]
    So $x\in \bigcap_{n\geq 1} (a-n^{-1},b+n^{-1})$. If $x$ an element of the left member, then for every $n\geq 1$,
    \[
    a-n^{-1}<x\implies a-x\leq 0
    \]
    Similarly for $x\leq b$, therefore equation \eqref{closed interval as countable intersection of open intervals} is valid, and $\cal{E}_2\subseteq \borel=\cal{M}(\cal{E}_1)$. To show the reverse estimate, every open interval can be written as a countable union of closed intervals,
    \begin{equation}\label{open interval as countable union of closed intervals}
    (a,b)=\bigcup_{n\geq 1}[a+n^{-1},b-n^{-1}]
    \end{equation}
    To show that the above estimate is indeed true, fix any $x\in(a,b)$, then
    \begin{align*}
        a<x<b&\iff a<a+n^{-1}\leq x\leq b-n^{-1}<b\\
        &\iff x\in\bigcup_{n\geq 1}[a+n^{-1},b-n^{-1}]
    \end{align*}
    So that equation \eqref{open interval as countable union of closed intervals} holds. By similar argumentation we have $\cal{E}_1\subseteq\cal{M}(\cal{E}_2)\implies\cal{M}(\cal{E}_2)=\cal{M}(\cal{E}_1)$.\\
    
    For $\cal{E}_3$, $\cal{E}_4$ 
    \begin{itemize}
        \item $(a,b] = \bigcap_{n\geq 1}(a, b+n^{-1})$, proves $\gen{3}\subseteq\gen{1}$,
        \item $(a,b) = \bigcup_{n\geq 1}(a, b-n^{-1}]$, proves $\gen{1}\subseteq\gen{3}$,
        \item $[a,b) = \bigcup_{n\geq 1}[a, b-n^{-1}]$, proves $\gen{4}\subseteq\gen{2}$,
        \item $[a,b] = \bigcap_{n\geq 1}[a, b+n^{-1})$, proves $\gen{2}\subseteq\gen{4}$
    \end{itemize}
    So that $\gen{1}=\gen{2}=\gen{3}=\gen{4}=\borel$. By taking complements of each element we get $\gen{5}=\gen{8}$ and $\gen{6}=\gen{7}$. Notice also that
    \begin{itemize}
        \item $(a,b] = (a,+\infty)\cap (-\infty,b]$, proves $\cal{E}_3\subseteq\gen{5}$, and $\gen{3}\subseteq\gen{5}$.
        \item $(a,+\infty) = \cup_{n\geq 1}(a,a+n]$, proves $\cal{E}_5\subseteq\gen{3}$, and $\gen{5}\subseteq\gen{3}$.
        \item $[a,b) = [a,+\infty)\cap (-\infty, b)$, proves $\cal{E}_4\subseteq\gen{6}$, and $\gen{4}\subseteq\gen{7}$,
        \item $[a,+\infty) = \bigcup_{n\geq 1}[a,a+n)$, proves $\cal{E}_7\subseteq\gen{4}$, and $\gen{7}\subseteq\gen{4}$.
    \end{itemize}
    Finally, $\gen{3}=\gen{5}=\gen{8}=\borel$ and $\gen{4}=\gen{6}=\gen{7}=\borel$.
\end{proof}

\end{document}