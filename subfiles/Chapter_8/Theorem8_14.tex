\documentclass[../../main.tex]{subfiles}

\begin{document}
\providecommand{\cnv}{\ast}
\providecommand{\zint}{\int\limits_{z\in\real^n}}
\providecommand{\ubc}{\operatorname{UBC}}
\problem{8.14}
\begin{wts}
    Suppose $\phi\in L^1$, and $\int\phi(x)dx=a$.
    \begin{enumalpha}
        \item If $f\in L^p$, $p\in[1,+\infty]$, then $f\ast \phi_t\to af$ in the $L^p$ norm as $t\to 0$.
        \item If $f$ is bounded and uniformly continuous, then $f\ast \phi_t\to af$ uniformly as $t\to 0$.
        \item If $f\in L^\infty$ and $f$ is continuous on an open set $U$, then $f\ast \phi_t\to af$ uniformly on compact subsets of $U$ as $t\to 0$
    \end{enumalpha}
\end{wts}
\begin{proof}[Proof of Part A]
    First, the convolution $f\cnv \phi_t$ is in $L^p$ by Young's Inequality (Theorem 8.7). Furthermore, 
    \begin{equation}\label{convolution difference}
        f\cnv \phi_t - af = \int\limits_{y\in\real^n} f(x-y)t^{-n}\phi(t^{-1}y)dy - \int\limits_{y\in\real^n} f(x)\phi(y)dy
    \end{equation}
    Now apply Theorem 2.44, with $y\mapsto y/t$, and denote this invertible map by $T\in GL(n,\real)$, so that $|\det(T)|=t^{-n}$, then $y = T(y)t$ for every $t>0$. It follows that
    \begin{align}
    (f\cnv \phi_t)(x)&=|\det(T)|\cdot\int\limits_{y\in\real^n}f(x-t\cdot Ty)\phi(T(y))dy\nonumber\\
    &=\int\limits_{z\in\real^n}f(x-tz)\phi(z)dz\nonumber\\
    &=\int\limits_{z\in\real^n}\tau_{tz} f(x)\phi(z)dz\label{convolution equality}
    \end{align}
    Next,  $a=\int\phi$ so $af = \int f(x)\phi(z)dz$. Using Equations \eqref{convolution difference} and \eqref{convolution equality} we get
    \begin{equation}\label{cnv diff}
        (f\cnv \phi_t - af)(x) = \int\limits_{z\in\real^n}(\tau_{tz}f-f)\phi(z)dz
    \end{equation}
    We wish to apply Minkowski's Inequality for integrals, which states, roughly speaking:
    \begin{quote}
        The norm of an integral is less than the integral of the norm.
    \end{quote}
    to Equation \eqref{cnv diff}, and 
    \begin{equation}\label{minkowski inequality cnv}
        \norm{f\ast\phi_t -af}_p\leq \int_{z\in\real ^n}\norm{(\tau_{tz}f-f)\phi(z)}_pdz
    \end{equation}
    The assumptions for Theorem 6.19 are satisfied by
    \begin{enumerate}
        \item  Notice for every $z\in\real^{n'}$, \[\norm{(\tau_{tz}f-f)\phi(z)}_p = \biggl(\int\limits_{x\in\real^n}|(\tau_{tz}f(x)-f(x))\phi(z)|^pdx\biggr)^{1/p}\leq |\phi(z)|\biggl(2\norm{f}_p\biggr)<+\infty\]
        Since $\norm{\phi}_1<+\infty$, $|\phi(z)|<+\infty$ almost everywhere. 
        \item Next, to show $z\mapsto \norm{\phi(z)(\tau_{tz}f-f)}_p$ is in $L^1{\real^n,z}$. Reuse the last estimate in the previous bullet point, and \[\norm{\phi(z)(\tau_{tz}f-f)}_p\leq |\phi(z)|\biggl(2\norm{f}_p\biggr)\]
        Taking the integral in $L^+$ with respect to $z$, we get
        \[\bignorm{\biggl(\norm{\phi(z)(\tau_{tz}f-f)}_p\biggr)}_1<+\infty\]
        so both assumptions are satisfied.
    \end{enumerate}
    Therefore Equation \eqref{minkowski inequality cnv} holds. Next, fix any sequence of $t_n>0$ with $t_n\to 0$. The Dominated Convergence Theorem gives, since $|\phi(z)|\norm{\tau_{t_nz}f-f}_p$ is dominated by $|\phi(z)|\cdot2\norm{f}_p\in L^1\cap L^+$
    \begin{align*}
    \lim_{n\to\infty}\zint\norm{\tau_{t_nz}f-f}_p|\phi(z)|dz&=\zint\lim_{n\to\infty}\norm{\tau_{t_nz}f-f}_p|\phi(z)|dz\\
    &=\zint 0 dz\\
    &= 0
    \end{align*}
    The second last equality is from Lemma 8.4, as translation is continuous in the $L^p$ norm for $p\in[1,+\infty)$. So almost every $z\in\real^n$ (since again, $|\phi(z)|$ can be infinite on a null set),
    \[\norm{\tau_{t_nz}f-f}_p\to 0\implies\norm{\tau_{t_nz}f-f}_p|\phi(z)|\to 0\]
    as $n\to+\infty$. It follows that
    \[\lim_{n\to\infty}\norm{f\cnv\phi_{t_n}-af}_p=\lim_{n\to\infty}\bignorm{\,\,\zint[\tau_{t_nz}f(x)-f(x)]\phi(z)dz\,}_p =0\]
    Since the sequence $t_n\to 0$ is arbitrary, we conclude that the function $t\mapsto \norm{f\cnv \phi_{t}-af}_p$ has a limit of $0$ as $t\to 0$.
\end{proof}
\begin{proof}[Proof of Part B]
    Suppose $f\in \ubc(\real^n)$, so that $f$ is uniformly continuous and bounded. We wish to show $f\cnv \phi_t\to af$ uniformly as $t\to 0$. In symbols,
    \[g:t\mapsto\norm{f\cnv \phi_t-af}_u,\,g\to 0,\,\text{as }t\to 0\]
    The convolution between $f$ and $\phi_t$ makes sense at every $x\in\real^n$, as \[\int|\tau_yf(x)||\phi(y)|dy\leq \norm{f}_u\cdot\norm{\phi}_1<+\infty\]
    Taking the supremum norm on both sides of Equation \eqref{cnv diff}, we get
    \begin{align}
        \norm{f\cnv \phi_t -af}_u &= \sup_{x\in\real^n}\left|\,\zint(\tau_{tz}f-f)\cdot\phi(z)dz\right|\nonumber\\[2ex]
        &\leq \sup_{x\in\real^n}\zint|\tau_{tz}f-f|\cdot|\phi(z)|dz\nonumber\\[2ex]
        &\leq\zint\sup_{x\in\real^n}|    \tau_{tz}f-f|\cdot|\phi(z)|dz\nonumber\\[2ex]
        &=\zint\norm{\tau_{tz}f-f}_u\cdot|\phi(z)|dz\label{uniform estimate}
    \end{align}
    the last equality is a simple consequence of the monotonicity of the integral in $L^+$, indeed. For every $x\in\real ^n$, the following holds pointwise for almost every $z$\[|\tau_{tz}f-f|\leq\norm{\tau_{tz}f-f}_u\implies\sup_{x\in\real^n}|\tau_{tz}f-f|\leq\norm{\tau_{tz}f-f}_u\]
    Apply the Dominated Theorem to the right member of \eqref{uniform estimate}, noting that it is dominated by $|\phi(z)|\cdot2\norm{f}_u\in L^1\cap L^+$ as we have done for Part A of the proof. Since this holds for every sequence $t_n\to 0$, the proof is complete.
\end{proof}
\begin{proof}[Proof of Part C]
    Next, suppose that $f\in L^\infty$, and $f\in C(U)$, where $U$ is open in $\real^n$. We claim that 
    \[f\cnv \phi_t\to af\]
    within the topology of uniform convergence on compact subsets of $U$. So that for every $K\in\cpt$, $K\subseteq U$\[\sup_{x\in K}\biggl|f\cnv\phi_t-af\biggr|\to 0,\,\text{as }t\to 0\]
    First, a small technical Lemma.
    \begin{lemma}\label{814 lemma}
        If $\phi\in L^1(\realn)$, then for every $\varepsilon>0$, there exists $E\in\cpt$, with
        \[\int_{E^c}|\phi|=\norm{\phi\chi_{E^c}}_1<+\varepsilon\]
    \end{lemma}
    \begin{proof}
        Assume that $\phi\geq 0$, if not, replace $\phi$ by $|\phi|$. Since $\cc{\realn}$ is dense in $L^1$ for every $\varepsilon2^{-1}>0$ there exists some $\psi\in\cc{\realn}$ with $\norm{\psi-\phi}_1<\varepsilon^{-1}$, and denote $E=\supp{\psi}\in\cpt$, then
        \[\norm{|\psi|-|\phi|}_1\leq\norm{\psi-\phi}_1<\varepsilon2^{-1}\]
        So we can assume $\psi\geq 0$ as well, perhaps by relabelling $\psi$ by $|\psi|$. Then,
        \[\norm{\psi-\chi_E\phi}_1=\norm{\chi_E(\psi-\phi)}_1\leq\norm{\psi-\phi}_1<\varepsilon2^{-1}\]
        by monotonicity in $L^+$. The Triangle Inequality in $L^1$ gives
        \[\norm{\chi_{E^c}\phi}_1=\norm{\phi-\chi_E\phi}_1=\norm{\phi(1-\chi_E)}_1\leq \norm{\phi - \psi}_1 + \norm{\psi - \chi_{E}\phi}_1<\varepsilon\]
    \end{proof}
        Back to the main proof of Part C, fix any $\varepsilon>0$, then by Lemma \ref{814 lemma}, $\phi$ induces some $E\in\cpt$ with $\norm{\chi_{E^c}\phi}_1<+\varepsilon$. By Lemma 8.4, $\chi_Kf\in \cc{\realn}\subseteq\ubc(\real^n)$. Uniform continuity of $\chi_Kf$ gives us the continuity of translations. Now for the same $\varepsilon>0$, there exists $r>0$, for every $w\in\realn$,
        \begin{equation}\label{814 condition 1}
            |w|<r\implies\norm{\tau_w\chi_Kf-\chi_Kf}_u<+\varepsilon
        \end{equation}
        Since $E\in\cpt$, it is bounded, and let $t$ be a small positive number such that for every $z\in E$,
        \[|tz|<t\cdot(1+\sup_{z\in E}|z|)<r\]
        There exists such a a $t$, namely $t = r2^{-1}(1+\sup_{z\in E}|z|)^{-1}$. And for this $t>0$, it follows that for every $z\in E$, \[\sup_{x\in K}\left|\tau_{tz}f-f\right|<+\varepsilon\]
        Since this holds for every $z\in E$, we write
        \[\sup_{x\in K,z\in E}\left|\tau_{tz}f-f\right|<+\varepsilon\]
        And
        \[|\phi(z)|\!\left[\sup_{x\in K,z\in E}|\tau_{tz}f-f|\right]<|\phi(z)|\varepsilon\]
        Monotonicity in $L^+(E,z)$ reads, for every $x\in K$,
        \[\int\limits_{z\in E}|\phi(z)(\tau_{tz}f-f)|dz\leq\int\limits_{z\in E}|\phi|\varepsilon dz=\varepsilon\norm{\chi_E\phi}_1\leq\varepsilon\norm{\phi}_1\]
        Since this holds for every $x\in\realn$,
        \begin{equation}\label{814 estimate 1}
            \sup_{x\in K}\biggl\{\int_{z\in E}|\phi(z)|\cdot|\tau_{tz}f-f|dz\biggr\}\leq\varepsilon\norm{\phi}_1
        \end{equation}
        Next, notice for every $t,z$, we have
        \[|\tau_{tz}f-f|\leq \norm{\tau_{tz}f}_u + \norm{f}_u\leq 2\cdot\norm{f}_u\]
        And the following holds $z\in E^c$ a.e,
        \[|\phi(z)|\cdot|\tau_{tz}f-f|\leq|\phi(z)|\cdot2\norm{f}_u\]
        Taking the integral, and applying the condition we imposed on $E$ from Lemma \eqref{814 lemma}, so that
        \[
            \int_{z\in E^c}|\phi(z)|\cdot|\tau_{tz}f-f|dz\leq2\norm{f}_u\int_{z\in E^c}|\phi(z)|dz\leq 2\norm{f}_u\varepsilon
        \]
        Taking the supremum of the above estimate, so
        \begin{equation}\label{814 estimate 2}
            \sup_{x\in K}\biggl\{\int_{z\in E^c}|\phi(z)(\tau_{tz}f-f)|dz\biggr\}\leq 2\norm{f}_u\varepsilon
        \end{equation}
        Combining Equations \eqref{814 estimate 1} and \eqref{814 estimate 2}. Applying the additivity of the supremum (of $x\in K$), since both members are finite,
        \[\sup_{x\in K}\,\biggl\{\int_E|\phi(z)|(\tau_{tz}f-f)dz+\int_{E^c}|\phi(z)|(\tau_{tz}f-f)dz\biggr\}<\varepsilon(2\norm{f}_u+\norm{\phi}_1)
        \]
    The left member above is equal to $\sup_{x\in K}|f\cnv \phi_t - af|$. Since $\varepsilon>0$ is arbitrary, this completes the proof of Part C.
\end{proof}

\end{document}