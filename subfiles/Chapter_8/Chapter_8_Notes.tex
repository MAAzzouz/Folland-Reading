% !TeX root = ../../main.tex
\documentclass[../../main.tex]{subfiles}
%Elements of Fourier Analysis
\begin{document}
\providecommand{\szz}{\mathcal{S}}
\providecommand{\ccinf}{C_c^\infty}
\section*{Notes on Chapter 8}
\begin{wts}
    If $\szz$ denotes the Schwartz Space, where
    \[
    \szz = \biggl\{f\in C^\infty,\, \norm{f}_{(N,\alpha)}<+\infty,\,\forall N\in\nat_0,\,\alpha\in\nat_0^n\biggr\}
    \]
    where the multi-index $\alpha$ norm with power $N$ is 
    \[
    \norm{f}_{(N,\alpha)}=\sup_{x\in\real^n}\biggl((1+|x|)^N\left|\partial^\alpha f(x)\right|\biggr)
    \]
    Show that $\ccinf\subseteq \szz$.
\end{wts}
\begin{proof}
    The outline for the proof is as follows,
    \begin{enumerate}
        \item First, we show that for any $f\in\ccinf$, if $\alpha = (0,\ldots,0)=0$, and for any $N\in\nat_0$, then 
        \[
        \sup_{x\in\real^n}(1+|x|)^N|f(x)|<+\infty
        \]
        \item Now for any multi-index $\alpha$ where $|\alpha| = 1$, so that $\alpha = (0,\ldots,1,\ldots,0)=e_j$, then 
        \[
        \dfrac{\partial}{\partial x_j}f(x)=\partial_j f\in \ccinf
        \]
        $\partial_j f\in C^\infty$ is obvious, so we only have to show that $\partial_j f$ has compact support, we will show that $\supp{\partial_j f}\subseteq \supp{f}$. Using the first part of the outline, we can conclude that
        \[
        \sup_{x\in\real^n}(1+|x|)^N|\partial/\partial x_j\, g|<+\infty
        \]
        \item Since $f\in\ccinf$, and all partial derivatives of $f$ are continuous, we can apply Clairaut's theorem to interchange the order of differentiation, and it suffices to show that for any function $f\in \ccinf$. Then for any multi-index $\alpha$, $\partial^\alpha f\in\ccinf$, and by the first step,
        \[
        \norm{f}_{(N,\alpha)}<+\infty,\quad\forall N,\alpha
        \]
        and we conclude that $\ccinf\subseteq \szz$.
    \end{enumerate}
    Let $f\in\ccinf$ be arbitrary, and denote $\supp{f}=K\in\cpt$. Since $(1+|x|)$ is continuous, on $K$, $\norm{\chi_K(1+|x|)}_\infty<+\infty$, it follows that for any power of $N$, $\norm{\chi_K(1+|x|)}^N=\norm{\chi_K(1+|x|)^N}<+\infty$. Hence $\norm{f}_{(N,\alpha)}<+\infty$ for every $N\in\nat_0$, $\alpha=0$.\\
    
    Next, fix a $j\in[1,n]$, and if $x\notin \supp{f}$. Then there exists some neighbourhood of radius $r>0$ about $x$ that does not intersect $\supp{f}$. Therefore
    \[
    \partial_j f(x) = \lim_{h\to 0}\dfrac{f(x+he_j)-f(x)}{h}=0
    \]
    Which means $\supp{f}^c\subseteq\{x,\,\partial_j f=0\}$, and
    \[
    \{x,\,\partial_j f\neq 0\}\subseteq \supp{f}\implies \supp{\partial_j f}\subseteq\supp{f}
    \]
    and $\partial_j f\in\ccinf$.\\
    
    Finally, a simple induction will show that $\norm{f}_{(N,\alpha)}<+\infty$, and $\ccinf\subseteq\szz$.
\end{proof}\newpage

\begin{wts}
    If $f\in\szz$, then for every $p\in[1,+\infty]$, $\partial^\alpha f\in L^p$.
\end{wts}
\begin{proof}
    Let $\alpha$ be fixed, and let $p\in[1,+\infty)$. And find an $N$ so large using the Archimedean Property that $N>np^{-1}$, then there exists an $a<n$ with $ap^{-1}<np^{-1}$. Notice also that $(1+|x|)\geq 1$, and
    \[
    N>ap^{-1}\implies (1+|x|)^{-N}\leq |x|^{-N}\leq |x|^{-ap^{-1}}
    \]
    Taking the $p$-th power on both sides gives us
    \[
    (1+|x|)^{-Np}\leq |x|^{-a}
    \]
    But Corollary 2.5 reads that if $a<n$, and if both functions are supported on a bounded set (because compact), then $|x|^{-a}\in L^{1}(m)$ by polar integration. Therefore $(1+|x|)^-N\in L^p$. It immediately follows that for this $N\in\nat_0$, and for any multi-index $\alpha$, we have
    \begin{align*}
    |\partial^\alpha f(x)|&\leq \norm{f}_{(N,\alpha)}\:(1+|x|)^{-N}\\
    |\partial^\alpha f(x)|^p&\leq \norm{f}_{(N,\alpha)}^p\:(1+|x|)^{-Np}\\
    &\implies \partial^\alpha f\in L^p
    \end{align*}
    If $p=+\infty$, then $|\partial^\alpha f(x)|=|\partial^\alpha f(x)|(1+|x|)^{-N}(1+|x|)^{N}$, and
    \[
    \norm{\partial^\alpha f}_\infty \leq \norm{f}_{(N,\alpha)}\:\norm{(1+|x|)^{-N}}_{u}<+\infty
    \]
    therefore $\partial^\alpha f\in L^p$ for every $p\in[1,+\infty]$.
\end{proof}
\newpage
\subsection{Notes on Frechet Space}
A Frechet space is a complete, Hausdorff, topological vector space whose topology is defined by a countable family of semi-norms.
\begin{itemize}
    \item Completeness: every Cauchy net converges
    \item Hausdorff, if $x\neq y\in X$, then there exists disjoint open sets $U_x, U_y\in\Tau$ with $x\in U_x$ and $y\in U_y$,
    \item Topological vector space: vector space equipped with a topology such that
    \begin{align*}
        f_1:X\times X\to X,&\quad (x,y)\mapsto (x+y)\\
        f_2:K\times X\to X,&\quad (\lambda, x)\mapsto \lambda x
    \end{align*}
    both maps $f_1$ and $f_2$ are continuous.
    \item A family of semi-norms, generate the topology if and only if (from Theorem 5.14), the topology $\Tau_X$ is generated sets of the form, for every $x\in X$, $\alpha\in A$, $\varepsilon>0$
    \[
    U_{x\alpha\varepsilon}=\biggl\{y\in X,\,p_\alpha(y-x)<\varepsilon\biggr\}
    \]
\end{itemize}
\newpage

\end{document}