\documentclass[../../main.tex]{subfiles}
\begin{document}
% \graphicspath{{./images}{subfiles/images/}}
\providecommand{\xx}{\mathbf{X}}
\problem{Limit of Norms}
\begin{wts}
    Show that $\lim_{p\to\infty}\norm{f}_p = \norm{f}_\infty$.
\end{wts}
\begin{proof}
    Fix any measure space $(\xx, \mcal, \mu)$ (no restriction). The proof will be divided into two parts.
    \begin{itemize}
        \item $\limsup_{q\to\infty} \norm{f}_q\leq \norm{f}_\infty$, and 
        \item $\norm{f}_\infty\leq\liminf_{q\to\infty}\norm{f}_q$
    \end{itemize}
    By log-convexity, suppose $f\in L^p\cap L^\infty$, and by interpolation, $f\in L^q$ for $q\in (p,\infty]$. 
    \begin{equation}\label{log-convex expression}
        q^{-1} = \lambda p^{-1} + (1-\lambda)r^{-1}\implies \norm{f}_q\leq\norm{f}_p^\lambda\cdot\norm{f}_\infty^{1-\lambda}
    \end{equation}
    So
    \begin{align*}
        \norm{f}_q\leq\biggl[\norm{f}_p^p\cdot\norm{f}_\infty^p \biggr]^{1/q}\cdot\norm{f}_\infty
    \end{align*}
    Let $\{q_n\}_{n\geq 1}$ be arbitrary with $q_n\to\infty$. And for sufficiently large $q_n$,
    \[
        \limsup_{n\to\infty}\norm{f}_{q_n}\leq\biggl[\norm{f}_p^p\cdot\norm{f}_\infty^p\biggr]^{1/(q_n)}\cdot\norm{f}_\infty
    \]
    With $\norm{f}_p^p\cdot\norm{f}_\infty^p\in[0,+\infty)\implies \biggl[\norm{f}_p^p\cdot\norm{f}_\infty^p\biggr]^{1/(q_n)}\to 1$, and the proof is complete.\\

    Next, normalize $f$ so that $\norm{f}_\infty = 1$, if this is impossible, then $f = 0$ a.e. Now, define
    \[
        A_{\varepsilon} = \bigset{|f| > (\norm{f}_\infty - \varepsilon)}
    \]
    fix $q_n\to\infty$ as in the previous part of the proof. At every $n$ sufficiently large,
    \[
        |f|^{q_n} > (\norm{f}_\infty - \varepsilon)^{q_n} \chi_{A_{\varepsilon}}\implies\norm{f}_{q_n}^{q_n}\geq (\norm{f}_\infty - \varepsilon)\cdot\mu(A_\varepsilon)^{1/{q_n}}
    \]
    $A_\varepsilon$ is a non-null set of finite measure. The finiteness comes from the last equation, and if $A_\varepsilon$ were null then it contradicts the definition of $\norm{f}_\infty$. Sending $n\to\infty$, then $\varepsilon\to 0$ finishes the proof.
\end{proof}
\begin{corollary}
    For the special case where $x\in\realn$, and $\lim_{p\to\infty}\norm{x}_p=\sup |x_i|$, take $\xx = \real^{\{1,\ldots, n\}}$, and $\mu$ be the counting measure on $\{1,\ldots, n\}$. Because the transpose of $x$ is a functional, etc. bla bla..
\end{corollary}

\end{document}