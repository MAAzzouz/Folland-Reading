\documentclass[../../main.tex]{subfiles}

\begin{document}
\subsection{Theorem 4.10}
\begin{wts}
    If $X_\alpha$ is Hausdorff for each $\alpha\in A$, then $X=\prod_{\alpha\in A}X_\alpha$ is Hausdorff.
\end{wts}
\newcommand{\pimap}[2]{\pi_{#1}({#2})}
\newcommand{\pinverse}[2]{\pi_{#1}^{-1}({#2})}
\begin{proof}
If two elements in $X$, $x\neq y$ then there exists some $\alpha\in A$ such that $\pimap{\alpha}{x}\neq\pimap{\alpha}{y}\in X_\alpha$, but this $X_\alpha$ is Hausdorff, then there exists two open, disjoint sets $V_x, V_y\subseteq X_\alpha$ such that
\begin{itemize}
    \item $x\in \pinverse{\alpha}{V_x}$, and $y\in \pinverse{\alpha}{V_y}$
    \item $\pinverse{\alpha}{V_x}\cap\pinverse{\alpha}{V_y} = \pinverse{\alpha}{V_x\cap V_y}=\varnothing$
    \item $\pinverse{\alpha}{V_x},\pinverse{\alpha}{V_y}\in \Tau_X$
\end{itemize}
Where for the last bullet point we used the fact that the product topology makes all the projection maps continuous. This proves that $X$ is Hausdorff.
\end{proof}

\end{document}