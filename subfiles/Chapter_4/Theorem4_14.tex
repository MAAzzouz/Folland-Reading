\documentclass[../../main.tex]{subfiles}

\begin{document}
\subsection{Theorem 4.14}
\begin{wts}
    Suppose that $A$ and $B$ are disjoint closed subsets of the normal space $X$, and let $\Delta = \{k2^{-n}: n\geq 1 \text{ and } 0<k<2^n\}$ be the set of dyadic rationals in $(0,1)$. There is a family $\{U_r:r\in\Delta\}$ of open sets such that
    \begin{enumerate}
        \item $A\subseteq U_r\subseteq B^c$ for every $r\in \Delta$ and 
        \item $\overline{U_r}\subseteq U_s$ for $r<s$.
    \end{enumerate}
\end{wts}
\renewcommand{\cl}[1]{\overline{#1}}
\newcommand{\ksmall}{(k-1)/2^N}
\newcommand{\kbig}{(k+1)/2^N}
\newcommand{\jiggle}[1]{\mathcal{J}({#1})}
\begin{proof}
    The goal of this proof is to show that for every $r\in \Delta$, there exists a open $U_r$ that satisfies the above. As usual for these types of proofs we will proceed by induction. We can divide the problem by 'layers' (as I will hereinafter explain).\\
    
    Let us suppose that for some $N\geq 1$ that all previous $U_r$ in previous layers have been constructed properly, meaning if $r = k/2^n$, then for every $1\leq n \leq N-1$, we have
    \[
    r = \dfrac{k}{2^n},\:1\leq n\leq N-1,\: 1\leq k\leq 2^{n-1}
    \]
    And by 'constructed properly', we mean that for each $U_r$,
    \begin{itemize}
        \item $A\subseteq U_r\subseteq B^c$ and 
        \item $U_r\in \Tau_X$
    \end{itemize}
    Then for this fixed layer $N\geq 1$, we only have to construct the $U_{k/2^N}$ for every odd $k$, this is because if $k$ is an even number, then $k=2j$ and $r = 2j/2^N = j/2^{N-1}$ and for this particular $U_r$ is already constructed. So for every odd $k = 2j+1$, the sets of the form $U_{(k-1)/2^N}$ and $U_{(k+1)/2^N}$ are already defined, and satisfy
    \[
    A\subseteq \cl{U}_{\ksmall}\subseteq U_{\kbig}\subseteq B^c
    \]
    For every $k-1\neq 0$ and $k+1\neq 1$. (We will consider these cases later). We claim that for every pair of open sets, $E_1, E_2\in\Tau_X$, then there exists some open set $G\in\Tau_X$ such that if $(E_1,E_2)\in H\subseteq (\Tau_X\times\Tau_X)$ where $H$ is defined as the set
    \[
    H = \left\{(E_1,E_2)\subseteq(\Tau_X\times\Tau_X): \cl{E_1}\cap E_2^c=\varnothing\right\}
    \]
    Then there exists some $G=\jiggle{E_1,E_2}\in\Tau_X$ such that
    \[
    E_1\subseteq\cl{E_1}\subseteq G\subseteq \cl{G}\subseteq E_2
    \]
    Now consider any any $(E_1,E_2)\in H$, then this pair induces a pair of disjoint sets $\cl{E_1}$ and $E_2^c$ since 
    \[
    \cl{E_1}\subseteq E_2\implies\cl{E_1}\cap E_2^c=\varnothing
    \]
    And by normality, there exists disjoint open sets $G_1$, $G_2$ such that
    \begin{itemize}
        \item $\cl{E_1}\subseteq G_1\in \Tau_X$
        \item $E_2^c\subseteq G_2\in\Tau_X$
        \item $G_1\cap G_2=\varnothing\implies G_1\subseteq G_2^c\subseteq E$
        \item Since $G_2^c$ is a closed set that contains $G_1$ as a subset, $\cl{G_1}\subseteq G_2^c\subseteq E$
    \end{itemize}
    It is at this point that we will make no further mention of $G_2$ (so we may discard the notion of $G_2$ in our minds). Let us now replace $G$ with $G_1$ then it is an easy task to verify that $G=G_1=\jiggle{E_1,E_2}$ has the required properties.\\
    
    Now define for every odd $k$, since $(U_{\ksmall},U_{\kbig})\in H$ (we note in passing that $\mathcal{J}$ is not a function as the set $G$ may not be unique).
    \[
    U_{k/2^N} = \mathcal{J}\left(U_{\ksmall},U_{\kbig}\right)
    \]
    Then, if $U_{\ksmall}$ and $U_{\kbig}$ is 'well constructed' we have
    \[
    A\subseteq\cl{U}_{\ksmall}\subseteq U_{\kbig}\subseteq B^c
    \]
    Therefore $U_{k/2^N} = \jiggle{U_{\ksmall},U_{\kbig}}$ sits 'right inbetween' the two sets so that
    \begin{itemize}
        \item $A\subseteq \cl{U}_{\ksmall}\subseteq U_{k/2^N}$ and
        \item $\cl{U}_{k/2^N}\subseteq U_{\kbig}\subseteq B^c$
    \end{itemize}
    Combining the above two estimates will give us a 'well constructed' $U_{k/2^N}$ for every $k-1\neq 0$ and $k+1\neq 1$. Now let us deal with the remaining pathological cases.\\
    
    If $k-1$ so happens to be $0$, then no $r\in\Delta$ satisfies $r = 0/2^N$, and we substitute 
    \[
    \cl{U}_0 = A,\quad \text{ or alternatively, } U_0 = A^o
    \]
    Then $U_0\in\Tau_X$, $\cl{U}_0=A\subseteq B^c$. It is at this point that we must mention that $0,1 \notin \Delta$, so $U_0$ and $U_1$ do not have to obey the rules we have laid out for $U_{r\in\Delta}$.\\
    
    Now if $k+1$ is equal to $2^N$ (this makes $r = (k+1)/2^N = 1$) we define
    \[
    U_1=B^c\in\Tau_X
    \]
    With this, for every $0\leq m \leq 2^N -1, U_{m/2^N}$ must staisfy 
    \[
    \cl{U}_{m/2^N}\subseteq B^c = U_1
    \]
    And the pair $(U_{\ksmall},U_{\kbig})\in H$ (even for when $N=1$, since $A = \cl{U}_0\subseteq U_1 = B^c$) and a corresponding $U_{k/2^N} = \jiggle{\cdot,\cdot}$ such that 
    \begin{itemize}
        \item $A\subseteq \cl{U}_{\ksmall}\subseteq U_{k/2^N}$
        \item $\cl{U}_{\kbig}\subseteq B^c$
    \end{itemize}
    Now as a final step, we complete the base case for when $N=1$. We would only have to construct for $k=1$, since 
    \[
    U_{1/2} = \jiggle{U_0,U_1} = \jiggle{A,B^c}
    \]
    Apply the induction step, and the proof is complete, at long last.
\end{proof}


\end{document}