\documentclass[../../main.tex]{subfiles}

\begin{document}
\subsection{Theorem 4.1}
\begin{wts}
Suppose that $A$ is a subset of $X$, let $\acc{A}$ be the set of accumulation points of $A$, then
\begin{equation}
    \overline{A}=A\cup\acc{(A)}
\end{equation}
and $A$ is closed if and only if $\acc{(A)}\subseteq A$.
\end{wts}
\begin{proof}
Suppose that $x\notin \overline{A}$, then $x\in (\overline{A})^c=A^{co}$, then $A^c\in\nb{x}$. But this means that $x\notin\acc{(A)}$, since there exists a neighbourhood of $x$ (in the form of $A^c$), such that 
\[
A\cap A^c\setminus\{x\}=A\cap A^c=\varnothing
\]
Also, $A\subseteq \overline{A}\implies (\overline{A})^c\subseteq A^c$ which means that
\[
x\notin \overline{A}\implies x\notin A
\]
Since $x\notin \overline{A}\implies x\notin A$ and $x\notin \acc{(A)}$,
\[
(\overline{A})^c\subseteq A^c\cap \acc{(A)}^c = (A\cup \acc{(A)})^c
\]
Now, if $x\notin \acc{(A)}\cup A$, then $x\notin\acc{(A)}$, therefore there exists some $U\in\nb{x}$ such that 
\[
A\cap U\setminus\{x\} = A\cap U=\varnothing
\]
Where for the second last equality we used the fact that $x\notin A\implies A\setminus\{x\}=A$, and taking complements gives us
\[
U\subseteq A^c
\]
And since $U\in\nb{x}$, then $x\in U^o\substeq A^{co}$ (since $U^o$ is an open subset of $A^c$). then
\[
x\in A^{co} = (\overline{A})^c\implies x\notin (\overline{A})^c
\]
Therefore $(A\cup\acc{(A)})^c\subseteq (\overline{A})^c$.
\end{proof}

\end{document}