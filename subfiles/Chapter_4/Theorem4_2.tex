\documentclass[../../main.tex]{subfiles}

\begin{document}
\subsection{Theorem 4.2}
\begin{wts}
    If $\Tau_X$ is a topology on $X$ and $\Epsilon\subseteq\Tau_X$ then $\Epsilon$ is a base for $\Tau_X$ if and only if for every 
    \[
    \forall U\in\Tau_X,\:U\neq\varnothing,\implies U=\bigcup_{V\in B}V
    \]
    Where $B$ is a subset of $\Epsilon$.
\end{wts}
\begin{proof}
    Suppose that $\Epsilon$ is a base, then fix any non-empty $U\in\Tau_X$, then for every $x\in U$, there exists a neighbourhood base for this $x$ and a member $V\in\Epsilon$ such that $x\in V_x\subseteq U$. Take the union over all $V_x$ and
    \[
    U\subseteq \bigcup_{x\in U}V_x
    \]
    But each $V_x\subseteq U$, so $U=\bigcup_{x\in U}V_x$, where $\{V_x\}\subseteq\Epsilon$.\\
    
    Conversely, if every non-empty $U$ is a union of members in $\Epsilon$ then fix any $x\in X$, we claim that we have a neighbourhood base in
    \[
    \{V\in\Epsilon,\:x\in V\}
    \]
    The reason is as follows
    \begin{itemize}
        \item $x$ belongs to every $E\in \{V\in\Epsilon,\:x\in V\}$ and
        \item For every open $U$, if $x\in U$ then there exists a union of members of $\Epsilon$ such that $U = \bigcup E_\alpha$, then $x\in U\iff\exists
        E_\alpha\in\{V\in\Epsilon,\: x\in V\}$ and
        \item Using this particular $E_\alpha\in\Epsilon$ that we just found, $x\in E_\alpha\subseteq U$, and we are done.
    \end{itemize}
\end{proof}

\end{document}