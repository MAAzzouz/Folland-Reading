\documentclass[../../main.tex]{subfiles}
%Elements of Fourier Analysis
\begin{document}
\providecommand{\szz}{\mathcal{S}}
\providecommand{\ccinf}{C_c^\infty}
\section*{Notes on Chapter 4}
\exercise{39}
\begin{wts}
    Show that every sequentially compact space is countably compact. With the following definitions:
    \begin{itemize}
    \item Sequential compactness: every sequence has a convergent subsequence.
    \item Countably Compact: every countable open cover has a finite subcover.
    \end{itemize}
\end{wts}

\begin{proof}
    We will proceed by proving the contrapositive. Suppose $\xx$ admits a countable open cover where no finite subcover exists. Intuitively, we want to produce a sequence that 'leads to nowhere'.\\
    
    Denote this 'bad' open cover by $\{U_n\}_{n\geq 1}$. Since $\{1,2,\ldots,n\}$ is finite, we can choose a sequence 
    \[
        x_n\in U_n \setminus \left(\bigcup U_{j\leq n-1}\right)\neq\varnothing
    \]
    where $U_0=\varnothing$ for convenience.  If $U_j$ is empty for any $j\geq 1$, we can discard this $U_j$ from our open cover, and if what remains is a finite
\end{proof}\newpage

\end{document}