\documentclass[../../main.tex]{subfiles}
%Elements of Fourier Analysis
\begin{document}
\providecommand{\szz}{\mathcal{S}}
\providecommand{\ccinf}{C_c^\infty}
\section*{Notes on Chapter 4}
\subsection*{General Topology Proofs}

\begin{definition}\label{chp4:interior-definition}
    $A^o$ is defined to be the largest open subset of $A$, 
    \[
        A^o = \bigcup_{\text{$U$ open, }U\subseteq A} U
    \]
\end{definition}
\begin{corollary}\label{chp4:interior-subset}
    The union of subsets of $A$ is again a subset of $A$, therefore \Cref{chp4:interior-subset} implies $A^o\subseteq A$ for any $A\subseteq \xx$. 
\end{corollary}
%%%%
\begin{definition}\label{chp4:closure-definition}
    and $\cl{A}$ is the smallest closed superset of $A$,
    \[
        \cl{A} = \bigcap_{\mathclap{\text{$K$ closed, } A\subseteq K}} K
    \]
\end{definition}
%%%%
\begin{wts}\label{chp4:flipping-interior-to-closure}
    The complement of the closure is the interior of the complement, or equivalently: $(\cl{A})^c = A^{co}$
\end{wts}
\begin{proof}
    Taking complements, and the substitution $U = K^c$ reads
    \begin{align*}
        (\cl{A})^c &= \left(\bigcap_{{\text{$K$ closed, } A\subseteq K}} K \right)^c\\[4ex]
        &= \bigcup_{{\text{$K$ closed, } K^c\subseteq A^c}} K^c\\[4ex]
        &= \bigcap_{{\text{$U$ open, } U\subseteq A^c}} U\\[4ex]
        &= A^{co}
    \end{align*}
\end{proof}
\begin{remark}
    Personally, I remember this as pushing the complement inside and flipping the bar to a $c$!
\end{remark}
\newpage


\begin{definition}\label{chp4:neighbourhood-definition}
    A neighbourhood of $x\in\xx$ is a set $U\subseteq\xx$ where $x\in U^o$. The set of neighbourhoods for a point $x\in\xx$ will sometimes be denoted by $\N(x)$.
\end{definition}
\begin{wts}
    If $W = \bigset{x\in\xx, \: \text{there exists a neighbourhood $U$ of $x$,}\: U\subseteq A}$, then $W = A^o$.
\end{wts}
\begin{proof}
    If $x\in A^o$, then $A$ is a neighbourhood of $x$, and $A\subseteq A$, so $x\in W$. Conversely, if $x$ is a member of $W$, it has a neighbourhood $U\subseteq A$ (not necessarily open). By monotonicity of the interior,
    \[
        x\in U^o\subseteq A^o
    \]
    and $x\in A^o$.
\end{proof}
It is easy to see that $A$ is open $\iff A^o = A \iff A$ is a neighbourhood of itself. 
\begin{itemize}
    \item The first equivalence follows from:
    \[
        E\subseteq\xx\implies E^o\subseteq E
    \]
    and if $A$ is an open set, it is an open subset of itself, by \Cref{chp4:interior-subset} $A\subseteq A^o$. If $A^o = A$, then it suffices to show that $A^o$ is open. Which it is, since it is the arbitrary union of open sets.
    \item To prove the second equivalence: suppose $A^o = A$, then each $x\in A$ has a neighbourhood contained (as a subset) in $A$, namely $A$ itself. (This statement is hard to parse, the reader is encouraged to really work through this and be honest).
    \[
        x\in A^o\subseteq A\implies A\subseteq A^o
    \]
    so $A$ is a neighbourhood of itself. Conversely, if $A\subseteq A^o$, then $A = A^o$, since the reverse inclusion follows immediately from \Cref{chp4:interior-subset}.
\end{itemize}
\newpage

We will now discuss the closure of a set.
\begin{wts}\label{chp4:closure-adherent}
    Let $A\subseteq X$, if $W = \bigset{x\in\xx,\: \text{ every neighbourhood $U$ of $x$, }\: U\cap A\neq\varnothing}$, then $\cl{A}=W$
\end{wts}
\begin{proof}
    Suppose $x\notin W$, then there exists a neighbourhood $U$ of $x$ where
    \[
        U\cap A=\varnothing\iff U\subseteq A^c
    \]
    this is exactly the definition of the interior of $A^c$, so $x\in A^{co}$ and recall (from \Cref{chp4:flipping-interior-to-closure}) that $(\cl{A})^c = A^{co}$, so $x\notin \cl{A}$. For the reverse inclusion, read the proof backwards, by flipping $\forall\to\exists$ within the set, and we see that
    \[
        W^c = A^{co} = (\cl{A})^c
    \]
\end{proof}





\newpage
\subsection*{Urysohn's Lemma Notes}
Notes on the construction of the countable 'onion' sequence within a normal space $\xx$.\\

If $\xx$ is a normal space, and $A$ and $B$ are disjoint closed subsets, then we can easily find an open $U$ with
\begin{equation}\label{eq:UrysohnLemma-Seashells}
    A\subseteq U \subseteq \cl{U}\subseteq B^c
\end{equation}
We say that $U$ hides in $B^c$ if the closure of $U$ is contained in $B^c$. Define $\Delta_n = \bigset{k2^{-n},\: 1 < k < 2^{n}}$, so that $\Delta_n\subseteq(0,1)$ for all $n\geq 1$. Notice 
\[
    \Delta_1\supseteq \cdots\supseteq \Delta_n\supseteq \Delta_{n+1}
\]
and the even indices for $\Delta_{n+1}$ are contained in $\Delta_n$. Suppose $\Delta_n$ is well defined, it suffices to choose the odd indices for $\Delta_{n+1}$. If $r = j2^{-(n+1)}$, where $j$ is odd, then $r$ sits in between precisely two elements in $\Delta_n\cup\{0,1\}$. If $r$ sits between an endpoint, then define $\cl{U_0} = A$, and $B^c = U_1$. And denote the closest left and neighbours by $s$, $t$ respectively. If $s<r<t$, it is clear that $\cl{U_s}$ and $U_t^c$ are disjoint closed sets.\\

Use the 'normal space' construction to obtain an superset of $\cl{U_s}$ that hides in $U_t$, denote this open set by $U_r$, and similar to \Cref{eq:UrysohnLemma-Seashells}
\[
    \cl{U_s}\subseteq U_r\subseteq \cl{U_r}\subseteq U_t
\]
Now that the construction of this sequence is complete, we wish to prove Urysohn's Lemma. Let $A$ and $B$ be disjoint closed sets. And define 
\[
    f(x) = \inf\bigset{r\in\Delta\cup\{1\},\: x\in U_r}
\]
where $U_1 = \xx$. So that $0\leq f(x)\leq 1$ is immediate. If $x\in A$, then $x$ is in all $U_r$, and by density of $\Delta\subseteq(0,1)$, we have $f(x)=0$. Conversely, if $x\in B$ then $x\notin U_r$ for all $r\in\Delta$, if $E$ denotes the indices in $\Delta$ where $x\in U_s$ when $s\in E$,
\begin{equation}\label{inequality shortcut infimum intervals}
    (-\infty,\: r)\subseteq E^c\iff E\subseteq [r,\: +\infty)\iff\inf(E)\geq r
\end{equation}
Send $r\to 1$ and $f(x) = 1$. Thus $f|_{A}=0$ and $f|_{B}=1$.\\

To show continuity, it suffices to show that the inverse images of the open half $\bigset{(x > \alpha),\: (x < \alpha)}_{\alpha\in\real}$ lines are indeed open in $\xx$. Let $\alpha$ be fixed. And if $x\in \{f<\alpha\}$, we can 'wiggle' the infimum towards the right (towards $\alpha$), and using density of $\Delta$ within $(0,1)$, there exists a $r\in E$ that satisfies $f(x) < r < \alpha$. This is equivalent to 
\[
    x\in \bigcup_{r<\alpha} U_r
\]
If there exists an $r<\alpha$ st $x$ belongs to $U_r$ as an element, then $f(x)\leq r < \alpha$.\\

If $f(x) > \alpha$, then $(-\infty,\: \alpha)\subseteq E^c$, by  \Cref{inequality shortcut infimum intervals}. Suppose $\alpha<1$, otherwise $\{f>\alpha\}=\varnothing$. Wiggle $f(x)$ to the left and obtain an $r\in\Delta$, $\alpha<r<f(x)$ with $x\notin U_r$. By density again, take any $s<r$ by a small amount (st $s>\alpha$, $s\in \Delta$), and
\[
    \cl{U}_s\subseteq U_r\iff U_r^c\subseteq\cl{U}_s
\]
so that $x\in \cl{U}_s^c$ for some $s>\alpha$. This is equivalent to 
\[
    x\in\bigcup_{s>\alpha}\cl{U}_s^c
\]
Conversely, if $x\notin \cl{U}_s^c$ for some $s>\alpha$,  since $\{U_r\}$ (thus $\{\cl{U}_r\}$) is increasing, and $x\notin U_r$ for every $r\leq s$. Hence,
\[
    (-\infty,\: s]\subseteq E^c\iff E\subseteq (s,\: +\infty)\iff f(x)\geq s>\alpha
\]

\end{document}