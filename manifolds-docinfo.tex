\author{Me}
\title{Manifolds}


% Page Style
\pagestyle{fancy}
\fancyhead{}\fancyfoot{}
\setlength{\headheight}{52pt} % At least 51.7pt to prevent error
\renewcommand{\footrulewidth}{0.4pt}% default is 0pt so no footer line

% Headers
% Headers% 
\fancyhead[L]{%
\hyperlink{toc}{\textbf{Manifolds}}
}
\fancyhead[R]{%
\small\rightmark
}

\fancyfoot[L]{\small\textbf{Updated \today\, at \,\currenttime}}
\fancyfoot[R]{\small\textbf{Page \thepage}}

% Alternate Date-Time, if the above does not work.
%
% Also see 'latexmkrc' to adjust timezone. That file has to be placed in the root directory. If we have /Lab7/main.tex, then it should be placed /latexmkrc, instead of /Lab7/latexmkrc
%
% Alternate latexmkrc: $ENV{'TZ'}='America/Toronto';
%
% See https://www.overleaf.com/learn/latex/Questions/How_do_I_make_%5Ctoday_display_the_date_according_to_my_time_zone%3F#.Vt4dR6grLow
% \fancyfoot[L]{\textbf{Updated \today\, at \,\DTMcurrenttime\: \hfill Page \thepage}}


% Additional shortcuts
\providecommand{\half}{\mathbb{H}}
\providecommand{\halfn}{{\mathbb{H}^n}}
\providecommand{\halfm}{{\mathbb{H}^m}}
% \providecommand{\df}{\mathcal{D}}


% Shortcuts

% Relations
\providecommand{\htp}{\simeq}    % Homotopy
\providecommand{\phtp}{\simeq_p} % Path Homotopy


% Path and PathClass
\NewDocumentCommand{\Path}{oo}{
      {\operatorname{Path}}\IfNoValueF{#1}{({#1}\IfNoValueF{#2}{,\: {#2}})}
    }
\NewDocumentCommand{\Loop}{o}{
  {\operatorname{Loop}}\IfNoValueF{#1}{({#1})}
}
\NewDocumentCommand{\PathClass}{oo}{
      {\operatorname{PathClass}}\IfNoValueF{#1}{({#1}\IfNoValueF{#2}{,\: {#2}})}
    }
    
\providecommand{\fgroup}[2]{\pi_{1}({#1},{#2})}
\providecommand{\inbhd}{\overset{\mathring}{\in}} % in neighbourhood
\providecommand{\opensubset}{\overset{\mathring}{\subseteq}} % open subset
\providecommand{\opensupset}{\overset{\mathring}{\supseteq}} % open superset

\providecommand{\Alt}[1]{\operatorname{Alt}(#1)}

\NewDocumentCommand{\Ul}{moo}{
  \IfNoValueTF{#3}
    {{#1}_{\underline{#2}}}
    {{#1}_{#2,\underline{#3}}}
}

\providecommand{\hook}{\mathop{\lrcorner}} % x into omega, holding the first coordinate at x
\providecommand{\vField}{\mathfrak{X}} % vector field 
\providecommand{\cvField}{\mathfrak{X}} % covector field
\def\isum{\mathop{\sum\nolimits{'}}} % increasing sum