\documentclass[../main-manifolds.tex]{subfiles}

\begin{document}

\fchapter{B: Abstract Algebra}\newpage

\topheader{Groups}

\begin{definition}[Semigroups, Monoids]\label{hungerford-chp1:semigroups-monoids-definition}
    A non-empty set $G$ equipped with an associative binary operation $G\times G\to G$ is called a semigroup. For every $a,b,c\in G$, we have
    \begin{equation}\label{hungerford-chp1:associativity-semigroup}
        a(bc) = (ab)c 
    \end{equation}
    A \emph{monoid} is a semigroup $G$ which contains a \emph{two-sided identity} element $e\in G$ such that $ae = ea$ for all $a\in G$. (not necessarily unique)
\end{definition}

Monoids admit unique two-sided identities.
\begin{lemma}[Monoids: unique identity]\label{hungerford-chp1:monoid-unique-identity}
    Let $e$ and $i$ be two-sided identities for a monoid $G$, then
    \begin{proof}
    \[
        e = ei = i
    \]    
    \end{proof}
\end{lemma}
    



\begin{definition}[Group]\label{hungerford-chp1:group-definition}
    A semigroup $G$ is a group if every element $a\in G$ admits a two-sided inverse $a^{-1}$. (not necessarily unique)
    \[
        aa^{-1} = a^{-1}a = e
    \]
\end{definition}

\begin{wts}[Properties of Groups (Hungerford: Theorem 1.2)]\label{hungerford-chp1:theorem1.2}
    Let $G$ be a group with identity $e$, which is unique by \cref{hungerford-chp1:monoid-unique-identity}. Then
    \begin{enumroman}
        \item $c\in G$ and $cc=c$ implies $c=e$. 
        
        \item Left/Right cancellation: 
        \[
            \begin{cases}
                ab=ac\implies b=c\\
                ba=ca\implies b=c
            \end{cases}
        \]
        
        \item If $a\in G$, its two-sided inverse is unique. 
        
        \item Let $a\in G$, then the inverse of its two-sided inverse (uniqueness guaranteed by iii), is $a$ itself; or $(a^{-1})^{-1}=a$.
        
        \item If $a,b\in G$, then the following equations in $x,y$ admit unique solutions
        \[
            \begin{cases}
                ax=b\\
                ya=b
            \end{cases}
        \]
    \end{enumroman}
\end{wts}
\begin{proof}[Proof of \Cref{hungerford-chp1:theorem1.2}]
    \begin{enumerate}[label={{Proof of Part} (\roman*): },leftmargin=*]
        \item[]
        \item \[cc=c\implies (cc)c^{-1}=cc^{-1}\implies c(cc^{-1})=e\implies ce=c=e\]
        \item First claim:
            \begin{multline*}
                ab=ac\implies a^{-1}(ab)=a^{-1}(ac)\\
                \implies (a^{-1}a)b=(a^{-1}a)c\implies eb=ec\implies b=c
            \end{multline*}
            Second claim is the same, just cancel from the right using $aa^{-1}=e$ and associativity.
        \item Suppose $b$ and $c$ are two-sided inverse for $a$, it follows from Part ii that 
        \[
            ab=ac\implies b=c=a^{-1}
        \]
        \item From Part iii, the two-sided inverses of group elements exist and are unique, and $a^{-1}a=aa^{-1}$ so $a$ is an inverse for $a^{-1}$, and it is the only inverse.
        \item First equation: write $ax=b=a(a^{-1}b)$, left-cancelling reads $x=a^{-1}b$, uniqueness follows from Part ii. Second equation is similar.
    \end{enumerate}
\end{proof}

\begin{lemma}[Group: equality lemma]\label{hungerford-chp1:equality-lemma}
     For any pair of elements $a,b\in G$, $a=b\iff ab^{-1}=e$.
    \begin{proof}
    ($\implies$): $a=b\implies ab^{-1}=bb^{-1}=e$. ($\impliedby$): $ab^{-1}=e\implies a(b^{-1}b)=eb\implies a=eb=b$.
    \end{proof}
\end{lemma}


\begin{wts}[Semigroup: upgrade to group I (Hungerford Proposition 1.3)]\label{hungerford-chp1:theorem1.3}
    Let $G$ be a semigroup, $G$ is also a group iff both of the conditions below hold
    \begin{itemize}
        \item Existence of a left-identity: there exists $e\in G$ for every $a\in G$, $ea=a$.
        \item Existence of left-inverses: for every $a\in G$, there exists a $a^{-1}\in G$ with $a^{-1}a=e$, where $e$ is any left-identity element.
    \end{itemize}
\end{wts}
\begin{proof}
    ($\impliedby$) is trivial. Suppose both conditions hold, notice the proof for \Cref{hungerford-chp1:theorem1.2} Part (i) we only used left-cancellation. $cc=c\implies e$. To prove $a^{-1}$ is also a right-inverse for $a$, we can force it as follows:
    \[
        (aa^{-1})(aa^{-1})=a(a^{-1}a)a^{-1}=aea^{-1}=e\implies aa^{-1}=e
    \]
    and $a^{-1}$ is also a right-inverse, so every element $a\in G$ admits a two-sided inverse denoted by $a^{-1}$. To show $e$ is also a right-identity for any arbitrary element $a\in G$, 
    \begin{align*}
        ae&=a(a^{-1}a) && \text{left inverse}\\
        &=(aa^{-1})a && \text{associativity}\\
        &=ea && \text{right inverse}\\
        &=a && \text{left identity}
    \end{align*}
\end{proof}

\begin{wts}[Semigroup: upgrade to group II (Hungerford Proposition 1.4)]\label{hungerford-chp1:prop1.4}
    Let $G$ be a semigroup, $G$ is a group iff for every pair of elements $a,b\in G$, the equations in $x$ and $y$
    \begin{equation}\label{hungerford-chp1:prop1.4-equations}
        \begin{cases}
            ax=b\\
            ya=b
        \end{cases}
    \end{equation}
    have solutions (not necessarily unique).
\end{wts}
\begin{proof}
    If $G$ is a group, the existence of the solutions to \cref{hungerford-chp1:prop1.4-equations} follow from \Cref{hungerford-chp1:theorem1.2}. We will attempt the contrapositive. Suppose $G$ has no left identity, for every $e\in G$ we can always find an element $a\in G$ such that $ea\neq a$, but this is precisely the (first) equation for $a=a$ and $b=a$. \\

    Now suppose $G$ has a left identity element (not necessarily unique). Fix $e\in G$ as any left-identity, and suppose there is an element $a\in G$ with no left inverse, so for every $b\in G$, $ba\neq e$. But $b$ is precisely the solution to the (second) equation with parameters $a=a$ and $b=e$. The negation of \Cref{hungerford-chp1:theorem1.3} is precisely the negation of \Cref{hungerford-chp1:prop1.4}, and the proof is complete.
\end{proof}

\begin{wts}[Hungerford Theorem 1.5]\label{hungerford-chp1:theorem1.5}
    Let $R/\sim$ be an equivalence relation on a group $G$, such that it 'preserves' the group multiplication. More precisely, 
    \[
        \begin{cases}
            a_1\sim a_2\\
            b_1\sim b_2
        \end{cases}\implies a_1b_1\sim a_2b_2
    \]
    Then the set $G/R$ of all equivalence classes of $G$ under $R$ is a monoid under the binary operation defined by 
    \begin{equation}\label{hungerford-chp1:theorem1.5-binary-operation}
        (\cl{a})(\cl{b})=\cl{ab}\quad\quad\substack{\text{reads: the product of two classes is the class}\\  \text{containing the product of any pair}\\ \text{ of elements from the two classes}}
    \end{equation}
    where $\cl{a}$ denotes the equivalence class containing $a$. If $G$ is a group, so is $G/R$, if $G$ is an abelian group, so is $G/R$.
\end{wts}
\begin{proof}
    First, notice the binary operation in \Cref{hungerford-chp1:theorem1.5-binary-operation} is well defined. It is independent of the equivalence class representatives chosen, as we have restriction on $R$ that 'forces' the operation on $G/R$ to be well defined. Indeed, let $\cl{a}$ and $\cl{b}$ be elements of $G/R$, if $a_1, a_2\in \cl{a}$, and $b_1,b-2\in\cl{b}$, by definition of $R$:
    \[
        a_1\sim a_2\qqtext{and}b_1\sim b_2
    \]
    by \Cref{hungerford-chp1:theorem1.5-binary-operation}, $a_1b_1\sim a_1b_2\implies \cl{a_1b_1}=\cl{a_2b_2}$.\\

    Associativity is proven similarly, fix $\cl{a},\cl{b},\cl{c}\in G/R$, we pass the argument to any of the representatives of the three classes, so
    \[
        (\cl{a}\cl{b})\cl{c}\defined \cl{ab}\cl{c} =  \cl{(ab)c}=\cl{a(bc)}\defined\cl{a}\cl{bc}=\cl{a}(\cl{b}\cl{c})
    \]
    Pass the argument to the representatives, let $e$ denote the identity element in $G$, it is easily shown that $\cl{e}$ is the identity element in $G/R$, similarly for two-sided inverses and commutativity of the binary operation.
\end{proof}

\topheader{Homomorphisms}

\begin{definition}[Homomorphism]\label{hungerford-chp1:homomorphism-definition}
    Let $G$ and $H$ be semigroups, $f:G\to H$ is a semi-group \emph{homomorphism} if for all $a,b\in G$, 
    \begin{equation}\label{hungerford-chp1:homomorphism-equation}
        f(ab)=f(a)f(b)
    \end{equation}
\end{definition}

\begin{definition}[Monomorphism]\label{hungerford-chp1:monomorphism-definition}
    Injective homomorphism.
\end{definition}

\begin{definition}[Epimorhpism]\label{hungerford-chp1:epimorphism-definition}
    Surjective homomorphism.
\end{definition}

\begin{definition}[Isomorphism]\label{hungerford-chp1:isomorphism-definition}
    Bijective homomorphism.
\end{definition}

\begin{definition}[Endomorphism]\label{hungerford-chp1:endomorphism-definition}
    Homomorphism for which the domain and codmain (not the range) are equal; i.e $H = G$.
\end{definition}

\begin{definition}[Automorphism]\label{hungerford-chp1:automorphism-definition}
    Bijective endomorphism.
\end{definition}

\begin{definition}[Kernel of a homomorphism]\label{hungerford-chp1:homomorphism-kernel}
    The kernel of $f\in\Hom[G,H]$ is defined
    \begin{equation}\label{hungerford-chp1:homomorphism-kernel-equation}
        \Ker{f} = \bigset{a\in G,\: f(a)=e\in H}
    \end{equation}
    as the set of elements in $G$ that get sent to the identity of $H$.
\end{definition}

\begin{wts}[Hungerford Theorem 2.3]\label{hungerford-chp1:theorem2.3}
    Let $G$ and $H$ be groups and let $f\in \Hom[G,H]$. Denote the identity elements of $G$ and $H$ by $e_G$ and $e_H$
    \begin{enumroman}
        \item $f(e_G)=e_H$,
        \item $f(a^{-1})=\qty(f(a))^{-1}$ for every $a\in G$.
        \item $f$ is a monomorphism iff $\ker f = \{e_G\}$,
        \item $f$ is an isomorphism iff there exists a homomorphism $f^{-1}: H\to G$ that is also a two-sided inverse for $f$. In symbols:
        \begin{equation}\label{hungerford-chp1:theorem2.3-isomorphism-equation}
            f\circ f^{-1}=\id{H}\qqtext{and}f^{-1}\circ f=\id{G}
        \end{equation}
    \end{enumroman}
\end{wts}
\begin{proof}[Proof of \Cref{hungerford-chp1:theorem2.3}]
    \begin{enumerate}[label={Proof of Part (\roman*): },leftmargin=*]
        \item[]
        \item We will use \Cref{hungerford-chp1:theorem1.2} (i). Since $f(e_G)=f(e_Ge_G)=f(e_G)f(e_G)$ in $H$, we see that $f(e_G)=H$ and $e_G\in \Ker{f}$
        \item Let $a\in G$ be arbitrary, using Part (i), we can 'pass the multiplication' between $f(a)$ and $f(a^{-1})$ into $G$, 
        \[
            f(a)f(a^{-1})=f(e_G)=e_H\implies f(a^{-1})=\qty(f(a))^{-1}
        \]
        \item Suppose $\ker{f}={\idg}$. Let $a,b\in G$ such that $f(a)=f(b)$. The equality lemma \Cref{hungerford-chp1:equality-lemma} tells us $\qty(f(a))^{-1}=f(b)$ and $b=a^{-1}$, so $a=b$ by the Lemma again; $f$ is injective.\\

        Conversely, suppose $f$ is injective, Part (i) tell us $\{\idg\}\subseteq \ker{f}$. Suppose $a\in \ker{f}\subseteq G$, but $\idg\in\ker{f}$, so $f(a)=f(\idg)=\idh$ forces $a=\idg$, and $\ker{f}=\{\idg\}$.
        \item ($\impliedby$) is trivial since the existence of a (functional) two-sided inverse is equivalent to bijectivity. Suppose $f$ is an isomorphism, and define $f^{-1}$ as its two-sided (functional) inverse, it suffices to show that $f^{-1}\in\Hom[H,G]$. Fix $f(a)$ and $f(b)$ as arbitrary elements in $H$. We can do this because $f$ is a bijection, so every element in $H$ has a unique 'representative' in $G$. 
        \[
            f^{-1}\qty(f(a))\,f^{-1}\qty(f(b))=ab=f^{-1}\qty(f(ab))=f^{-1}\qty(f(a)f(b))
        \]
    \end{enumerate}
\end{proof}

\begin{definition}[Subgroup {$H< G$}]\label{hungerford-chp1:subgroup-definition}
    Let $G$ be a group. If $H$ is a non-empty subset of $G$ that is closed under group operations, then it is a \emph{subgroup of $G$}. We write $H<G$.\\

    The \emph{trivial subgroup} of $G$ is $\{e\}$ and consists of one element. $H$ is called a \emph{proper subgroup} if $H\neq G$ and $H\neq \{e\}$.
\end{definition}

\begin{wts}[Hungerford Exercise 9]\label{hungerford-chp1:exercise9}
    Let $f\in\Hom[G,H]$, $A<G$ and $B<H$,
    \begin{enumroman}
        \item $\Ker{f}$ is a subgroup of $G$,
        \item $f(A)$ is a subgroup of $H$,
        \item $f^{-1}(B)$ is a subgroup of $G$.
    \end{enumroman}
\end{wts}

\topheader{Subgroups}

\begin{wts}[Subgroup criteria (Hungerford Theorem 2.5)]\label{hungerford-chp1:theorem2.5}
    A non-subset $H\subseteq G$ is a subgroup iff for any $a$, $b$ in $H$, $ab^{-1}\in H$.
\end{wts}
\begin{proof}
    ($\impliedby$): Choose $a=b$, then $aa^{-1}=e\in H$ acts as the two-sided identity in $H$, and $ea^{-1}=a^{-1}\in H$ for every $a\in H$. So $H$ is a subgroup by \Cref{hungerford-chp1:theorem1.3}. ($\implies$): If $H$ is a subgroup, then \Cref{hungerford-chp1:theorem1.3} tells us $b^{-1}\in H$ for every $b\in H$, hence $ab^{-1}\in H$ for elements $a,b\in H$.
\end{proof}
\begin{corollary}[Hungerford Corollary 2.6]\label{hungerford-chp1:corollary2.6}
    If $G$ is a group and $\{H_i,\: i\in I\}$ is a nonempty family of subgroups, then their intersection $H\defined \cap_{i\in I}H_i$ is again a subgroup in $G$.
\end{corollary}
\begin{proof}
    Let $a,b\in H$, then $ab^{-1}\in H_i$ for every $i\in I$, hence $ab^{-1}\in \cap_{i\in I}H_i=H$, and $H$ is a subgroup by \Cref{hungerford-chp1:theorem2.5}.
\end{proof}

\begin{definition}[Subgroup generated by $A\subseteq G$]\label{hungerford-chp1:subgroup-generated-definition}
    Let $A$ be a subset of $G$, the \emph{subgroup generated by $A$} is the smallest subgroup $H<G$ that contains $A$ as a subset, denoted by $\abrackets{A}$.\\

    \Cref{hungerford-chp1:theorem2.5} gives us an explicit formula for $H$,
    \[
        H = \bigcap_{\mathclap{\substack{H_i<G\\ A\subseteq H_i}}}H_i
    \]

    If $A$ is finite, and $H=\abrackets{A}$ is said to be \emph{finitely generated}. We also write 
    \[
        H = \abrackets{a_1,\ldots,a_n}=\abrackets[\bigg]{\{a_1,\ldots,a_n\}}
    \]
    
    If $A$ consists of one element, $\{a\}=A$, then $\abrackets{a}=\abrackets{\{a\}}$ is called the \emph{cyclic group generated by $a$}.
\end{definition}
\begin{wts}[Hungerford Theorem 2.8]\label{hungerford-chp1:theorem2.8}
    If $G$ is a group and $A\subseteq G$ is a non-empty subset, the subgroup generated by $A$ is precisely the collection of all finite products (powers included), or
    \[
        \abrackets{A}=\bigset{a_1^{n_1}a_2^{n_2}\cdots a_t^{n_t},\: a_{i}\in A,\: n_i\in\mathbb{Z}}
    \]
    If $A=\{a\}$, then $\abrackets{A}=\bigset{a^k,\: k\in \mathbb{Z}}$.
\end{wts}
\begin{proof}
    Let $W=\bigset{a_1^{n_1}a_2^{n_2}\cdots a_t^{n_t},\: a_{i}\in A,\: n_i\in\mathbb{Z}}$. We first show $W$ is a subgroup of $G$. Indeed, if $n_1=n_2=\cdots=n_t=0$, then $a_1^{n_1}a_2^{n_2}\cdots a_t^{n_t}=e$. And for any $b\in W$, 
    \[
        b = a_1^{n_1}a_2^{n_2}\cdots a_t^{n_t}\implies b^{-1}=a_t^{-n_t}\cdots a_2^{-n_2}a_1^{-n_1}
    \]
    where each $-n_i\in \mathbb{Z}$, so $b^{-1}\in H$ as well. $\abrackets{A}$ is the smallest subgroup containing $A$, so $\abrackets{A}\subseteq W$. Conversely, fix an element $b\in W$, so $b$ has the form
    \[
        b = a_1^{n_1}a_2^{n_2}\cdots a_t^{n_t},\: a_i\in A,\: n_i\in \mathbb{Z}
    \]
    and a simple induction will show each $a_i^{n_i}\in \abrackets{A}$ for $1\leq i\leq t$, so $b\in \abrackets{A}$ and $\abrackets{A}=W$.
\end{proof}

\begin{definition}[Lattice of subgroups]\label{hungerford-chp1:subgroup-lattice}
    Let $\{H_i\}_{i\in I}$ be a collection of subgroups of $G$, then 
    \[
        \glb\{H_i\}=\bigcap_{i\in I}H_i,\quad \lub=\abrackets[\bigg]{\bigcup_{i\in H}H_i}
    \]
    The collection of subgroups of $G$ is a \emph{complete lattice}.
\end{definition}
\topheader{Cyclic groups}
The proof for the following is straight-forward, the book separates the case into $H=\abrackets{0}$ and $H\neq\abrackets{0}$, then $H$ contains a non-zero element $h\neq 0$, then $|h|\in \nat^+$ is an element in $H$ as well, so $H$ contains a least positive element by invoking the Well Ordering Property. For the second half of the proof, we force $r=0$ by the Division Algorithm.

\begin{definition}[Order of a subgroup $H<G$]\label{hungerford-chp1:order-subgroup-definition}
    The order of a subgroup $H$ is its cardinality $|H|$.
\end{definition}

\begin{definition}[Order of an element $a\in G$]\label{hungerford-chp1:order-element-definition}
    The order of an element $a\in G$ is the order of $|\abrackets{a}|$.
\end{definition}

\begin{wts}[Hungerford Theorem 3.1]\label{hungerford-chp1:theorem3.1}
    Let $\ints$ be equipped with its usual addition operation. Then every subgroup $H<\ints$ is cyclic, either $H=\abrackets{0}$ or $H=\abrackets{m}$. With $m$ being the least positive integer in $H$. If $H\neq\abrackets{0}$, then $H$ is infinite.
\end{wts}
\begin{wts}[Hungerford Theorem 3.2]\label{hungerford-chp1:theorem3.2}
    Every infinite cyclic group $G=\abrackets{a}$ is isomorphic to the additive group $\ints$, and every finite cyclic group of order $0<m<+\infty$ is isomorphic to the additive group $\ints_m$.
\end{wts}
\begin{proof}[Proof of \Cref{hungerford-chp1:theorem3.2}]
    Let $f:\ints\to G$ and $f(k)=a^k$. By \Cref{hungerford-chp1:theorem2.8}, 
    \[G = \bigset{a^k,\: k\in\ints}\]
    $f$ is clearly surjective and $f\in\Hom[\ints,G]$ is an easy exercise to verify. The proof splits into two parts
    
    \begin{enumerate}[wide,leftmargin=0pt]
        \item[(i) $\Ker{f}$ is trivial: ] By \Cref{hungerford-chp1:theorem2.3}, $f$ is an isomorphism from $\ints$ to $G$, and $G\cong\ints\implies |G|=|\ints|$.
        \item[(ii) $\Ker{f}$ is not trivial: ] Arguing as in \Cref{hungerford-chp1:theorem3.2}, $\Ker{f}$ is a non-trivial subgroup of $\ints$, so it contains a least positive element $m$, and $\Ker{f}=\abrackets{m}$. $m$ is an element of $\ker{f}$, so 
        \begin{equation}\label{hungerford-chp1:theorem3.2-equation}
            f(m)=a^m=e\implies f(jm)=a^{jm}=\prod_{l=1}^j a^m=e,\: j\in \ints
        \end{equation}
            
        
        Now suppose $r$ and $s$ are integers with $f(r)=f(s)$,
        \begin{align*}
            a^r=a^s&\iff a^{r-s}=e\\
            &\iff r-s\in\Ker{f}\\
            &\iff r-s\in\abrackets{m}\\
            &\iff r\equiv s\mod{m}\\
            &\iff \cl{r}=\cl{s}
        \end{align*}
        where $\cl{r}$ denotes the $\ints_m$ equivalence class of $r$. Let $\beta$ be a map from $\ints_m\to G$, such that
        \[
            \beta(\cl{k})=f(k)=a^k
        \]
        This is well defined, since $\beta$ is an invariant on each equivalence class, if $r-s$ differ by a multiple of $m$, then \Cref{hungerford-chp1:theorem3.2-equation} states that $\beta(\cl{r})=\beta({\cl{s}})$.  $G$ is finite, as
        \[
            G =\abrackets{a}=\bigset{a^k,\: k\in\ints,\: m<k<m}
        \]
        and the kernel of $\beta$ is trivial, it is an isomorphism and $\ints_m\cong G$.
    \end{enumerate}
\end{proof}

\topheader{Cosets}


\topheader{Normal Subgroups}


\topheader{Isomorphism Theorems}

\end{document}

