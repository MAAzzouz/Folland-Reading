%Headers and Sections

% Colors

\definecolor{tab-blue}{rgb}{0.341,0.471,0.643}
\definecolor{tab-orange}{rgb}{0.894,0.580,0.267}
\definecolor{tab-red}{rgb}{0.820,0.380,0.365}
\definecolor{tab-teal}{rgb}{0.522,0.714,0.698}
\definecolor{tab-green}{rgb}{0.416,0.624,0.345}
\definecolor{tab-yellow}{rgb}{0.906,0.792,0.376}
\definecolor{tab-purple}{rgb}{0.659,0.486,0.624}
\definecolor{tab-pink}{rgb}{0.945,0.635,0.663}
\definecolor{tab-brown}{rgb}{0.588,0.463,0.384}
\definecolor{tab-grey}{rgb}{0.722,0.690,0.675}


% Custom QED symbol
\renewcommand\qedsymbol{$\blacksquare$}

% Problem Number for assignments
\newcounter{theproblem}

\newcommand{\problem}[1]{%
    \phantomsection\subsection*{Theorem #1}%
    \markright{Theorem #1}%
    \addcontentsline{toc}{section}{Theorem #1}%
    \setcounter{secnumdepth}{1}%
    \stepcounter{theproblem}
}

\newcommand{\fexercisesHeader}[1]{%
    \phantomsection\section*{Exercises}%
    \markright{Exercises}%
    \addcontentsline{toc}{section}{Exercises}%
    \setcounter{theproblem}{0}%
    \setcounter{secnumdepth}{1}%
}

\newcommand{\exercise}[1]{%
    \phantomsection\subsection*{Exercise #1}%
    \markright{Exercise #1}%
    \addcontentsline{toc}{subsection}{Exercise #1}%
    \setcounter{secnumdepth}{0}%
    \stepcounter{theproblem}
}

\newcommand{\topheader}[1]{%
    \phantomsection\subsection*{#1}%
    \markright{#1}%
    \addcontentsline{toc}{section}{#1}%
    \setcounter{secnumdepth}{1}%
    \stepcounter{theproblem}
}


\newtheoremstyle{subtheorem}    % name
    {1.5\topsep}        % space above
    {1.5\topsep}        % space below
    {\itshape}          % name of font to use in the body of the theorem
    {1em}               % measure of space to indent
    {\bfseries}         % name of head font
    { }                 % punctuation between head and body
    { }                 % space after theorem head; " " = normal interword space
    {}

% WTS, now numbered
% \newtheorem{wts}{Proposition}[theproblem]

\theoremstyle{subtheorem}                       % Lemma, Corollary follows Theorem 4.16.X


% Yellow Boxes for definitions
\mdfdefinestyle{definitionstyle}{%
    %backgroundcolor=,
    linewidth=1pt,
    innerleftmargin=6pt,
    innerrightmargin=6pt,
    innertopmargin=6pt, % Adjusted value
    innerbottommargin=6pt,
    skipabove=5pt,
    skipbelow=5pt,
    topline=true, % Added option
    bottomline=true,
    backgroundcolor=tab-yellow!20
}

% Copy the same style but modify the colour only
\mdfdefinestyle{theoremstyle}{%
    style=definitionstyle,
    backgroundcolor=tab-blue!20
}


\mdfdefinestyle{stepstyle}{%
    style=definitionstyle,
    backgroundcolor=tab-teal!20
}
\newcounter{thestep}
\mdtheorem[style=stepstyle]{mysteps}{}[thestep]
\renewcommand{\themysteps}{Step \arabic{thestep}}


\NewDocumentEnvironment{step}{o}
{
    \stepcounter{thestep}
    \IfNoValueTF{#1}
    {
        \begin{mysteps}
    }
    {
        \begin{mysteps}[{#1}]
    }
}
{
    \end{mysteps}
}

\mdfdefinestyle{remarkstyle}{%
    style=definitionstyle,
    backgroundcolor=tab-blue!10,
    leftline=true,
    rightline=true,
    topline=true,
    bottomline=true,
}

\mdtheorem[style=definitionstyle]{definition}{Definition}[theproblem]
\mdtheorem[style=theoremstyle]{wts}{Proposition}[theproblem]

\mdtheorem[style=theoremstyle]{lemma}{Lemma}[theproblem]
\mdtheorem[style=theoremstyle]{corollary}{Corollary}[theproblem]



\mdtheorem[style=remarkstyle]{remark}{Remark}[theproblem]
\mdtheorem[style=theoremstyle]{example}{Example}[theproblem]


% Other header definitions or configurations
%\newtheorem{definition}{Definition}[theproblem]         % Definition

\theoremstyle{subtheorem}
%What to remember
\newtheorem*{wtr}{What to Remember}

% Two custom enumerate environments.
% The labels for these are (a), (iv) respectively.


% enumalpha environment for 
% a), b), c)
\newenvironment{enumalpha}%
    {\begin{enumerate}[label=(\alph*), leftmargin=*]}%
    {\end{enumerate}}

% enumroman environment for
% i), ii), iii)
\newenvironment{enumroman}%
    {\begin{enumerate}[label=(\roman*), leftmargin=*]}%
    {\end{enumerate}}


%for putting chapter numbers after the title
\titleformat{\section}[hang]{%
    \normalfont\Large\bfseries% 
}%
{}{0em}{}{}

% Chapters
\newcommand{\fchapter}[1]%
{%
    \phantomsection\chapter*{Chapter #1}%
    \setcounter{theproblem}{0}%
    \addcontentsline{toc}{chapter}{Chapter #1}\newpage% Chapter Number
} 

% Proof Outline, remove soon
\newenvironment{outline}%
    {\addtocounter{theproblem}{-1}\begin{proofoutline}}%
    {\end{proofoutline}}

\theoremstyle{definition}
\newtheorem{proofoutline}%
[theproblem]{Outline of Problem}




% Here we manage Clever Headers
\crefname{lemma}{lemma}{lemmas}
\Crefname{lemma}{Lemma}{Lemmas}

\crefname{corollary}{corollary}{corollaries}
\Crefname{corollary}{Corollary}{Corollaries}

\crefname{problem}{theorem}{theorems}
\Crefname{problem}{Theorem}{Theorems}

\crefname{exercise}{exercise}{exercises}
\Crefname{exercise}{Exercise}{Exercises}

\crefname{definition}{definition}{definitions}
\Crefname{definition}{Definition}{Definitions}

\crefname{wts}{proposition}{propositions}
\Crefname{wts}{Proposition}{Propositions}