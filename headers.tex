%Headers and Sections

% Custom QED symbol
\renewcommand\qedsymbol{$\blacksquare$}

%Want to show we want it to be numbered with the section at a later date
\newtheorem*{wts}{WTS}

% Problem Number for assignments
\newcounter{theproblem}

\newcommand{\problem}[1]{%
    \phantomsection\subsection*{Theorem #1}%
    \markright{Theorem #1}%
    \addcontentsline{toc}{subsection}{Theorem #1}%
    \setcounter{secnumdepth}{0}%
    \stepcounter{theproblem}
}

\newtheoremstyle{subtheorem}    % name
    {1.5\topsep}        % space above
    {1.5\topsep}        % space below
    {\itshape}          % name of font to use in the body of the theorem
    {1em}               % measure of space to indent
    {\bfseries}         % name of head font
    { }              % punctuation between head and body
    { }                 % space after theorem head; " " = normal interword space
    {}

\theoremstyle{subtheorem}                       % Lemma, Corollary follows Theorem 4.16.X

\newtheorem{lemma}{Lemma}[theproblem]           % Lemma, follows 'Problem' counter
\newtheorem{corollary}{Corollary}[theproblem]   % Corollary
\newtheorem{remark}{Remark}[theproblem]         % Unnumbered remark


\theoremstyle{subtheorem}
%What to remember
\newtheorem*{wtr}{What to Remember}

% Two custom enumerate environments.
% The labels for these are (a), (iv) respectively.


% enumalpha environment for 
% a), b), c)
\newenvironment{enumalpha}%
    {\begin{enumerate}[label=(\alph*)]}%
    {\end{enumerate}}

% enumroman environment for
% i), ii), iii)
\newenvironment{enumroman}%
    {\begin{enumerate}[label=(\roman*)]}%
    {\end{enumerate}}


%for putting chapter numbers after the title
\titleformat{\section}[hang]%
{\normalfont\Large\bfseries}%
{Chapter }{0em}{}{}

\newcommand{\fchapter}[1]%
{\phantomsection\chapter*{Chapter #1}%
\setcounter{theproblem}{0}%
\addcontentsline{toc}{chapter}{Chapter #1}\newpage} % Chapter Number


\theoremstyle{definition}
\newtheorem{proofoutline}%
[theproblem]{Outline of Problem}

\newenvironment{outline}%
    {\addtocounter{theproblem}{-1}\begin{proofoutline}}%
    {\end{proofoutline}}