\documentclass[../main-manifolds.tex]{subfiles}

\begin{document}
\fchapter{1: Manifolds}
% Introduce terminology. linear, toplinear, morphism, isomorphism. And Laut, List, etc. How List is open in the space of toplinear endomorphisms on a space $E$.


\topheader{The structure of a manifold}
It is fruitful to \emph{construct} the manifold rather than \emph{define} it. We also insist on working with open sets of Banach spaces instead coordinate functions as our primary data.\\

We will be working in the category of $C^p$ Banach spaces (all Banach spaces are assumed to be over $\real$). Its morphisms are $C^p$ morphisms: the maps which are continuously $p$-times differentiable (but not necessarily linear). Note that if $p\geq 0$, every toplinear morphism is a $C^p$ morphism, and every toplinear isomorphism is a $C^p$ isomorphism. However, a bijective $C^p$ morphism is usually not a $C^p$ isomorphism. 

\begin{definition}[Chart]\label{def:chart}
    Let $X$ be a non-empty set. A \emph{chart on $X$ modelled on a Banach space $E$} is a tuple $(U,\varphi)$, such that $U\subseteq X$,  $\varphi(U)=\hat{U}$ is an \emph{open} subset of $E$, and $\varphi$ is a bijection into $\hat{U}$.
\end{definition}
\begin{definition}[Compatibility]\label{def:compatibility}
    Let $(U,\varphi)$ and $(V,\psi)$ be charts on $X$ modelled on $E$, they are called $C^p$ compatible if $U\cap V=\varnothing$, or 
    \begin{itemize}
        \item $\varphi(U\cap V)$ and $\psi(U\cap V)$ are \emph{both} open subsets of $E$, and
        \item the \emph{transition map} $\psi\circ\varphi^{-1}: \varphi(U\cap V)\to \psi(U\cap V)$ is a $C^p$ isomorphism between open subsets of $E$.
    \end{itemize}
    It should be clear that compatibility is an equivalence relation on the space of charts of $X$ (that are modelled on $E$).
\end{definition}
\begin{definition}[Atlas]\label{def:atlas}
    Let $X$ be a non-empty set. A \emph{$C^p$ atlas on $X$ modelled on $E$} is a pairwise $C^p$ compatible collection of charts $\{(U_\alpha,\varphi_\alpha)\}$ whose union over the domains cover $X$.
\end{definition}
\begin{remark}[Omissions]
    If the \emph{model space} \( E \) is implied, we will not explicitly reference it. When operating 'within category', we might refer to two charts as \emph{compatible} or \emph{smoothly compatible}, implying they are $C^p$ compatible. This comes from the perspective that, in the context of $C^p$ manifolds, any smoothness exceeding $C^p$ is deemed sufficiently smooth for our purposes.
\end{remark}

Let $X$ be a non-empty set, equipped with a $C^p$ atlas $\{(U_\alpha,\varphi_\alpha)\}$ modelled on $E$. If $\alpha$ and $\beta$ both index the atlas, we write  $U_{\alpha\beta} = U_{\alpha}\cap U_{\beta}$.\\

Suppose $U_{\alpha\beta}$ is non-empty. Then, (by definition) the images $\varphi_{\alpha}(U_{\alpha\beta})$, $\varphi_{\beta}(U_{\alpha\beta})$ are \emph{both} open subsets of $E$, and we will denote the transition map by

\begin{equation}\label{def-transition-map}
    \varphi_{\beta}\circ\varphi_{\alpha}^{-1}=\varphi_{\beta\alpha^{-1}}: \varphi_{\alpha}(U_{\alpha\beta})\to\varphi_{\beta}(U_{\alpha\beta})
\end{equation}

If $p\in (U,\varphi)$, we write $\hat{p}$ for $\varphi(p)$ if there is no room for ambiguity. From \Cref{def:compatibility,def:atlas}, the compatibility relation on charts descends into a compatibility relation on the space of atlases, whose properties are summarized in the following note.

\begin{note}[Descent of an equivalence relation]\label{note:equivalence-relation-descends}
    Let \( \Omega \) be a non-empty set with an associated equivalence relation \( \sim \). This relation \( \sim \) induces another equivalence relation on the set containing all subsets of equivalence classes from \( \Omega \). Suppose \( A \) and \( B \) are subsets of the equivalence classes \( [A] \) and \( [B] \) respectively. The condition \( A \sim B \) holds if and only if for all elements \( x \) in \( A \) and \( y \) in \( B \), \( x \sim y \).\\
    
     This is equivalent to stating that the union \( A \cup B \) lies entirely within some equivalence class, and further, that \( [A] \sim [B] \). The class \( [A] \) represents the largest subset of \( \Omega \) that is entirely contained within a single equivalence class (namely $[A]$ itself) and contains $A$ as a subset.
\end{note}

\begin{definition}[Structure determined by an atlas]\label{def:structure-of-manifold}
    The maximal atlas that contains $\mathcal{A}$ as a subset is called the \emph{$C^p$ structure determined by $\mathcal{A}$}. This maximal atlas is unique, by \cref{note:equivalence-relation-descends}.
\end{definition}
\begin{definition}[Manifold]\label{def:manifold}
    A \emph{$C^p$ manifold modelled on $E$} is a non-empty set $X$ with a $C^p$ structure modelled on $E$. We sometimes refer to the manifold as the smooth structure, rather than the set $X$ itself. $\operatorname{Man}^p$ refers to the \emph{category of $C^p$ manifolds}.
\end{definition}
\begin{wts}[$E$ is a manifold]\label{prop:banach-space-is-manifold}
    Let $p\geq 1$. The identity map $\id{E}: E\to E$ defines an atlas on $E$, which determines a structure called the \emph{standard $C^p$ structure on $E$} or \emph{standard structure on $E$} if the class of morphisms is understood. We will call $(E,\id{E})$ the \emph{standard chart}, or the \emph{global chart} on $E$.
\end{wts}

\begin{wts}[Topology is unique on a manifold]\label{def:topology-unique-on-manifold}
    Let $X$ be a manifold modelled on $E$, it has a unique topology such that the domain for each chart in its smooth structure is open, and each chart is a homeomorphism onto its range (with respect to the subspace topology of $E$).
\end{wts}
\begin{proof}
    We offer a sketch of the proof. Fix a chart $(U,\varphi)$, it is clear that $U$ has to be in the topology of $X$, and because $\varphi: U\to \hat{U}$ is required to be a homeomorphism, we duplicate all the open sets in $\hat{U}$ by using the inverse image through $\varphi$. The collection of all such inverse images form a sub-basis, thus defines a unique topology as is well known.\\

    There is an alternate way of thinking about this 'induced topology'. Given a chart domain, there exists a unique coarsest topology such that all charts with the same chart domain are homeomorphisms onto their images. We can stitch these weak topologies together to form a ambient topology on $X$, as the chart domains cover $X$.
\end{proof}
\begin{remark}[Not necessarily Hausdorff]
    The topology generated is not necessarily Hausdorff, nor second countable. So $X$ may not admit partitions of unity, but for our current purposes we will work with this general definition. Because of the uniqueness of the topology, we sometimes refer to the topology as being part of the \emph{structure} of the manifold.
\end{remark}
\begin{wts}[Open subsets of manifolds]\label{prop:open-subsets-of-manifolds}
    If $U$ is an open subset of a $C^p$ manifold $X$, then $U$ is a $C^p$ manifold whose structure is determined by the atlas
    \begin{equation}\label{eq:open-subset-atlas}
    \bigset{(V,\varphi) \text{ in the structure of }X,\:\text{where } V\subseteq U}
    \end{equation}
\end{wts}
\begin{proof}
    The smooth structure of $X$ includes all possible restrictions to open sets; hence the set in \cref{eq:open-subset-atlas} defines an atlas, and a unique structure by \cref{def:structure-of-manifold}.
\end{proof}
\topheader{Morphisms in $\operatorname{Man}^p$}
\begin{definition}[$C^p$ morphisms between manifolds]\label{def:smoothness}
Let $X$ and $Y$ be $C^p$ manifolds over the spaces $E$ and $F$. A map $F: X\to Y$ is a morphism in $\operatorname{Man}^p$ if for every $p\in X$, there exists charts $(U,\varphi)$ in $X$ and $(V,\psi)$ in $Y$ such that the image $F(U)$ is contained in $V$, and the conjugation of $F$ with respect to the two charts is $C^p$ smooth between open subsets of Banach spaces.

\begin{equation}\label{eq:smoothness}
    F_{U,V}\defined \psi F \varphi^{-1} \in C^p(\hat{U},\hat{V})    
\end{equation}

The map defined in \cref{eq:smoothness} is called the \emph{coordinate representation of $F$} with respect to the charts $(U,\varphi), (V,\psi)$.
\end{definition}

\begin{remark}[Identifying charts with their domains]
Consistent with our notation for the chart domains and $\hat{p}$, we write $\hat{F} = F_{U,V}$ for suitably chosen charts $(U,\varphi)$, $(V,\psi)$ in the respective structures. If we wish to place less emphasis on the specific charts, we say $\hat{F}$ is a coordinate representation \emph{about} $p$. 
\end{remark}

% Motivation for the definition of smoothness
\Cref{def:smoothness} may leave one unsatisfied. A common question that comes to mind is: why do we require the image $F(U)$ be contained in another chart domain in $Y$? There are two reasons.
\begin{enumerate}
    \item First, it is easily verified that the $C^p$ maps between open subsets of Banach spaces satisfy the usual functoral properties in its category. The definition of smoothness between Banach spaces is a purely local one, and it is defined between open subsets; and recall: \textbf{every chart domain $U$ in a manifold $X$ corresponds to an open subset $\hat{U}\subseteq E$ in the model space}. The necessity that $F(U)$ must be contained in a single chart domain of $Y$ is a relic of the original definition.
    \item Second, suppose $f$ is a map between $E$ and $F$, and the restriction of $f$ onto a family of open subsets $U_{\alpha}\subseteq E$ is $C^p$ for $p\geq 0$. If $\{U_\alpha\}$ is an open cover for $E$, then $f$ is continuous. \Cref{prop:smoothness-implies-cont-functorality} shows this equally holds for manifolds.
\end{enumerate}

%

\begin{wts}\label{prop:smoothness-implies-cont-functorality}
    Every $C^p$ morphism between manifolds is a continuous map, and the composition of $C^p$ morphisms is again a morphism.
\end{wts}
\begin{proof}
    The first claim follows immediately from \cref{eq:smoothness}, since $p$ is arbitrary, choose any neighbourhood $W$ of $F(p)$, by shrinking this neighbourhood, it suffices to assume it is a subset of the chart domain $V$. The charts on $X$ and $Y$ are homeomoprhisms, and unwinding the formula shows that $F\vert_{U} = \psi^{-1}F_{U,V}\varphi$, so that
    \[
        U\cap F^{-1}(W) = (F\vert_U)^{-1}(W)\quad\text{is open in }X
    \]
    To prove the second, let $\UL{X}[3]$ be manifolds modelled over $\UL{E}[3]$, and $F_1$, $F_2$ is smooth between $X_i$ such that $F_2\circ F_1$ makes sense. Since $F_1$ is smooth, there a pair of charts $(U_i,\varphi_i)\in X_i$ for $i = 1,2$ about each $p\in X_1$ such that $F_1{_{U_1,U_2}}$ is $C^p$ between open subsets.\\

    $F_2(F_1(p))$ induces another pair of charts $(V_i,\psi_i)\in X_i$ for $i=2,3$. Since $F_2$ is smooth, it is continuous. $F_1^{-1}\circ F_2^{-1}(V_3)$ is open in $X_1$, and we can shrink all of our charts so that $F_2F_1(U_1)$ is contained in $V_3$. Finally, because $C^p$ morphisms between open subsets of Banach spaces is closed under composition, $F_{U_1\cap F_1^{-1}F_2^{-1}(V_3), V_3}$ is smooth.
\end{proof}
\begin{remark}[Concluding remarks]
    Manifolds hereinafter will be assumed of class $C^p$, where $p\geq 1$. If $(U,\varphi)$ is a chart in the structure of $X$, we will simply say $(U,\varphi)$ is in $X$; or $(U)$ is in $X$.
\end{remark}
\topheader{Tangent spaces}
The next question that we will address is taking derivatives of smooth maps between manifolds. There is no reason to demand $C^p$ smoothness between maps, or even a $C^p$ category of manifolds if we cannot borrow something more other than the morphisms on open sets.\\

Suppose $U$ is an open subset of $E$ and $f: U\to Y$ is $C^p$ for $p\geq 1$. The derivative $Df(x)$ is a linear map $E\to F$, not from $U$ to $F$ ($U$ might not even be a vector space). This suggests the 'derivative' of a morphism $F: X\to Y$ between manifolds can in some sense be interpreted as the \emph{ordinary derivative} of its coordinate representation $DF_{U,V}(\hat{p})$, adhering to our principle of using open sets.\\

But there is a problem with this 'derivative': it gives different values for different charts. With infinitely many charts in $X$ and $Y$, this definition becomes useless. To see this, let $X$ be a manifold modelled on $E$ and $p\in X$. If $g: X\to Y$ is a morphism, and $(U_1,\varphi_1)$, $(U_2,\varphi_2)$ are charts defined about $p$ such that the representations $g_{U_1, V}$ and $g_{U_2, V}$ are morphisms. Writing $p_i = \varphi_i p$, and $\varphi_{1,2}=\varphi_2\varphi_{1}^{-1}$ (because it goes from the domain $U_1$ to $U_2$), a simple computation yields
\begin{align}
    Dg_{U_1, V}(p_1)(v) &= D(\psi g \varphi_2^{-1}\varphi_2\varphi_1^{-1})(p_1)(v) \nonumber\\
    &= Dg_{U_2,V}(p_2)\biggl(D\varphi_{1,2}(p_1)(v)\biggr)\nonumber\\
    &= Dg_{U_2,V}(p_2)\circ D\varphi_{1,2}(p_1)\cdot (v)\label{eq:tangent-space-quotient-motivation}
\end{align}
where $\cdot(v)$ denotes the evaluation at $v\in E$, and is assumed to be left associative over composition. The computation in \cref{eq:tangent-space-quotient-motivation} suggests that interpreting the derivative by pre-conjugation is dependent on the chart being used to interpret the derivative. In fact, $D\varphi_{1,2}(p_1)$ can be replaced with any toplinear isomorphism on $E$ (relabel $\varphi_2 = A\varphi_1$ where $A$ is any linear automorphism on $E$), so the right hand side of \cref{eq:tangent-space-quotient-motivation} can be interpreted as $Dg_{U_2,V}(p_2)(w)$ where $w$ is any vector in $E$. 

\begin{definition}[Concrete tangent vector]
    Let $X$ be a manifold on $E$, and $p\in X$. If $(U,\varphi)$ is any chart containing $p$, for each $v\in E$ we call $(U,\varphi,p,v)$ a \emph{concrete tangent vector at $p$} that is \emph{interpreted} with respect to the chart $(U,\varphi)$. The disjoint union of

    \begin{equation}\label{eq:concrete-tangent-space-eq}
        \bigcup_{v\in E}\{(U,\varphi,p,v)\}
    \end{equation}

    is called the \emph{concrete tangent space at $p$} interpreted with respect to $(U,\varphi)$ and inherits a TVS structure from $E$.
\end{definition}

Fix a point $p$ in a manifold $X$. Suppose $(U_i,\varphi_i)$ are charts containing $p$, from \cref{eq:tangent-space-quotient-motivation} we see that there exists a natural correspondence between the interpretations of the concrete tangent space, namely 
\begin{equation}\label{eq:concrete-tangent-vector-relation}
    (U_1,\varphi_1,p,v_1)\sim (U_2,\varphi_2,p,v_2)\qqtext{iff} v_2 = D\varphi_{1,2}(p_1)(v_1)
\end{equation}
where $p_i = \varphi_i p$. 
\begin{definition}[Tangent vector]
    A \emph{tangent vector} (or an \emph{abstract} tangent vector) at $p$ is defined as an equivalence class of concrete tangent vectors at $p$, under the relation in \cref{eq:concrete-tangent-vector-relation}.
\end{definition}

From \cref{eq:concrete-tangent-vector-relation}, since $D\varphi_{1,2}(x)$ is a toplinear automorphism on $E$, this correspondence is a bijection. This means the set of tangent vectors at $p$ inherits a TVS structure from $E$, as $p$ is in the domain of at least one chart $(U,\varphi)$. This is because the concrete tangent space defined in \cref{eq:concrete-tangent-space-eq} admits an obvious (linear) isomorphism with $E$, and each abstract tangent vector admits a unique interpretation with respect to $(U,\varphi)$.

\begin{definition}[Tangent space]
    The \emph{tangent space} at $p$, denoted by $T_p X$ is the set of all tangent vectors at $p$. It is toplinearly isomorphic to the model space $E$.
\end{definition}

\begin{definition}[Differential of a morphism]
    Let $X$ and $Y$ be modelled on the spaces $E$ and $F$. If $g$ be a morphism between $X$ and $Y$, the \emph{differential of $g$ at $p$} is the unique linear map $dg_p: T_p X\to T_{g(p)} Y$. For any tangent vector $v_p$ at $p$ that admits a representative in $(U,\varphi,p,v)$. The concrete vector $(V,\psi,g(p), Dg_{U,V}(\hat{p})(v))$ represents $dg_p(v_p)$, where $(U)$ and $(V)$ satisfy the morphism condition in \cref{eq:smoothness} about $p$.
\end{definition}
\begin{note}[Interpretation using co-product]
    There is another way of interpreting the construction above. Each concrete tangent space is toplinearly isomorphic to $E$, the projection maps onto $\{p\}$ and $E$ can be glued together using the universality of the coproduct, where $\{p\}$ is interpreted as a $0$-dimensional vector space. The construction of $T_pM$ follows by invoking the property of the quotients.
\end{note}
\begin{remark}[Omission of chart in concrete representation]\label{rmk:omission-of-chart-in-concrete-rep}
    If $p$ is a point on a manifold $X$, $v\in T_p M$, we sometimes say $(U,\cl{v})$, or $\cl{v}$ is an interpretation of $v$ if it is clear $(U,\varphi)$ is a chart in $X$. If $X=E$, we will \emph{identify} $v$ with its concrete representation in the \emph{standard chart} $(E,\id{E})$. We say $\cl{w}$ is the \emph{standard representation}, or \emph{standard interpretation} of $w$.
\end{remark}
% Recap of this section, the tangent space at a point is defined to be such that when the chart in the codomain is fixed, the coordinate interpretation of the derivative is independent of the chart used. This is done by gluing together the inputs. Now, we let the chart in the codomain to vary, and we also see that it is independent of the interpretation used.
% Let $F$ be a morphism between manifolds $X$ and $Y$. Because the equivalence relation in eq3 is compatible with linear operations. ie: if we relabel (U_1,v_1) = (U_1,\varphi_1, p, v_1), similarly for U_2.
% (U_1,v_1)\sim (U_2,v_2)\qqtext{and}(U_1,w_1)\sim(U_2,w_2), then their difference and their dilations will be \sim as well.
% So, DF_{, V} is well defined, and is the unique linear map by the property of the quotients.
% on the side of the codomain, we just have to compose DF_{, V} with the map that sends each concrete tangent vector to its abstract tangent vector. This map is linear.
\topheader{Curves}
In the previous section, we motivated the definition of $T_pX$ using the computation of the derivative of a morphism from $X$. Dually, the tangent space allows us compute the derivatives of morphisms into $X$ in a coordinate independent manner.\\

Let $J_\varepsilon = (-\varepsilon, +\varepsilon)$ be an open interval in $\real$. Viewing $J_\varepsilon$ as a manifold, the morphisms $\gamma: J_\varepsilon\to X$ are \emph{curves in $X$} and $\gamma(0)$ is called the \emph{starting point of $\gamma$}.
\begin{definition}[Velocity of a curve]\label{def:velocity-of-a-curve}
    Let $\gamma$ be a curve in $X$ and $t\in J_\varepsilon$. The \emph{velocity} at $t$, denoted by $\gamma'(t)$ --- is the tangent vector with representation $d_{J_\varepsilon,V}\gamma(\cl{1})$; where $(J_\varepsilon,\cl{1})$ is a concrete tangent vector in $T_t J_\varepsilon$.
\end{definition}
\begin{wts}[Tangent vectors are velocities]\label{prop:tangent-vectors-are-velocities}
    Let $p$ be a point on a manifold $X$. For every tangent vector $v\in T_p X$, there exists a curve starting at $p$ whose velocity is $v$.
\end{wts}
\begin{proof}
    Find a chart $(U)$ in the structure of $X$ such that $\hat{p}=0$ in some open set $\hat{U}$ of $X$. Such a chart exists, because translations and dilations are $C^p$ isomorphisms. If the tangent vector $v$ has interpretation $\cl{v}$ in $U$, there exists $\varepsilon>0$ so small that the range of $\hat{\gamma}$, as defined \cref{eq:velocity-curve-gamma-coord-rep}, lies in $\hat{U}$
    \begin{equation}\label{eq:velocity-curve-gamma-coord-rep}
        \hat{\gamma}: J_\varepsilon\to \hat{U}\quad \gamma(t) = \int_0^t \cl{v}dt
    \end{equation}
    $\hat{\gamma}$ is a curve in $\hat{U}$ starting at $\hat{p}$ with velocity $\cl{v}$. Defining $\gamma$ as the composition of $\hat{\gamma}$ with the chart inverse finishes the proof.
\end{proof}

\topheader{Derivations}
Let $X$ be a manifold and $F$ be a Banach space, which is a manifold by \cref{prop:banach-space-is-manifold}. There is more to be said about the differentials of morphisms between the two spaces. Indeed, let $f\in \Mor(X,F)$ and fix a point $p\in X$. Let us write $q = f(p)$ for convenience. Identifying a tangent vector $w\in T_q F$ with its canonical interpretation $\cl{w}$ (see \cref{rmk:omission-of-chart-in-concrete-rep}), we further identify $df_p$ as a continuous linear map $T_p X\to F$.\\

The most important case however, is when $F = \real$. Recall, the space of $C^p$ morphisms from a Banach space $E\to \real$ form a commutative ring denoted by $C^p(E,\real)$. If $\UL{f}[k]\in C^p(E,\real)$, for every $x,y\in E$ we have \cref{eq:derivations-product-rule}
\begin{equation}\label{eq:derivations-product-rule}
    Dm(\UL{f}[k])(x)(y) = \sum_{i=\underline{k}}m(\UL{f}[i-1](x),Df_i(x)(y),\UL{f}[i+][k-i](x))
\end{equation}
where $m$ denotes scalar and pointwise multiplication on $\real$ and $C^p(E,\real)$. Let us introduce some formalism. A \emph{derivation at $x$} is a $v$ linear functional on $C^p(U,\real)$, where $U$ is any open set containing $x$ such that
\begin{equation}\label{eq:derivations-product-rule-abstract}
    vm(\UL{f}[k]) = \sum_{i=\underline{k}}m(\UL{f}[i-1](x), vf_i,\UL{f}[i+][k-i](x))
\end{equation}
We will denote the space of derivations at $x$ by $\dzz_x(E)$. It is not hard to see that it is a vector space over $\real$, and $E$ injects into $\dzz_x(E)$ through its action on $C^p$ functions in \cref{eq:derivations-product-rule}. If $E = \realn$, \cref{prop:tangent-space-isomorphic-to-derivations-realn} shows that it is a surjection.
\begin{wts}[$T_x \realn$ is isomorphic to $\dzz_x(\realn)$]\label{prop:tangent-space-isomorphic-to-derivations-realn}
    
\end{wts}

% \fchapter{2: Submanifolds}
% Let X be a topological space, a subset $S\subseteq X$ is locally closed if every point $p\in S$ admits a neighbourhood $U\osub X$ such that $U\cap S$ is closed in $U$.

%wts: all locally closed sets are the intersection of a closed set and an open set.

% Let U_\alpha cover our locally closed S, such that U_\alpha\cap S is a closed subset of $U_\alpha$, so that $U_\alpha\setminus S$ is open in $U_\alpha$. Since $U_\alpha$ is open (Munkres Lemma 16.2) implies $U_\alpha\setminus S$ is open in $X$.

% Take the union over all such $U_\alpha\setminus S$, since $U_\alpha$ covers $S$. Denote $\bigcup U_\alpha = U$. And $U\setminus S$ is open in $X$. Its complement relative to $X$ is closed. And

% X\setminus (U\setminus S) = U^c + S is closed in $X$. 
% Take the intersection with $U$, which is an open subset of $X$,
% U\cap [X\setminus (U\setminus S)] = U\cap (U^c + S) = U\cap S = S
% Therefore $S$ is the intersection between an open set and a closed set.

% Embedded submanifolds are precisely the subsets of a manifold $X$ where we can pull existing charts (if we can 'drop' some of the coordinates) from the smooth structure of $M$. Since these charts cover $S$, it defines an atlas: and the uniqueness of the structure (which includes the topology) follows.

% Lang: Fix a manifold $X$ (of class $C^k$), and a subset $S\subseteq X$. For every $p\in S$, we can find a chart $(U,\varphi)$ in the structure of $X$ containing the point $p$, such that

% U is isomorphic, through \varphi to a product manifold $V_1\times V_2$. Where $V_i$ are open subsets of Banach space $E_i$. And that

% \varphi(U\cap S) = V_1\times {a_2}, where $a_2$ is some point in $V_2$. So $U\cap S$ belongs precisely to some fiber. The isomorphism above should be interpreted in the $C^k$ manifold sense. 

%Because open subsets of Banach spaces are submanifolds. (have we proven this yet?)

% Lee interpretation: 
% Theorem 5.16: every embedded submanifold is locally a levle set. A subset $S\subseteq X$ is an embedded submanifold if and only fi for every $p\in S$ there exists a chart in the smooth structure on $$ such tht $U\cap S$ is precisely a level set of a morphism $f_{U\cap S}$ with codomain $\real^{n-k}$. In this case we say $S$ is an embedded submanifold of dimension $k$.

% The dimension of the codomain of the local defining function of the embedded submanifold is the codimension of the embedded submanifold


\topheader{Splitting}
\begin{definition}[Splitting in $E$]\label{def:splitting-subspace}
Let $E_1$, and $E_2$ be closed, vector subspace complements of each other in $E$; this means . $E_1 + E_2 = E$, $E_1\cap E_2 = 0$. If the addition map $(\cdot,\cdot): E_1\times E_2\to E$ is a toplinear isomorphism $(x,y)\mapsto x + y$ then we say $E_i$ \emph{splits} in $E$.    
\end{definition}
\begin{remark}[Every linear subspace splits in finite dimensions.]
Every finite dimensional or finite codimensional linear subspace of $E$ splits. If $E$ is finite dimensional, then every linear subspace splits.    
\end{remark}
If $\lambda\in L(E,F)$ is injective, we would like to describe the situation where we can think $E$ being toplinearly isomorphic to its range, similar to the matrix canonical form $\begin{bmatrix}I_{k} & 0_{k\times (n-k)} \end{bmatrix}$.
\begin{definition}[Splitting in $L(E,F)$]\label{def:splitting-clm}
A continuous, injective linear map $\lambda\in L(E,F)$ \emph{splits} if there exists a toplinear isomorphism $\alpha: F\to F_1\times F_2$ such that $\lambda$ composed with $\alpha$ induces a toplinear isomorphism from $E$ onto $F_1\times 0$ - which we identify with $F_1$.
\end{definition}
\topheader{Submanifolds}
Before we state the definition of a submanifold, it is important to recapitulate the construction of a manifold $X$.
\begin{enumerate}
    \item Given a non-empty set $X$ and an atlas modelled on a space $E$.
    \item The purpose of each chart in the atlas is to borrow open subsets $\hat{U}\osub E$. If we single out a single chart, \textbf{the construction is entirely topological}. It is of little importance \emph{how} the individual chart domains $U$ are mapped onto $\hat{U}$,
    \item Each chart is in \textbf{bijection with its range}, which is an open subset of $E$, and
    \item the transition maps $\varphi_{\beta\alpha^{-1}}$ are \textbf{morphisms between open subsets of $E$}.
\end{enumerate}
In the spirit of borrowing definitions and properties from existing objects, it makes (functoral) sense a submanifold $S$ should be modelled a linear subspace of $E_1$ of $E$. The natural charts we can borrow from the structure of $X$ are those with the 'other coordinates' muted. If $(U,\varphi)$ is a chart whose domain intersects $S$, the restriction of $\varphi$ onto $U\cap S$ should be in bijection with an open subset of $E_1$. 
\begin{equation}\label{eq:motivation-submanifold}
    \varphi(S\cap U) = \hat{U}_1\times ?,\quad \hat{U}_1\osub E_1
\end{equation}
There is a problem with \cref{eq:motivation-submanifold} however, $\varphi$ is a $C^p$ isomorphism onto $\hat{U}$; not onto open subsets of the product space $E_1\times E_2$. An easy fix to this would be to require $E_1$ \textbf{to split in $E$} (and shrinking $U$ using a basis argument). Let $\alpha$ be a $C^p$ isomorphism between $E$ and $E_1\times E_2$. \Cref{eq:motivation-submanifold} becomes
\begin{equation}\label{eq:motivation-submanifold-modified}
    \alpha\varphi(S\cap U) = \hat{U}_1\times a_2\qqtext{where} \hat{U}_1\osub E_1 \text{ and }a_2\in E_2
\end{equation}
Identifying $\hat{U}$ with $\alpha(\hat{U})$, and requiring $U_1\times a_2$ to be in $\alpha(\hat{U})$, we arrive at the following definition.
\begin{definition}[Submanifold]
    Let $X$ be a manifold, and $S$ a subset of $X$. We call $S$ a \emph{submanifold} of $X$ if there exist split subspaces $E_1$, $E_2$ of $E$; such that, every $p\in S$ is contained in the domain of some chart $(U,\varphi)$ in $X$. Where
    \begin{equation}\label{eq:submanifold-slice-chart-splits}
        \varphi: U\to \hat{U}\cong \hat{U}_1\times \hat{U}_2,\qqtext{where} U_i\osub E_i\quad i=1,2
    \end{equation}
    and there exists an element $a_2\in U_2$
    \begin{equation}\label{eq:submanifold-slice-chart-level-set}
        \varphi(U\cap S) = \hat{U}_1\times a_2
    \end{equation}
\end{definition}
We call a chart satisfying \cref{eq:submanifold-slice-chart-splits,eq:submanifold-slice-chart-level-set} a \emph{slice chart} of $S$; to simplify what follows, we write $\varphi^i = \operatorname{proj}_i\varphi$ for $i = 1,2$ for any slice chart $(U)$. Given that $\operatorname{proj}_i$ is a morphism between open subsets of Banach spaces, $\varphi^i$ is again a morphism. In particular, $\varphi^1$ is in bijection from $U^s=U\cap S$ onto $\hat{U}_1$; the latter being an open subset of $E_1$. To show $S$ is indeed a manifold it remains to show the collection of charts in \cref{eq:submanifold-induced-atlas} forms a $C^p$ atlas modelled $E_1$, which we will prove in \cref{prop:structure-of-submanifold}
\begin{equation}\label{eq:submanifold-induced-atlas}
    \acal = \bigset{(U^s, \varphi^1),\:\: (U,\varphi)\text{ is a slice chart of }S}
\end{equation}
\begin{wts}[Structure of a submanifold]\label{prop:structure-of-submanifold}
    If $S$ is a submanifold of $X$, \cref{eq:submanifold-induced-atlas} defines a $C^p$ atlas over the space $E_1$. The manifold $S$ has a topology that coincides with the subspace topology, and the inclusion map $\iota_S: S\to X$ is a morphism and a homeomorphism onto its range.
\end{wts}
\topheader{Properties of Differentials}

\end{document}