\documentclass[../main-manifolds.tex]{subfiles}

\begin{document}
\providecommand{\szz}{\mathcal{S}}
\providecommand{\ccinf}{C_c^\infty}

% Topologies
\providecommand{\Taux}{\Tau_\xx}
\providecommand{\Tauy}{\Tau_\yy}
\providecommand{\Tauxy}{\Tau_{\xx\times\yy}}

% Basis
\providecommand{\Bx}{\borel_\xx}
\providecommand{\By}{\borel_\yy}
\providecommand{\Bxy}{\borel_{\xx\times\yy}}


\fchapter{9: Integral Curves and Flows}
\topheader{Time invariant flows}
% Stuff here


\topheader{Lie Derivatives of Vector Fields}
\begin{definition}[Lie Derivatives of Vector Fields]\label{lee-chp9:lie-derivative-definition}
    Let $M$ be a smooth manifold and $V, W$ be smooth vector fields on $M$, and $\theta$ be the flow of $V$. The \emph{Lie derivative of $W$ with respect to $V$} is a rough vector field $\mathcal{L}_V W$
    \begin{equation}\label{lie-derivative-def-1-vector}
        (\mathcal{L}_V W)_p \defined \eval{\dv{t}}_{t=0} d(\theta_t)_{\theta_t(p)}(W_{\theta_t(p)})
    \end{equation}
    The above uses the Frechet derivative on the (normed) vector space $T_p M$, and the result is an element in $T_p M$. Equivalently,
    \begin{equation}\label{lie-derivative-def-2-vector}
        (\mathcal{L}_V W)_p = \lim_{t\to 0}\dfrac{d(\theta_{-t})_{\theta_t (p)}(W_{\theta_t(p)}) - W_p}{t}
    \end{equation}
\end{definition}
\begin{remark}
    If we use the regular the 'velocity of curves' definition, we will see the result is an element in $T_{W_p}(T_pM)$, which is not what we want. However, Proposition 3.13 allows us to identify, through the canonical linear isomorphism $T_pM\cong T_{W_{p}}(T_p M)$ for a fixed $W_p\in T_pM$.
\end{remark}

\begin{wts}[Existence of {$\mathcal{L}_V W$}]\label{lee-chp9:lemma-9.36} % Lee Lemma 9.36
    Let $M$ be a smooth manifold and $V,W\in \vField(M)$, $(\mathcal{L}_V W)_p$ exists for every $p\in M$, and $\mathcal{L}_V W$ defines a smooth vector field.
\end{wts}
\begin{proof}
    Let $\theta$ denote the flow of $V$, and $p\in M$ be arbitrary. Let $(x^i)$ be the local coordinates in some neighbourhood $U$ about $p$. Since $(0,p)$ is a point in the flow domain of $V$, there exists an open interval containing $0$ $J_0$, and an open neighbourhood $U_0$ containing $p$, such that $U_0\subseteq U$.\\
    Since every single point $\theta_t(x)$ for $(t,x)\in J_0\times U_0$ share the same local coordinates as $p$, the matrix representation of the differential $d(\theta_{-t})_{\theta_t(x)}: T_{\theta_t(x)}: T_{\theta_t(x)}M \to T_{x}M$ is precisely the Jacobian of the coordinate representation of $\theta$. 
    \[
        d(\theta_{-t})_{\theta_t(x)} = \qty( \pdv{\theta^i}{x^j}) (-t,\theta(t,x))
    \]
    is a matrix of smooth functions $\pdv{\theta^i}{x^j}$ evaluated at $(-t,\theta(t,x))\in J_0\times U_0$. Thinking of $T_pM$ as a normed vector space, it has basis 
    \[
        \qty(\eval{\pdv{x^1}}_p,\ldots \eval{\pdv{x^n}}_p)
    \]
    To show that $(\mathcal{L}_V W)_p$ exists, let $X: J_0\to T_pM$ be a map (not a curve) starting at $W_p$, given by $X(t) = d(\theta_{-t})_{\theta_t(p)}(W_{\theta_t(p)})$, computing its Frechet derivative with respect to the above basis,
    \begin{equation}\label{lee-9.36-coefficients}
        \eval{\dv{t}}_{t=0} X(t) = \eval{\dv{t}}_{t=0} \underbrace{\pdv{\theta^i}{x^j}(-t,\theta(t,p))}_{\text{smooth matrix}} \underbrace{W^j(\theta(t,p))}_{W\text{ vector}} \underbrace{\eval{\pdv{x^i}}_{p}}_{\text{basis}}
    \end{equation}
    $X(t)$ clearly has smooth coefficients in $J_0\times U_0$, so the Frechet derivative exists. Further, the coefficients in \Cref{lee-9.36-coefficients} vary smoothly with $t$ and $p$ in $J_0\times U_0$. So the rough vector field $(\mathcal{L}_VW)$ is smooth around any $p$ in $M$, and the proof is complete.
\end{proof}
\begin{remark}
    The above holds for manifold with boundary as well, we require the vector field $V$ to be tangent to $\partial M$, for the flow to be defined on the boundary. Further, we can view this Lie derivative as the Frechet limit of the \emph{pullback} through the diffeomorphism $\theta_{-t}: M_{-t}\to M_{t}$. This point of view will be useful later on, when we look at Lie derivatives of (covariant) tensor fields.
\end{remark}


\begin{wts}[Computing Lie derivatives at {$t_0$}]\label{lee-chp9:prop-9.41}
    Let $M$ be a smooth manifold, $V,W\in\vField(M)$, and the flow of $V$ by $\theta$ with domain $\mathcal{D}$. If $(t_0, p)\in \mathcal{D}$,
    \[
        \eval{\dv{t}}_{t=t_0} d(\theta_{-t})_{\theta_t(p)}(W_{\theta_t}(p)) = d(\theta_{-t_0})\qty((\mathcal{L}_V W)_{\theta_{t_0}(p)})
    \]
\end{wts}
\begin{proof}
    Let $J_0\times U_0$ is the product of an open interval containing $0$ and an open neighbourhood of $p$, and $X: J_0\to T_pM$, where
    \[
        X(t)\defined d(\theta_{-t})_{\theta_t(p)}(W_{\theta_t}(p))
    \]
    This is a smooth curve, and we can shrink $J_0$ and $U_0$ respectively so that $\theta$ maps $J_0\times U_0$ into the same chart $(U, (x^i))$. Writing $X$ in local coordinates, noting that $X$ maps into the \emph{vector space $T_pM$}. \Cref{lee-9.36-coefficients} tells us $X$ is indeed a smooth curve in $T_pM$. The composition of smooth maps is again smooth, define $\tau: J_0\to \tau(J_0)$, with $t\mapsto t_0 + t$
    \[
        (X\circ \tau)(t) = X(t+t_0)\quad\text{is smooth on }\tau(J_0)
    \]
    and
    \[
        (X\circ \tau)(t) = d(\theta_{-t_0})_{\theta_{t_0}(p)}\circ d(\theta_{-t})_{\theta_{\theta_{t}(\theta_{t_0}(p))}}(W_{\theta_{t}(\theta_{t_0}(p))})
    \]
    \begin{align*}
        X'(t_0) &=\eval{\dv{t}}_{t=t_0} X(t)\\
        &= \eval{\dv{t}}_{t=0} X(t + t_0)\\
        &= \eval{\dv{t}}_{t=0} d(\theta_{-t_0})\circ d(\theta_{-t})(W_{\theta_{t}(\theta_{t_0}(p))})\\
        &= d(\theta_{-t_0})\qty(\eval{\dv{t}}_{t=0} W_{\theta_t(\theta_{t_0}(p))} )\\
        &= d(\theta_{-t_0})(\mathcal{L}_VW)_{\theta_{t_0}(p)}
    \end{align*}
    One way to justify the second last equality is to view the whole computation using Frechet derivatives.
\end{proof}



\topheader{Flowouts from Boundary}
This will be a draft of Problem 9.11. We wish to modify the proof for Theorem 9.20 (Flowout Theorem). We need to modify the following aspects
\begin{itemize}
    \item $M$ is now a smooth manifold with boundary,
    \item The original embedded submanifold of the flow $\mathcal{D}$, is $\mathcal{O} = (\real\times S)\cap \mathcal{D}$. We need to change it to

    \[
        \mathcal{O} = ([0,+\infty)\times \partial M)\cap \mathcal{D}
    \]
    where $\mathcal{D}$ is the flow of the smooth vector field $N$, which is everywhere inward pointing on the boundary of $M$.
    \item The induced smooth embedding $\Phi: P_\delta\to M$ (and thus diffeomorphism, since $\partial M$ has codimension $1$), is defined on 
    \[
        P_\delta = \bigset{(t,p),\: p\in\partial M,\: 0\leq t< \delta(p)}
    \]
    where $\delta:\partial M\to \real^+$ is a smooth, strictly positive function.
    \item For each $p\in \partial M$, the map $t\mapsto \Phi(t,p)$ is an integral curve of $N$ starting at $p$. 
\end{itemize}

\topheader{Time varying flows}




\fchapter{12: Tensors}
\topheader{Tensor Products}
\begin{definition}[Covariant Tensor]
    Let $V$ be a finite-dimensional vector space. Covariant tensors are multi-linear maps $\alpha: V\times \cdots \times V\to \real$. If $\alpha$ sends $k$-copies of $V$ into $\real$, it is called a \emph{$k$-tensor}, or a \emph{covariant tensor of rank $k$}. It is clear that $k$-tensors form a vector space over $\real$ for any $k\geq 0$. The convention is that $0$-tensors are real numbers as they have $0$ arguments in $V$. 
    \[
        T^{k}(V^*) = \bigset{\alpha: \underbrace{V\times\cdots\times V}_{k\text{ copies}}\to\real,\: \alpha\text{ is multi-linear.}}
    \]
\end{definition}

\begin{definition}[Contravariant Tensor]
    Contravariant tensors are multi-linear maps $\alpha: V^*\times \cdots\times V^*\to \real$, similar to covariant tensors. Denoted by
    \[
        T^{k}(V) = \bigset{\alpha: \underbrace{V^*\times\cdots\times V^*}_{k\text{ copies}}\to\real,\: \alpha\text{ is multi-linear.}}
    \]
\end{definition}

\begin{definition}[Mixed Tensors]
    A $(k,l)$-mixed tensor, or just $(k,l)$-tensor is a multi-linear map that takes $k$ copies of $V^*$, and $l$ copies of $V$ into $\real$. In symbols, 
    \[
        T^{(k,l)}(V) = \bigset{\alpha: \underbrace{V^*\times\cdots\times V^*}_{k\text{ copies}}\times \underbrace{V\times\cdots\times V}_{l\text{ copies}}\to\real,\: \alpha\text{ is multi-linear.}}
    \]
\end{definition}

The space of $k$-covariant, $k$-contravariant, and $(k,l)$-tensors form a vector space over $\real$. We will rarely use contra-variant tensors, so we will use the word \emph{tensor} to refer to covariant tensors only.

\begin{definition}[Tensor Product]
    Let $F$ and $G$ be $k$ and $l$-tensors on $V$ respectively, we define a bi-linear map,
    \[
        F\otimes G: \underbrace{V\times \cdots \times V}_{k+l\text{ copies}}\to\real
    \]
    by
    \[
        (F\otimes G)(\Ul{v}[k],\Ul{w}[l]) = \underbrace{F(\Ul{v}[k])G(\Ul{w}[l])}_{\substack{\text{scalar}\\ \text{multiplication}\\ \text{ in }\real}}
    \]
    Tensor product is clearly associative, and commutative (although we will not use this fact).
\end{definition}

\begin{wts}[Basis of {$T^k(V^*)$}]
    Let $V$ be a $n$-dimensional vector space with basis $(E_i)$, and dual basis $(\varepsilon^i)$. $T^K(V^*)$ has basis
    \[
        \borel = \bigset{\bigotimes \varepsilon^{\Ul{i}[k]},\: \Ul{i}[k]\in \{\underline{n}\}} = \bigset{\varepsilon^{i_1}\otimes\cdots\otimes \varepsilon^{i_k},\: 1\leq i_j\leq n,\:  1\leq j\leq k}
    \]
    This means, if $A$ is a $k$-tensor, and for every multi-index $I = (\Ul{i}[k])$, define
    \[
        A_I = A_{i_1,\ldots i_k} = A_{\Ul{i}[k]} = A(E_{\Ul{i}[k]}) = A(E_{i_1},\cdots, E_{i_k}) = A(E_I)
    \]
    are precisely the \emph{basis coefficients} of $F$ with respect the basis $\borel$, 
    \begin{equation}
        F = F_{i_1,\ldots, i_k}\varepsilon^{i_1}\otimes\cdots\otimes\varepsilon^{i_k}
    \end{equation}
\end{wts}
\topheader{Tensor Fields on Manifolds}
Similar to the tangent bundle $TM = \coprod_{p\in M}T_p M = \bigcup_{p\in M}\{p\}\times T_p M$, is endowed with a unique smooth structure that makes the canonical projection $\pi: TM\to M$, $\pi(p,v)=p$ a smooth embedding. We can use the algebraic machinery we have built up from Chapter 10 to construct abstract tensor bundles on $M$. For completeness, we will state the definitions for covariant and mixed bundles as well.
\begin{definition}[Tensor Bundles]\label{lee-chp12:tensor-bundles}
    Let $M$ be a smooth manifold with or without boundary, define
    \begin{itemize}
        \item Covariant $k$-bundle over $M$ by
        \[
            T^kT^*M = \coprod_{p\in M} T^k(T_p^*M)
        \]
        \item Contravariant $k$-bundle over $M$ by
        \[
            T^kTM = \coprod_{p\in M} T^k(T_pM)
        \]
        \item $(k,l)$-mixed bundle over $M$ by
        \[
            T^{(k,l)}TM = \coprod_{p\in M}T^{(k,l)}(T_pM)
        \]
    \end{itemize}
\end{definition}

\begin{definition}[Tensor Field on $M$]
    A \emph{$k$-covariant tensor field}, or just a \emph{$k$-tensor field} over a smooth manifold (with or without boundary) $M$, is a smooth section of the vector bundle as defined in \Cref{lee-chp12:tensor-bundles}. As with vector fields $\vField(M)$, they form a $R$-module over $C^\infty(M)$ by pointwise multiplication.\\

    We denote the space of $k$-tensor fields over $M$ by $\Tau^k(M)$. Moreover, if $A\in \Tau^k(M)$, we can write $A$ in local coordinates $(x^i)$ by
    \[
        A = A_{i_1,\ldots,i_k}dx^{i_1}\otimes\cdots\otimes dx^{i_k}
    \]
    with summation convention in effect, since $(dx^{1},\ldots,dx^{n})$, when evaluated pointwise in local coordinates, form a dual basis of $T_pM$. Where each $dx^i$ is the Chapter 11 differential of the $i$-th coordinate function.
\end{definition}
\topheader{Technicalities for Tensor Fields}
We will skip the technicalities for now. Recall $k$-tensors on a vector space act on a $k$-tuple of vectors, the same is true for $k$-tensor fields over a manifold $M$. If $A\in \Tau^k(M)$, and $(X_1,\ldots, X_k)\subseteq\vField(M)$, then
\[
    A(\Ul{X}[k]) = A(X_1,\ldots,X_k)\quad\text{such that}\quad A(\Ul{X}[k])(p) = A_p(\Ul{X}[k]\vert_{p}) = A_p(X_1\vert_{p},\ldots,X_k\vert_{p})\in\real
\]
We should expect, that if $A$ is a rough $k$-tensor field over $M$, it is smooth if and only if for every $k$-tuple of smooth vector fields, the resulting function (as defined pointwise above), is $C^\infty(M)$. See Lee Proposition 12.19 for more details.


\topheader{Pullbacks of Tensor Fields}
\begin{definition}[Pointwise pullback {$dF_p^*: T^k(T_{F(p)}N)\to T^k(T_pM)$}]
    Let $M$ and $N$ be smooth manifolds with or without boundary. Let $F:M\to N$ be a smooth map. For every $p\in M$, if $\alpha$ is a $k$-tensor on $T_{F(p)}N$, (so that $\alpha\in T^k(T_{F(p)}^*N)$), the \emph{pointwise pullback} of $\alpha$ through $dF_p$ is a $k$-tensor on $T_p M$. Denoted by $dF^*_p(\alpha)$, if $(\Ul{v}[k])$ are tangent vectors in $T_pM$, then
    \begin{equation}\label{tensor-field-pointwise-pullback-eq}
        dF^*_p(\alpha)(\Ul{v}[k]) = \alpha\biggl(dF_p(\Ul{v}[k])\biggr)
    \end{equation}
\end{definition}

\begin{definition}[Tensor Pullback {$F^*: T^kN\to T^kM$}]
    Let $F: M\to N$ be a smooth map, between smooth manifolds with or without boundary, if $A$ is a $k$-tensor field on $N$, we define 
    \begin{equation}\label{tensor-field-pullback-eq}
        (F^*A)_p(\Ul{v}[k]) = dF^*_p(A_{F(p)}) = A_{F(p)}(dF_p(\Ul{v}[k]))
    \end{equation}
    the result is a $k$-tensor field (which is smooth, by the next Proposition).
\end{definition}

\begin{wts}[Proposition 12.25, Properties of Tensor Pullbacks]
    Let $F: M\to N$ and $G:N\to P$ be smooth maps between smooth manifold with or without boundary, and $A$ and $B$ are $k$-tensor fields on $N$, then
    \begin{enumroman}
        \item For every $f\in C^\infty(N)$,
        \[
            F^*(fB) = \underbrace{(f\circ F)}_{C^\infty(M)}F^*B
        \]
        \item $F^*$ commutes with tensor products,
        \[
            F^*(A\otimes B) = F^*(A)\otimes F^*(B)
        \]
        \item $F^*$ is linear over $\real$, 
        \[
            F^*(aA + bB) = aF^*(A) + bF^*(B)
        \]
        \item $F^*(B)$ is smooth,
        \item The tensor pullback satisfies the following co-functoral properties
        \begin{itemize}
            \item The (tensor) pullback of the composition is the (tensor) pullback of the \emph{pre}-composition,
            \[
                (G\circ F)^* = F^*\circ G^*
            \]
            \item The tensor pullback of the identity map on $M$ is the identity map on tensor fields over $M$,
            \[
                \id{M}^* =\id{\Tau^k(M)}
            \]
        \end{itemize}
    \end{enumroman}
\end{wts}
\begin{wts}[Tensor Field Pullback in Coordinates]\label{tensor-field-pullback-coordinates}
    Let $M$ and $N$ be smooth manifolds with or without boundary, and $F:M\to N$ be a smooth map. $A$ is a smooth covariant $k$-tensor field on $N$, and $p\in M$ induces local coordinates $(U,(x^i))$ about $p$, $(V,(y^j))$ about $F(p)$. If $A$ has the following coordinate representation about $F(p)$,
    \begin{equation}\label{tensor-field-pullback-coordinates-eq0}
        A = \underbrace{A_{i_1,\ldots,i_k}}_{\substack{\text{smooth }\\ \text{function in N}}} \: dy^{i_1}\otimes\cdots\otimes dy^{i_k} = A_{\Ul{i}[k]}\qty(\bigotimes dy^{\Ul{i}[k]})
    \end{equation}
    then $F^*A$ is given by
    \begin{align}\label{tensor-field-pullback-coordinates-eq1}
        F^*_pA &= (A_{i_1,\ldots,i_k}\circ F)d(y^{i_1}\circ F)\otimes \cdots\otimes d(y^{i_k}\circ F)\\
        &= (A_{i_1,\ldots,i_k}\circ F)d(F^{i_1})\otimes \cdots\otimes d(F^{i_k})\\
        &= (A_{\Ul{i}[k]}\circ F)\qty(\bigotimes d(F^{\Ul{i}[k]}))
    \end{align}
    in local coordinates.
\end{wts}

\topheader{Lie Derivatives of Tensor Fields}
\begin{definition}[Lie Derivative of Tensor Fields]
    Let $M$ be a smooth manifold and $V\in\vField(M)$, and let $\theta$ be the flow of $V$. If $A\in \Tau^k(M)$ is a smooth $k$-tensor field, we define the \emph{Lie derivative of $A$ with respect to $V$}, denoted by $\mathcal{L}_V A$, 
    \begin{equation}\label{lie-derivative-eq1-tensor}
        (\mathcal{L}_VA)_p \defined\eval{\dv{t}}_{t=0}(\theta^*_tA)_p = \lim_{t\to 0}\dfrac{d(\theta_t)^*_p(A_{\theta_t(p)}) - A_p}{t}
    \end{equation}
    Similar to \Cref{lee-chp9:lie-derivative-definition}, we view $T^k(T^*_pM)$ as a vector space, and $\eval{\dv{t}}_{t=0}(\theta_t^*A)_p$ is the Frechet Derivative on $T^k(T^*_pM)$.
\end{definition}

\begin{wts}[Existence of {$\mathcal{L}_V A$}]
    Let $M$ be a smooth manifold, and $V\in \vField(M)$ with integral flow $\theta$ defined on $\mathcal{D}\subseteq\real\times M$. If $A$ is a smooth covariant $k$-tensor field on $M$, then $(\mathcal{L}_V A)_p$ exists for every $p\in M$, and defines a smooth $k$-tensor field.
\end{wts}
\begin{proof}
    Fix $p\in M$, following $\Cref{lee-chp9:lemma-9.36}$ closely, since $(0,p)\in\mathcal{D}$ induces an open interval $J_0$ containing $0$ and an open neighbourhood $U_0$ containing $p$. We write $A$ in local coordinates for $(t,x)\in J_0\times U_0$. Shrinking $U_0$ if necessary, we it suffices to assume $\theta$ maps $J_0\times U_0$ into the same chart $(U,(x^i))$.\\

    If $(t,x)\in J_0\times U_0$, $A_{\theta_t(x)}\in T^k(T_{\theta_t(x)}M)$ has coordinate representation
    \[
        A_{\theta_t(x)} = \underbrace{A_{i_1,\ldots, i_k}(\theta(t,x))}_{\text{smooth function}}\: \underbrace{dx^{i_1}_{\theta_t(x)}\otimes\cdots\otimes dx^{i_k}_{\theta_t(x)}}_{\text{basis vector}}
    \]
    By \Cref{tensor-field-pullback-coordinates-eq1}, the pullback $d(\theta^*_t)_{p}(A_{\theta_t(p)})$ precomposes $A_{\Ul{i}[k]}$ with $\theta_{t}(p)$. Since $\theta(t,x)\in (U,(x^i))$, $\theta^j(t,x)$ is precisely the $j$th coordinate function for all $(t,x)\in J_0\times U_0$.
    \[
        d(\theta^*_t)_{p}(A_{\theta_t(p)}) = (A_{\Ul{i}[k]}\circ \theta^{(p)}) (t)\qty(\bigotimes dx_{p}^{\Ul{i}[k]})\in T^k(T^*_pM)
    \]
    $(A_{\Ul{i}[k]}\circ \theta^{(p)}) (t)$ is clearly smooth in both arguments. So $(\mathcal{L}_V A)_p$ exists, and is smooth about every $p\in M$.
\end{proof}

\begin{wts}[Computing Lie derivatives at $t_0$ (Tensor Fields)]
    Let $M$ be a smooth manifold, $V\in\vField(M)$, and the flow of $V$ by $\theta$ with domain $\mathcal{D}$. Let $A$ be a smooth covariant $k$-tensor field on $M$. If $(t_0, p)\in \mathcal{D}$,
    \[
        \eval{\dv{t}}_{t=t_0} d(\theta_{-t})_{\theta_t(p)}(A_{\theta_t}(p)) = d(\theta_{-t_0})\qty((\mathcal{L}_V A)_{\theta_{t_0}(p)})
    \]
\end{wts}
\begin{proof}
    The proof is the same as in \Cref{lee-chp9:prop-9.41}.
\end{proof}

\fchapter{13: Riemannian Metrics}
% Tangent co-tangent isomorphism
% Musical Isomorphisms
% Definition of Gradient

\fchapter{14: Differential Forms}
\topheader{Multi-linear algebra}

\begin{definition}[Wedge Product]
    Let $\omega\in \Lambda^k(V^*)$, $\eta\in\Lambda^l(V^*)$ the wedge product of $\omega$ and $\eta$ is defined 

\begin{equation}\label{lee-chp14:wedge-product}
    \omega\wedge \eta \defined \dfrac{(k+l)!}{k!l!}\Alt{\omega\otimes \eta} 
\end{equation}

with $\Alt{\omega\otimes\eta} = \dfrac{1}{(k+l)!}\sum_{\sigma\in S_{k+l}}(\sgn\sigma)^\sigma{(\omega\otimes\eta)}$, and if $(v_1\ldots v_k, v_{k+1},\ldots v_{k+l})$ are vectors in $V$, then

\begin{equation}\label{lee-chp14:wedge-product-evaluated}
\Alt{\omega\otimes\eta}(v_1\ldots v_k, v_{k+1},\ldots v_{k+l})\defined\dfrac{1}{(k+l)!}\sum_{\sigma\in S_{k+l}}(\sgn\sigma)\omega(v_{\sigma(1)},\ldots, v_{\sigma(k)})\eta(v_{\sigma(k+1)},\ldots, v_{\sigma(k+l)})
\end{equation}

To simplify the expressions, we will use the following shorthand, 

\begin{align}
    (v_1,\ldots v_k) &= (\Ul{v}[k]) \\
    (v_{k+1},\ldots v_{k+l}) &= (\Ul{v}[k+1][k+l])
\end{align}

and \Cref{lee-chp14:wedge-product-evaluated} becomes

\begin{equation}
    \Alt{\omega\otimes\eta}(\Ul{v}[k+l]) = \dfrac{1}{(k+l)!}\sum_{\sigma\in S_{k+l}}(\sgn\sigma)\omega(\Ul{v}[\sigma({k})])\eta(\Ul{v}[\sigma({k+1})][\sigma({k+l})])
\end{equation}

\end{definition}

\begin{definition}[Elementary $k$-covectors]
    Let $V$ be a finite dimensional vector space with basis $(E_i)$, and if $(\varepsilon^i)$ is a dual basis for $V^*$, and if $I$ is a $k$-index, where $I = (\Ul{i}[k])$, then
    \[
        \varepsilon^I(\Ul{v}[k]) = \det\left(\begin{bmatrix}
            \varepsilon^{i_1}(v_1) & \cdots & \varepsilon^{i_1}(v_k)\\
            \vdots & \ddots & \ldots \\
            \varepsilon^{i_k}(v_1) & \cdots & \varepsilon^{i_k}(v_k)
        \end{bmatrix}\right) = \det\left(\begin{bmatrix}
            v_1^{i_1}  & \cdots & v_k^{i_1} \\
            \vdots     & \ddots & \ldots    \\
            v_1^{i_k}  & \cdots & v_k^{i_k}
        \end{bmatrix}\right) = \det((\varepsilon^{i_j})v_l)_{jl}
    \]
    where $\varepsilon^{i_j}(v_l)$ is the projection of $v_l$ onto the $i_j$ basis vector $E_{i_j}$. The space of all $\varepsilon^I$ where $I$ is a $k$-index are called \emph{elementary $k$-covectors}. It is clear that $\varepsilon^I$ are alternating $k$-covectors.
\end{definition}

\begin{definition}[Concatenating multi-indices]
    Let $I$ and $J$ be multi-indices, $I = (\Ul{i}[k])$ and $J = (\Ul{j}[l])$, then
    \[
    IJ = (\Ul{i}[k],\Ul{j}[l])
    \]
    is a $k+l$ multi-index obtained by concatenating $I$ and $J$.
\end{definition}

\begin{wts}[Proposition 14.8]\label{lee-chp14:prop-14.8}
    Let $V$ be a $n$-dimensional vector space with basis $(E_i)$, and if $(\varepsilon^i)$ is a dual basis for $V^*$, then $\Lambda^k(V^*)$ has the basis:
    \[
        \overline{\Epsilon}=\bigset{\varepsilon^I,\: I\text{ is an increasing }k\text{-index}}
    \]
    In particular, this means the vector space (algebra) of alternating $k$-covectors have dimension $\binom{n}{k}$. If $k=n$, then it is spanned by 
    \[
        \varepsilon^{(1,\ldots,n)}=\varepsilon^{(\underline{n})}
    \]
    and all alternating $n$-covectors are of the form $A\varepsilon^{(\underline{n})}$, where $A\in\real$. This roughly means there is, up to a scalar multiple of $\varepsilon^{\underline{n}}$, only one (oriented) way of measuring volume in a basis-independent, and dimension-independent manner.
\end{wts}

\begin{wts}[Lemma 14.10]
    Let $I$ and $J$ be multi-indices, $I = (\Ul{i}[k])$ and $J = (\Ul{j}[l])$, then
    \begin{equation}
        \varepsilon^I\wedge\varepsilon^J = \varepsilon^{IJ}
    \end{equation}
\end{wts}

\begin{wts}[Lemma 14.11, Properties of the Wedge Product]\label{lee-chp14:lemma14.11}
    The wedge product satisfies the following properties
    \begin{itemize}
        \item It is bi-linear over $\real$,
        \item It is associative,
        \item It is anti-commutative, for $\omega\in\Lambda^k(V^*)$ and $\eta\in\Lambda^l(V^*)$, then
        \[
            \omega\wedge\eta = (-1)^{kl}\eta\wedge\omega
        \]
        \item Formula for elementary covectors, let $\varepsilon^{\Ul{i}[k]}$ be covectors (covectors are assumed to have rank one), then
        \[
        \varepsilon^I\defined \varepsilon^{i_1}\wedge\cdots\wedge \varepsilon^{i_k}
        \]
        is an alternating $k$-covector.
        \item Determinant Law
        \[
            \omega^1\wedge\cdots\wedge \omega^k(\Ul{v}[k]) = \det(\omega^j(v_i))_{ji}
        \]
        where $(A^j_i)_{ji}$ denotes the matrix with entries $A^j_i$ in the $j$th row and $i$th column.
    \end{itemize}
\end{wts}

\begin{definition}[Interior multiplication]
    Let $V$ be a finite dimensional vector space.  If $x\in V$ is a vector, it induces a linear map $i_x: \Lambda^k(V^*)\to \Lambda^{k-1}(V^*)$ such that for every $\omega\in \Lambda^k(V^*)$, and $\Ul{v}[2][k]\in V$, 
    \[
        i_x(\omega)(\Ul{v}[2][k]) \defined \omega(x,\Ul{v}[2][k])
    \]
    by fixing $x$ in the first argument of $\omega$. The result is an alternating $k-1$ covector, and it is clearly linear in $\omega$ over $\real$. We also write
    \[
        i_x(\omega) = x\hook \omega
    \]
\end{definition}


\begin{definition}[Graded algebra]
    An algebra $A$ is said to be \emph{graded} if it has a direct sum decomposition $A = \bigoplus_{k\in \mathbb{Z}}A^k$ such that the algebra-product $(\cdot,\cdot): A\times A\to A$ satisfies 
    \[
        (a,b)\in A^{k+l}\quad\text{for every }a\in A^k,\: b\in A^l
    \]
    A graded algebra is said to be \emph{anti-commutative}, if the algebra product satisfies
    \[
        (a,b) = (-1)^{kl}(b,a)
    \]
    We see that $\bigoplus_{k\in \mathbb{Z}}\Lambda^k(V^*)$ is a graded algebra, if we define $\Lambda^k(V^*)=\{0\}$ for $k\leq -1$, or $k\geq n+1$. 

\end{definition}

\begin{wts}[Lemma 14.13, Properties of interior multiplication]
    Let $V$ be a finite dimensional vector space and fix $x\in V$,
    \begin{enumroman}
        \item $i_x\circ i_x\equiv 0$, where we interpret $i_v$ as a linear map on the entire graded algebra.
        \item $i_x$ satisfies some kind of product rule. For every $\omega\in \Lambda^k(V^*)$ and $\eta\in\Lambda^l(V^*)$,
        \[
            i_x(\omega\wedge\eta) = (i_x\omega)\wedge\eta + (-1)^{k}\omega\wedge(i_x\eta)
        \]
    \end{enumroman}
\end{wts}

\topheader{Differential Forms on Manifolds}
    \begin{definition}[Differential $k$-form on $M$]
        Let $M$ be a smooth manifold with or without boundary, and $p\in M$ be fixed, $\lambda^k(T_p^*M)$ is a linear subspace of $T^k(T_p^*M)$ of dimension $\binom{n}{k}$, and
        \[
            \Lambda^kT^*M \defined \coprod_{p\in M}\Lambda ^k(T^*_p M)
        \]
        is a smooth vector bundle of rank $\binom{n}{k}$ over $M$. \\

        A differential $k$-form on $M$ is a $k$-tensor field (or a smooth global section of the $k$-tensor bundle $T^kT^*M$) that is \emph{pointwise alternating}, or equivalently: it is a smooth global section of the vector bundle $\Lambda^kT^*M$. The space of differential $k$-forms on $M$ is denoted by 
        \[
            \Omega^k(M) = \Gamma(\Lambda^kT^*M)
        \]
    \end{definition}
    \begin{definition}[Graded algebra of all $k$-forms]
        $\Omega^k(M)$ is a vector space over $\real$, and a left $C^\infty(M)$ module. If we stick with scalars over $\real$, and define the wedge product of two tensor fields $\omega$, $\eta\in \Omega^k(M)$ (recall these are smooth global sections of some vector bundle),
        \[
            (\omega\wedge\eta)_p\defined \omega_p\wedge\eta_p
        \]
        We can inherit all the properties of the pointwise wedge product (see \Cref{lee-chp14:lemma14.11}), namely \begin{itemize}
            \item Bi-linearity over $\real$,
            \item Associativity,
            \item Anti-commutativity
        \end{itemize}
        With this, the direct sum of all $\Omega^k(M)$ forms an associative, anticommutative graded algebra over $\real$, with the tensor wedge product as the algebra product.
        \[
            \Omega^*(M) = \bigoplus_{k\geq 0}\Omega^k(M)
        \]
    \end{definition}
    
    \begin{definition}[{$\Omega^k(M)$} in local coordinates]
        Let $\omega\in\Omega^k(M)$, and $(x^i)$ be the local coordinates on some open subset $U$ of $M$. If 
        \begin{equation}\label{differential-form-lcoal-coordinate frame}
            \biggl(\pdv{x^1},\ldots,\pdv{x^n}\biggr)\quad\text{is a coordinate frame on }TU \defined \pi^{-1}(U)
        \end{equation}
            
        where $\pi: TM\to M$ is the canonical projection of the tangent bundle. Then,
        \[
            \biggl(dx^1,\ldots, dx^n\biggr)\quad\text{is the dual co-frame to the frame in  \Cref{differential-form-lcoal-coordinate frame}.}
        \]
        by \Cref{lee-chp14:prop-14.8}, we can write $\omega$ uniquely as a linear combination (over $C^\infty(M)$) of $k$-tensors which are \emph{increasing elementary $k$-covectors} pointwise.
        \[
            \omega = \isum_I \omega_I dx^{i_1}\wedge\cdots\wedge dx^{i_k} = \isum_I \omega_Idx^{I}
        \]
        where $\isum_I$ a sum over all increasing $k$-indices, and $dx^{I}$ is the shorthand for the wedge product above, and $\omega_I$ is a number obtained by evaluating $\omega$ at the following
        \[
            \omega_I \defined \omega\qty(\pdv{x^{i_1}},\ldots, \pdv{x^{i_k}}),\qqtext{where} I = (i_1,\ldots i_k)
        \]
    \end{definition}
    Since differential $k$-forms are (smooth) $k$-tensor fields, the pullback $F^*: \Omega^k(N)\to \Omega^k(M)$ is well defined, in particular,
    \begin{wts}[Lemma 14.16, Properties of $k$-form pullback]
           Let $F:M\to N$ be a smooth map, then
           \begin{itemize}
               \item The $k$-form pullback is linear over $\real$, 
               \item The $k$-form pullback commutes with wedge products, or 
               \[
                F^*(\omega\wedge\eta) = F^*\omega\wedge F^*\eta
               \]
               \item In any smooth chart, if $(y^i)$ are the local coordinates of $N$. For any $k$-form on $N$ $\omega = \isum_I \omega_Idy^{i_1}\wedge\cdots\wedge dy^{i_k} = \isum_I \omega_Idy^I$,
               \[
                    F^*\qty(\isum \omega_I dy^{i_1}\wedge\cdots\wedge dy^{i_k}) = \isum_I(\omega_I\circ F)d(y^{i_1}\circ F)\wedge\cdots\wedge d(y^{i_k}\circ F)
               \]
           \end{itemize}
    \end{wts}
    \begin{proof}
        Linearity of the $k$-form pullback over $\real$ is an immediate consequence of the linearity of the tensor field pullback. To prove the second claim, notice 
        \[
            F^*(A\otimes B) = F^*(A)\otimes F^*(B)
        \]
        and
        \begin{align*}
            F^*(\Alt{A\otimes B}) &= F^*\qty(\dfrac{1}{(k+l)!}\sum_{\sigma\in S_{k+l}}(\sgn\sigma)\sigma(A\otimes B))\\
            &= \dfrac{1}{(k+l)!}\sum_{\sigma\in S_{k+l}}(\sgn\sigma)F^*(\sigma(A\otimes B))
        \end{align*}
        The tensor pullback commutes with permutations, let $C$ be a $k$-tensor field on $N$, and $\sigma\in S_k$, then
        \[
            (\sigma F^*(C))_p(\Ul{v}[k]) = (\sigma C) (dF_p(\Ul{v}[k])) = C(dF_p(\Ul{v}[\sigma(k)])) = F^*(\sigma C)_p(\Ul{v}[k])
        \]
        Hence,
        \begin{align*}
            F^*(\Alt{A\otimes B}) &= \dfrac{1}{(k+l)!}\sum_{\sigma\in S_{k+l}}(\sgn\sigma)\sigma\qty[ F^*(A\otimes B)]\\
            &= \dfrac{1}{(k+l)!}\sum_{\sigma\in S_{k+l}}(\sgn\sigma)\sigma\qty[ F^*(A)\otimes F^*(B)]\\
            &= \Alt{F^*(A)\otimes F^*(B)}
        \end{align*}
        Since the wedge product differs by the alternating product by a constant, $F^*$ commutes with the wedge product. The general case follows upon induction, and associativity of the wedge product.\\

        For the third claim, use linearity, and the pullback of the wedge product is the wedge product of the pullback, noting that each $y^{i_j}$ is a smooth function on $N$, so $F^*(dy^{i_j}) = d(y^{i_j}\circ F)$ by Proposition 11.25b in Lee, we state this Proposition for completeness:

        \begin{quote}
            Let $F:M\to N$ be a smooth map, and $u$ is a smooth function on $N$, then
            \[
                F^*du = d(u\circ F)
            \]
        \end{quote}
        
    \end{proof}

    
    \begin{wts}[Corollary 14.21, Tensor Pullback of Top-Degree Forms]\label{lee-chp14:corollary-14.21}
        Let $F: M\to N$ be a smooth map between manifolds with or without boundary \emph{of the same rank $n$}. If $(x^i)$ and $(y^j)$ are local coordinates on open subsets $U$ and $V$ of $M$ and $N$ respectively, for every $u\in C(V,\real)$, 
        \[
            F^*(udy^1\wedge \cdots\wedge dy^n)= (u\circ F)(\det DF)dx^1\wedge\cdots\wedge dx^n
        \]
        on an open set $U\cap F^{-1}(V)$, where $DF$ is the Jacobian matrix of the coordinate representation of $F$. 
    \end{wts}
    \begin{proof}
        The proof uses the fact that both manifolds are of dimension $n$, and any $n$-form on $U$ is necessarily of the form
        \begin{equation}\label{lee-chp14:corollary-14.21-LHS-unevaluated}
            gdx^1\wedge\cdots\wedge dx^n
        \end{equation}
        where $g$ is some continuous, real valued function on $U$. Solving for $g$ pointwise yields the result. Suppose $p\in U\cap F^{-1(V)}$, and evaluating \cref{lee-chp14:corollary-14.21-LHS-unevaluated} with the coordinate derivations $\qty(\eval{\pdv{x^{\underline{n}}}}_p)$ reads
        \begin{align*}
            \qty(gdx^1\wedge\cdots\wedge dx^n)_p\qty(\eval{\pdv{x^{\underline{n}}}}_p) &= g(p)(dx^1\wedge\cdots\wedge dx^n)_p\qty(\eval{\pdv{x^{\underline{n}}}}_p) &&\text{(pw. eval. of tensor field)}\\
            &= g(p)dx^1_p\wedge\cdots\wedge dx^n_p\qty(\eval{\pdv{x^{\underline{n}}}}_p) && \text{(wedge pw. is pw. wedge)}\\
            &= g(p)\det(dx^i_p\qty(\pdv{x^j}))_{ij}&& \text{(det. law from \Cref{lee-chp14:lemma14.11})}\\
            &= g(p) \det(\delta^{i}_j)_{ij}\\
            &= g(p)
        \end{align*}
        
        Let $F^i = y^i\circ F$ be the $i$th coordinate component of $F$ (which is a smooth function on $M$). So that $dF^i$ and $\pdv{x^j}$ form a covector-vector pair if evaluated pointwise. Apply the determinant law from \Cref{lee-chp14:lemma14.11} for every $p\in U$ reads
        \begin{equation}\label{lee-chp14:corollary14.21-determinant}
            (dF^1_p\wedge\cdots\wedge dF^n_p)\qty(\eval{\pdv{x^1}}_{p},\ldots,\eval{\pdv{x^n}}_{p}) = \det(\qty[dF^i_p\qty(\eval{\pdv{x^j}}_p)]_{ij})
        \end{equation}
        and
        \[
            dF^1\wedge\cdots\wedge dF^n\qty(\pdv{x^1},\ldots,\pdv{x^n}) = \det(dF^i\qty(\pdv{x^j}))
        \]
        Using the tensor pullback formula from Chapter 12, and by Lemma 14.16, 
        \begin{align*}
            F^*(udy^1\wedge\cdots\wedge dy^n) &= (u\circ F)d(y^1\circ F)\wedge\cdots\wedge d(y^n\circ F)\\
            &= (u\circ F)dF^1\wedge\cdots\wedge dF^n
        \end{align*}
        Evaluating the last expression using the coordinate frame on   $\qty(\pdv{x^1},\ldots,\pdv{x^n})$ at some $p\in U\cap F^{-1}(V)\subseteq U$,
        \begin{align*}
            F^*(udy^1\wedge\cdots\wedge dy^n)_p\qty(\eval{\pdv{x^{\underline{n}}}}_p) &= \underbrace{(u\circ F)(p)}_{\text{scalar}}(dF^1\wedge\cdots\wedge dF^n)_p\qty(\eval{\pdv{x^{\underline{n}}}}_{p}) && \text{(pw. eval. of tensor field)}\\
            &= (u\circ F)(p)(dF^1_p\wedge\cdots\wedge dF^n_p)\qty(\eval{\pdv{x^{\underline{n}}}}p) && \text{(wedge pw. is pw. wedge)}\\
            &= (u\circ F)(p)\det(dF^{i}\qty(\pdv{x^j}))_p && \text{(\Cref{lee-chp14:corollary14.21-determinant})}
        \end{align*}
        Equating the two expressions we see that
        \[
            g(p) = (u\circ F)(p)\det(dF^{i}\qty(\pdv{x^j}))_p = (u\circ F)(p)\det(DF)_p
        \]
    \end{proof}
\topheader{Exterior Derivatives}
\begin{definition}[Exterior Derivative on Euclidean Space]
    Let $U$ be an open subset of $\realn$ or $\halfn$, and $(x^i)$ be the local coordinates in $U$. If $\omega\in\Omega^k(U)$, 
    \[
        \omega = \isum_J\omega_Jdx^J
    \]
    and we define the \emph{exterior derivative of $\omega$} by
    \[
        d\omega = d\qty(\isum_J\omega_J dx^J) = \isum_J d\omega_J\wedge dx^J
    \]
    where $d\omega_J$ is the Chapter 11 differential of the smooth function $\omega_J$, and $d\omega_J$ is a covector field on $U$. If $p\in u$ and $v\in T_p U$, 
    \[
        d(\omega_J)_p(v) \defined v(\omega_J)
    \]
    Since $\Omega^1(U) = \cvField(U)$, the wedge product $d\omega_J\wedge dx^J = d\omega_J\wedge dx^{j_1}\wedge\cdots\wedge dx^{j_k}$ is alternating. \\

    With this, $d: \Omega^{k}(\realn)\to \Omega^{k+1}(\realn)$ (resp. $\halfn$) is a linear map over $\real$. We can rewrite $d\omega_J$ in terms of the coframe on $U$, suppose
    \[
        d\omega_J = \pdv{\omega_J}{x^i}dx^i\implies d\omega_J\wedge dx^J = \pdv{\omega_J}{x^i}dx^i\wedge dx^J = \sum_{i=1}^n\pdv{\omega_J}{x^i}dx^i\wedge dx^J
    \]
    Summing over all increasing $k$-indices $J$, we get
    \[
        d\omega = d\qty(\isum_J \omega_j dx^J) = \isum_J\sum_{i=1}^n\pdv{\omega_J}{x^i}dx^i\wedge dx^{j_1}\wedge\cdots\wedge dx^{j_k}
    \]
\end{definition}
\end{document}