\documentclass[../main-manifolds.tex]{subfiles}

\begin{document}
\providecommand{\szz}{\mathcal{S}}
\providecommand{\ccinf}{C_c^\infty}

% Topologies
\providecommand{\Taux}{\Tau_\xx}
\providecommand{\Tauy}{\Tau_\yy}
\providecommand{\Tauxy}{\Tau_{\xx\times\yy}}

% Basis
\providecommand{\Bx}{\borel_\xx}
\providecommand{\By}{\borel_\yy}
\providecommand{\Bxy}{\borel_{\xx\times\yy}}


\fchapter{1: Topological Manifolds}
\newpage
\topheader{Topological Manifolds}
The study of differential geometry begins with tens of pages of definitions.
\begin{definition}[Topological Manifold]\label{lee-chp1:topological-manifold-definition}
    Let $M$ be a topological space. $M$ is a topological manifold of dimension $m$ if it is Hausdorff, second-countable, and locally homeomorphic to $\realn$.
\end{definition}

\begin{definition}[Local homeomorphism]\label{lee-chp1:locally-homeomorphic-definition}
    $M$ locally homeomorphic to $\realn$ if every point $x\in M$ an open set $U$, equipped with a homeomorphism which sends points in $U$ into an open subset of $\realn$. 
    \[
        \phi: U\to \phi(U)
    \] 
    The tuple $(U,\phi)$ is called a coordinate chart.
\end{definition}

\begin{definition}[More on coordinate charts]\label{lee-chp1:coordinate-chart-definition}
    \begin{itemize}
        \item A coordinate chart $(U,\phi)$ is centered at $p\in M$ if $p\in U$ and $\phi(p)=0\in\realn$.
        \item We call $U$ the coordinate domain, and
        \item we call $\phi$ the coordinate map.
        \item If the choice of $(U,\phi)$ is unambiguous, then the local coordinates of $p$ are simply the coordinates of $\phi(p)$ in $\realn$, and
        \item we sometimes also denote $\phi(U)$ by $\hat{U}$ if it is unambiguous to do so.
        \item If $\hat{U}$ is an open ball/cube, then $U$ is called a coordinate ball/cube.
    \end{itemize}
\end{definition}

The central theme of point-set topology (or even metric topology) is that of passing a topological argument to the basis or to a neighbourhood. Manifolds in particular have a nice basis.
\begin{wts}[Basis of precompact coordinate balls]
    Every topological manifold has a countable basis of precompact coordinate balls.
\end{wts}
\begin{wts}[Additional facts about topological manifolds]
If $M$ is a topological manifold, 
\begin{itemize}
    \item $M$ is locally compact. (Lee, Proposition 1.12)
    \item $M$ is paracompact, and every open cover has a refinement 
    that is another countably locally finite open cover whose elements are chosen from an arbitrary (but fixed) basis of $M$. (Lee, Theorem 1.15)
    \item $M$ is locally-path connected.
    \item $M$ is connected iff it is path-connected.
    \item $M$ is metrizable. (Munkres Chapter 6)
\end{itemize}
\end{wts}

\topheader{Smooth Manifolds}
We wish to perform calculus on manifolds.
\begin{definition}[Smooth function $F:\realn\to\realm$]\label{lee-chp1:real-smooth-function}
    Let $F:\realn\to\realm$, replacing $\realn$ and $\realm$ with open subsets if necessary. $F$ is smooth if its (scalar-valued) component functions has continuous partial derivatives of all orders. The set of smooth functions from $\realn$ to $\realm$ is sometimes denoted by $\cinf[\realn,\realm]$. If $m=1$, we sometimes write $\cinf[\realn]$, similar to a test function on the Schwartz Space.
\end{definition}

\begin{definition}[Transition map from $\phi$ to $\psi$]\label{lee-chp1:transition-maps}
    Let $(U,\phi)$ and $(V,\psi)$ be coordinate charts on $M$. The composite function (whenever $U\cap V\neq\varnothing$) 
    \[
    \psi\circ\phi^{-1}:\phi(U\cap V)\to\psi(U\cap V)
    \] 
    is called the transition map. Notice $\psi\circ\phi^{-1}$ is by definition a homeomorphism.
\end{definition}

\begin{definition}[Smoothly compatiable]\label{lee-chp1:smoothly-compatible}
    Two coordinate charts on $M$, $(U,\phi)$ and $(V,\psi)$ are called smoothly compatible if either their domains are disjoint, or their transition map is a diffeomorphism on $\realm$.
\end{definition}

\begin{definition}[Smooth atlas]\label{lee-chp1:smooth-atlas}
    An atlas $\acal$ of $M$ is a collection of charts $\{(U_\alpha, \phi_\alpha)\}$ whose collection of coordinate domains $\{U_\alpha\}$ for an open cover of $M$.\\ It is called a smooth atlas if any two charts in the atlas are pairwise smoothly compatible.
\end{definition}

\begin{definition}[Smooth manifold]\label{lee-chp1:smooth-manifold}
    A smooth atlas $\acal$ on $M$ is maximal if it is not contained (properly) in any other smooth atlas as a subset. In other words, if $(U',\phi')$ is a chart on $M$ that is smoothly compatible with all elements in $\acal$, then $(U',\phi')\in\acal$ already.\\
    
    This smooth atlas is often very large, it includes all translations of charts, dilations, and composition with diffeomorphisms in $\realm$, restrictions onto open subsets, etc. A maximal smooth atlas is sometimes called a complete atlas, or a smooth manifold structure.\\

    A smooth manifold is the tuple $(M,\acal)$, where $\acal$ is some smooth atlas. It can happen if $M$ is originally a topological manifold with a huge number of charts, some of which are smoothly compatible with others, that $\acal$ is a strict subset, and both $\acal_1$ and $\acal_2$ are maximal smooth atlases on $M$, but $\acal_1\neq\acal_2$. We often omit $\acal$ and write $M$ if the smooth atlas is understood or not of importance.
\end{definition}

\begin{definition}[Smooth coordinate terminologies]\label{lee-chp1:smooth-coordinate-definitions}
    Let $(M,\acal)$ be a smooth manifold. 
    \begin{itemize}
        \item Any coordinate chart $(U,\phi)\in\acal$ is called a smooth chart, similar to \cref{lee-chp1:coordinate-chart-definition}
        \item We call $U$ the \emph{smooth coordinate domain} or \emph{smooth coordinate neighbourhood} of any $p\in U$, and
        \item we call $\phi$ the \emph{smooth coordinate map}.
        \item The terms \emph{smooth coordinate ball} and \emph{smooth coordinate cube} are used similarly.
        \item A set $B\subseteq M$ is a \emph{regular coordinate ball} if its image is a smooth coordinate ball centered at the origin; and the closure of this ball in $\realm$ is a subset of the image of another smooth coordinate ball, centered at the origin.
    \end{itemize}
\end{definition}

\begin{definition}[Standard smooth structure on $\realn$]\label{lee-chp1:standard-smooth-structure-realn}
    The maximal smooth atlas containing $(\realn,\id{\realn})$ is called the \emph{standard smooth structure on $\realn$}.
\end{definition}

Manifolds with boundary are not that important as regular manifolds, but they are worth mentioning.
\begin{definition}[Closed n-dimensional upper half-plane $\halfn\subseteq\realn$]\label{lee-chp1:upper-half-plane-definitions}
    We define the following symbols for the upper half plane.
    \begin{itemize}
        \item $\halfn = \bigset{x\in\realn,\: x^n\geq 0}$,
        \item $\Int\halfn = \bigset{x\in\realn,\: x^n > 0}$,
        \item $\partial\halfn = \bigset{x\in\realn,\: x^n = 0}$
    \end{itemize}
\end{definition}
\begin{definition}[Manifolds with boundary]\label{lee-chp1:manifolds-with-boundary-definition}
    A topological space $M$ is called a manifold with boundary if it is Hausdorff, second-countable, and locally homeomorphic to an open subset of $\halfn$ (endowed with the subspace topology from $\realn$).\\

    A chart $(U,\phi)$ is an \emph{interior chart} if its coordinate image is disjoint from the 'boundary' of the upper-half plane. This means $\phi(U)\cap \partial\halfn=\varnothing$. Similarly, $(V,\psi)$ is a \emph{boundary chart} if its range contains a point in $\partial\halfn$; so $\psi(V)\cap\partial\halfn\neq\varnothing$.\\

    Similar to \cref{lee-chp1:coordinate-chart-definition} and \cref{lee-chp1:smooth-coordinate-definitions}, we use the terms \emph{coordinate half-ball}, \emph{coordinate half-cube}, \emph{regular coordinate half-ball}.\\

    Let $p\in M$, it is called an \emph{interior point of $M$} (not to be confused with the topological interior) if it is in the domain of some interior chart, and $p$ is called a \emph{boundary point of $M$} if there exists a boundary chart that sends $p$ into $\partial\halfn$. The set of interior points and boundary points of $M$ will be denoted by $\Int M$ and $\partial M$.
\end{definition}

\newpage

\begin{example}[Sphere as a topological manifold]
    The $n$-sphere as a topological manifold. Define 
\[
    S^n = \bigset{x\in\real^{n+1},\: |x|=1}
\]
We claim that $\{U_i^{\pm}\}_{i=1}^{n+1}$ form an open cover, where
\[
    U_i^+ = \bigset{x\in S^n, x^i>0}\quad U_i^- = \bigset{x\in S^n, x^i<0}
\]
Each $U_i^{\pm}$ is the inverse image of $\pnv{i}{(0,+\infty)}\cap S^n$ or $\pnv{i}{(0,-\infty)}\cap S^n$, hence open. For every $x\in S^n$, there exists at least some $1\leq j\leq n+1$ that makes the $j$-th coordinate of $x$, $x^j\neq 0$. So 
\[
    S^n = \bigcup_{i}U_i^\pm
\]
Denote the unit ball $\bigset{x\in\realn,\: |x|<1}$ in $\realn$ by $\borel^n$. 

\end{example}

\newpage

\fchapter{2: Smooth Maps}\newpage

\topheader{Smooth Maps}

\begin{definition}[Smooth functions {$\cinf[M,\real^k]$}]\label{lee-chp2:test-functions-on-manifolds}
    Let $F: M\to\real^k$ be a vector-valued function on a smooth manifold $M$. We say $F$ is a smooth function if for every $p\in M$, there exists a smooth chart $p\in (U,\phi)$ such that the \emph{coordinate representation of $F$ at $p$, with respect to $(U,\phi)$} is a smooth function from $\realm$ to $\real^k$, denoted by $\hat{F}$ (in the sense of \Cref{lee-chp1:real-smooth-function}).
    \[
        \hat{F}=F\circ \phi^{-1}:\phi(U)\to\real^k\in \cinf[\phi(U),\realk]
    \]
    if $k=1$, then we denote the space of \emph{test functions} on $M$ by $\cinf[M]=\cinf[M,\real]$
\end{definition}

\begin{definition}[Smooth maps between manifolds {$\cinf[N,M]$}]\label{lee-chp2:smooth-maps-between-manifolds-definition}
    Let $F:N\to M$ be a map between smooth manifolds $N$ and $M$ (note we switched the order). $F$ is a smooth map if at every $p\in M$, there exists 
    \begin{itemize}
        \item a chart in the smooth atlas of $N$ (the domain), $p\in (U,\phi)$,
        \item another chart in the smooth atlas of $M$ (the range), $F(U)\subseteq(V,\psi)$,
        \item such that, the \emph{coordinate representation of $F$ at $p$ with respect to $(U,\phi)$, and $(V,\psi)$} is a smooth function from $\realn$ to $\realm$, also denoted by $\hat{F}$.
        \begin{equation}\label{lee-chp2:eq-smooth-map-coordinate-representation}
            \hat{F} = \psi\circ F\circ \phi^{-1}:\phi(U)\to \psi(V)\in \cinf[\realn,\realm]
        \end{equation}
    \end{itemize}
\end{definition}

The following propositions summarizes common operations on smooth maps, a few sources of them.
\begin{wts}[Smooth maps are continuous]\label{lee-chp2:smooth-maps-are-continuous}
    If $F:N\to M$ is a smooth map, then $F$ is continuous with respect to the topologies on $N$ and $M$.
\end{wts}
\begin{proof}
    Let $p\in N$ be fixed, because $F$ is smooth this induces two smooth charts, one in the domain and another in the range; as in \cref{lee-chp2:smooth-maps-between-manifolds-definition}. $F(p)$ is a point in $M$. From \cref{lee-chp2:eq-smooth-map-coordinate-representation}, $\hat{F}|\phi(U)$ is a smooth hence continuous function. Since $\phi: U\mapsto \phi(U)$ and $\psi: V\mapsto \psi(V)$ are homeomorphisms, 
    \[
        F|U = \underbracket{\psi^{-1}}_{\text{continuous}}\circ \underbracket{\hat{F}|\phi(U)}_{\text{smooth}}\circ \underbracket{\phi}_{\text{continuous}}\qq{is continuous on }U
    \]
    Let the point $p$ range through all the points in $N$, so $F$ is continuous at every $p$, hence on $N$.
\end{proof}
\begin{wts}[Characterizations of Smooth Maps]{$\cinf[N,M]$}]\label{lee-chp2:characterizations-of-smooth-maps}
    Let $N$ and $M$ be smooth manifolds, and $F:N\to M$. $F$ is a smooth map iff
    \begin{itemize}
        \item For every $p\in N$, there exists smooth charts $p\in (U,\phi)$ and $F(p)\in (V,\psi)$ such that $U\cap F^{-1}(V)$ is an open set in $N$, and the composite map (the coordinate representation)
        \[
            \psi\circ F\circ \phi^{-1}|(U\cap F^{-1}(V)):\phi(U\cap F^{-1}(V))\to \psi(V)\qq{is smooth}
        \]
        \item $F$ is continuous and there exist smooth atlases $\{(U_\alpha,\phi_\alpha)\}\subseteq\acal_N$, and $\{(V_\beta,\psi_\beta)\}\subseteq\acal_M$ such that the coordinate representation 
        \[
            \psi_\beta\circ F\circ \phi_\alpha^{-1}:\phi(U_\alpha\cap F^{-1}(V_\beta))\to \psi(V_\beta)\qq{is smooth}
        \]
        whenever it makes sense.
        \item the restriction of $F$ onto any arbitrary open set $U$, $F|U: U\mapsto M$ is smooth (in the sense of open submanifold).
    \end{itemize}
\end{wts}
\begin{proof}
    By \Cref{lee-chp2:smooth-maps-are-continuous}, and the fact that complete atlases are closed under restrictions onto open sets, it is clear that the original definition implies the two. The first definition also clearly implies the original definition, as we can restrict 
    \[(U,\phi)\mapsto \biggl(U\cap F^{-1}(V),\phi|(U\cap F^{-1}(V))\biggr)\]
    since $U\cap F^{-1}(V)$ is open in the domain manifold.\\

    The second definition implies the original one as well, since the smooth atlases are taken from the maximal atlas, we can pass the argument to any smoothly-compatible chart. Atlases must cover both the domain and the range, and coordinate transitions between smoothly compatible charts are diffeomorphisms. If $F$ is smooth on a subcollection of those charts, meaning
    \[
        \psi_\beta\circ F\circ\phi_\alpha^{-1}\in \cinf[\phi_\alpha(U_\alpha)\cap F^{-1}(V_\beta),\psi_\beta(V_\beta)]
    \]
    it is smooth with respect to every pair of (smooth) charts in the two atlases $\acal_{N}$, $\acal_{M}$, as a composition of smooth maps:
    \[
        \psi\circ \underbracket{\psi^{-1}\circ \psi_\beta}_{\text{smooth}}\circ F\circ\underbracket{\phi_\alpha^{-1}\circ \phi}_{\text{smooth}}\circ\phi^{-1}
    \]
    where we can restrict $\phi_\alpha\mapsto \phi_\alpha|(U_\alpha\cap F^{-1}(V_\beta))$ by continuity of $F$.

    I will prove the third and last equivalence later.
\end{proof}
\begin{wts}[Sources of smooth maps]\label{lee-chp2:sources-of-smooth-maps}
    Let $N$, $M$, $P$ be smooth manifolds, then
    \begin{itemize}
        \item Every constant map is smooth,
        \item The identity map $\id{M}:M\to M$ is smooth,
        \item The inclusion map $\iota: W\to M$ is smooth, where $W$ is an open submanifold of $M$.
        \item The composition of smooth maps is again a smooth map: if $F\in \cinf[N,M]$ and $G\in \cinf[M,P]$, then $(G\circ F)\in\cinf[N,P]$
    \end{itemize}
\end{wts}

\topheader{Diffeomorphisms}
\begin{definition}[Diffeomorphism between Manifolds $\mathcal{D}(N,M)$]\label{lee-chp2:diffeomorphism-definition}
    Let $N$ and $M$ be smooth manifolds, $F:N\to M$ is a diffeomorphism if it is a smooth bijective map with a smooth inverse. We denote the space of diffeomorphisms from $N$ to $M$ by $\diffeo[N,M]$.
\end{definition}
\begin{wts}[Properties of Manifold Diffeomorphisms]\label{lee-chp2-diffeomorphism-properties}
    Let $N$, $M$ and $P$ be smooth manifolds, then
    \begin{itemize}
        \item The composition of diffeomorphisms is again a diffeomorphism, that is, if $F\in \diffeo[N,M]$ and $G\in \diffeo[M,P]$, then $(G\circ F)\in \diffeo[N,P]$.
        \item The open-manifold restriction of a diffeomorphism onto its image is again a diffeomorphism,
        \item Every diffeomorphism is a homeomorphism and an oepn map.
    \end{itemize}
\end{wts}
\begin{proof}
    Trivial.
\end{proof}

\topheader{Partitions of Unity}
See Folland Chapters 4 and 8. Including Urysohn's Lemma, Tietze extension, the usual construction of $\ccinf$ bump functions.
\newpage

\fchapter{3: Tangent Spaces}\newpage
\topheader{Algebra of Germs on $\cinf[N]$}
The tangent space is a powerful concept that acts almost like the dual in distribution theory.
\begin{definition}[Algebra of Germs at $p$: $C_p^\infty(N)$]\label{lee-chp3:algebra-of-germs-at-p}
    Let $N$ be a smooth manifold and $p\in N$. We define an equivalence relation on the space of test functions on $N$, $\cinf[N]$. If $f,g\in\cinf[N]$, we write $f\sim g$ if $f=g$ for some open neighbourhood about $p$. We denote this equivalence class by $C_p^\infty(N)$, and it is clear $\cinf[N]$ is closed under pointwise multiplication by the product rule, and form an algebra; so $C_p^\infty(N)$ is an algebra too.
\end{definition}

\topheader{Tangent space at $p$: $T_p N$}
\begin{definition}[Vector space of derivations at $p$: $T_p N$]\label{lee-chp3:vector-space-of-derivations-at-p}
    Let $\nu: C_p^\infty(N)\to\real$ be a linear functional on the vector space of germs at $p$. It is called a derivation at $p$ if it satisfies the product rule, if $f,g\in C_p^\infty(N)$, 
    \[
        \nu(fg)=g(p)\nu(f)+f(p)\nu(g)
    \]
    then we say 
    \begin{itemize}
        \item $\nu$ is a tangent vector at $p$, 
        \item $\nu\in T_p N$,
        \item $\nu$ is an element of the \emph{tangent space at $p$}.
        \item $\nu$ is a derivation on $N$ at $p$.
    \end{itemize}
\end{definition}
\begin{wts}[Properties of derivations at $p$]
    Let $N$ be a smooth manifold and $p\in N$.
    \begin{itemize}
        \item If $f\in C^\infty_p$ is constant in some neighbourhood of $p$, then $\nu(f)=0$ for every $\nu\in T_pN$,
        \item If $f(p)=g(p)=0$, then $\nu(fg)=0$ for tangent vector $\nu$ at $p$.
    \end{itemize}
\end{wts}

\topheader{Tangent space in $\realn$}

\begin{wts}[Basis of $T_p\realn$]
    Let $\realn$ be equipped with the standard smooth structure as in \Cref{lee-chp1:standard-smooth-structure-realn}. The vector space of derivations at $p\in\realn$ are spanned by the $n$ partial derivatives at $p$
    \[
        \eval{\pdv{x^j}}_{p}:f\mapsto \eval{\pdv{x^j}f(x)}_{p},\quad 1\leq j\leq n,\: f\in \cinf[\realn]
    \]
    Moreover, the $n$ vectors form a basis, and $\dim T_p\realn= n$.
\end{wts}
\begin{definition}[Standard Basis of $T_p\realn$]
    The standard basis for the tangent space at $p\in\realn$ is the $n$ partial derivatives at $p$.
    \begin{equation}\label{lee-chp3:standard-basis-for-tangent-space-realn}
        T_p\realn = \bigset{\eval{\pdv{x^1}}_{p},\ldots,\eval{\pdv{x^n}}_{p}}
    \end{equation}
\end{definition}
\topheader{Differential of a smooth map $F\in \cinf[N,M]$}


We will go through the section on the Change of Coordinates, and how different coordinate charts change the representation of a derivation at $p\in M$, where $M$ is some smooth  manifold.
\begin{wts}
    Let $M$ be a smooth manifold, and fix $p\in M$. If $\nu\in T_pM$ is given with respect to the bases
    \[
        \bigset{\eval{\pdv{x^1}}_{p},\ldots,\eval{\pdv{x^m}}_{p}}\qq{and}\bigset{\eval{\pdv{y^1}}_p,\ldots,\eval{\pdv{y^m}}_{p}}
    \]
    Defined by 
    \[\eval{\pdv{x^j}}_{p}\defined d\qty(\eval{\phi^{-1}}_{\phi(p)})\qty(\eval{\pdv{x^j}}_{\phi(p)})
    \qq{and}
    \eval{\pdv{y^j}}_{p}\defined d\qty(\eval{\psi^{-1}}_{\psi(p)})\qty(\eval{\pdv{y^j}}_{\psi(p)})\]
    and we write $\nu$ in terms of the first basis
    \[
        \nu = \nu^j\eval{\pdv{x^j}}_{p} = \sum_{j=1}^m \nu^j\eval{\pdv{x^j}}_{p}
    \]
    and the second basis
    \[
        \nu = \nu^j\eval{\pdv{y^k}{x^j}}_{\phi(p)}\eval{\pdv{y^k}}_{p} = \sum_{k=1}^m\sum_{j=1}^m \nu^j\eval{\pdv{y^k}{x^j}}_{\phi(p)}\eval{\pdv{y^k}}_p
    \]
    If $f\in \cinf[M]$, then 
    \[
        \nu(f) = \nu^j\eval{\pdv{x^j}}_{p} f =\nu^j\eval{\pdv{y^k}{x^j}}_{\phi(p)}\eval{\pdv{y^k}}_{p} f
    \]
\end{wts}
\begin{proof}
    Recall $\eval{\pdv{x^j}}_{p}f \defined \eval{\pdv{x^j}}_{\phi(p)} f\circ\phi^{-1}$, similarly for $\eval{\pdv{y^j}}_{p}f$. Deriving $f$ and $p$ and by vector space operations on $T_pM$, the first basis expansion gives
    \begin{equation}\label{lee-chap-1-basis-expansion-1}
        \nu^j\eval{\pdv{x^j}}_{p}f = \nu^j\eval{\pdv{x^j}}_{\phi(p)}f\circ\phi^{-1}
    \end{equation}
    and the second expression reads
    \begin{equation}\label{lee-chap-1-basis-expansion-2}
        \nu^j\eval{\pdv{y^k}{x^j}}_{\phi(p)}\eval{\pdv{y^k}}_p f = \nu^j\eval{\pdv{y^k}{x^j}}_{\phi(p)}\eval{\pdv{y^k}}_{\psi(p)}f\circ\psi^{-1}
    \end{equation}
    
    Since $f\circ\phi^{-1}\in \cinf[\realm,\real]$, we see the expressions are indeed equal. By the chain rule, if
    \[
        \psi\circ\phi^{-1}(x^1,\ldots x^m) = (y^1,\ldots y^m)
    \]
    then
    \[
        D(\psi\circ\phi^{-1})(\phi(p)) = \begin{bmatrix}\eval{\pdv{y^1}{x^1}}_{\phi(p)} & \eval{\pdv{y^1}{x^2}}_{\phi(p)} & \cdots & \cdots & \eval{\pdv{y^1}{x^m}}_{\phi(p)} \\[1em] \eval{\pdv{y^2}{x^1}}_{\phi(p)} & \eval{\pdv{y^2}{x^2}}_{\phi(p)} & \cdots & \cdots & \eval{\pdv{y^2}{x^m}}_{\phi(p)} \\[1em] \vdots & \vdots & \vdots & \vdots & \vdots \\[1em] \vdots & \vdots & \vdots & \vdots & \vdots \\[1em] \eval{\pdv{y^m}{x^1}}_{\phi(p)} & \eval{\pdv{y^m}{x^2}}_{\phi(p)} & \cdots & \cdots & \eval{\pdv{y^m}{x^m}}_{\phi(p)}\end{bmatrix}
    \]
    It follows from Proposition 3.6d) that the matrix $\eval{D(\psi\circ\phi^{-1})}_{\phi(p)}$ is invertible, as $\psi\circ\phi^{-1}$ is a diffeomorphism.
\end{proof}

An important application of this is the following. We begin with the $\realm\to\realn$ case. We will see that if $p$ and $F(p)$ are represented by another pair of coordinate charts (smoothly compatible with the previous pair), then the rank of $dF_p$ does not change. So the rank of the differential is an invariant of the choice of coordinate chart.

\begin{definition}[Matrix representation of the differential of $F:\realm\to\realn$]
Let $F\in \cinf[\realm,\realn]$, and $p\in\realm$ induces two charts $p\in (U,\id{\realm})$ and $F(p)\in(V\,\id{\realn})$, where $U\subseteq\realm$ and $V\subseteq\realn$. The matrix representation of the differential at $p$, $dF_p: T_p\realm\to T_{F(p)}\realn$ is nothing but the Jacobian matrix of $F$ at $p$.

\begin{equation}\label{lee-chap3:euclidean-differential-jacobian}
    \mcal\{dF_p\} = DF(p) = \begin{bmatrix}
        \eval{\pdv{F^1}{x^1}}_{p} & \eval{\pdv{F^1}{x^2}}_{p} & \cdots & \cdots & \eval{\pdv{F^1}{x^m}}_{p} \\[1em] \eval{\pdv{F^2}{x^1}}_{p} & \eval{\pdv{F^2}{x^2}}_{p} & \cdots & \cdots & \eval{\pdv{F^2}{x^m}}_{p} \\[1em] \vdots & \vdots & \vdots & \vdots & \vdots \\[1em] \vdots & \vdots & \vdots & \vdots & \vdots \\[1em] \eval{\pdv{F^n}{x^1}}_{p} & \eval{\pdv{F^n}{x^2}}_{p} & \cdots & \cdots & \eval{\pdv{F^n}{x^m}}_{p}
    \end{bmatrix}
\end{equation}

\end{definition}


\begin{definition}[Matrix representation of the differential of $F: M\to N$]
Let $F\in \cinf[M,N]$, and $p\in M$ induces two charts $p\in (U,\phi)$ and $F(p)\in(V\,\psi)$. The matrix representation of the differential at $p$, $dF_p: T_p N\to T_{F(p)}N$ is nothing but the Jacobian matrix of the coordinate representation at $p$.
\begin{equation}\label{lee-chap3:jacobian-matrix-representation}
        \mcal\{dF_p\} = \begin{bmatrix}\eval{\pdv{\hat{F}^1}{x^1}}_{\phi(p)} & \eval{\pdv{\hat{F}^1}{x^2}}_{\phi(p)} & \cdots & \cdots & \eval{\pdv{\hat{F}^1}{x^m}}_{\phi(p)} \\[1em] \eval{\pdv{\hat{F}^2}{x^1}}_{\phi(p)} & \eval{\pdv{\hat{F}^2}{x^2}}_{\phi(p)} & \cdots & \cdots & \eval{\pdv{\hat{F}^2}{x^m}}_{\phi(p)} \\[1em] \vdots & \vdots & \vdots & \vdots & \vdots \\[1em] \vdots & \vdots & \vdots & \vdots & \vdots \\[1em] \eval{\pdv{\hat{F}^n}{x^1}}_{\phi(p)} & \eval{\pdv{\hat{F}^n}{x^2}}_{\phi(p)} & \cdots & \cdots & \eval{\pdv{\hat{F}^n}{x^m}}_{\phi(p)}\end{bmatrix}
\end{equation}

Alternately, if we write $\hat{p} = \phi(p)$ as the $\realm$ coordinates at $p$, then

\begin{equation}\label{lee-chap3:jacobian-matrix-representation-alternate}
        \mcal\{dF_p\} = \begin{bmatrix}\eval{\pdv{\hat{F}^1}{x^1}}_{\hat{p}} & \eval{\pdv{\hat{F}^1}{x^2}}_{\hat{p}} & \cdots & \cdots & \eval{\pdv{\hat{F}^1}{x^m}}_{\hat{p}} \\[1em] \eval{\pdv{\hat{F}^2}{x^1}}_{\hat{p}} & \eval{\pdv{\hat{F}^2}{x^2}}_{\hat{p}} & \cdots & \cdots & \eval{\pdv{\hat{F}^2}{x^m}}_{\hat{p}} \\[1em] \vdots & \vdots & \vdots & \vdots & \vdots \\[1em] \vdots & \vdots & \vdots & \vdots & \vdots \\[1em] \eval{\pdv{\hat{F}^n}{x^1}}_{\hat{p}} & \eval{\pdv{\hat{F}^n}{x^2}}_{\hat{p}} & \cdots & \cdots & \eval{\pdv{\hat{F}^n}{x^m}}_{\hat{p}}\end{bmatrix}
\end{equation}
\end{definition}

\begin{wts}\label{lee-chap3:rank-of-differential-invariant-under-coordinate-change}
    Let $F$ be a smooth map between $M$ and $N$, at every $p\in M$, $\rank dF_p$ is an invariant over (smoothly compatible) pairs of charts in $M$ and $N$.
\end{wts}
\begin{proof}
    Let $p\in (U_1,\phi_1)\cap (U_2,\phi_2)$, and $F(p)\in (V_1,\psi_1)\cap (V_2,\psi_2)$. Where all charts are smoothly compatible if it makes sense to talk about it. Both $\phi_2\circ\phi_1^{-1}$ and $\psi_2\circ\psi_1^{-1}$ are diffeomorphisms, and the change of basis matrices $\eval{D(\phi_2\circ\phi_1^{-1})}_{\phi_1(p)}$ and $\eval{D(\psi_2\circ\psi_1^{-1})}_{\psi_1(F(p))}$ are invertible by Proposition 3.6d) again, so the ranks $dF_p$ with respect to any of the two charts are equal.
    \[
        \underbracket{\eval{D(\psi_2\circ\psi_1^{-1})}_{\psi_1(F(p))}}_{\text{invertible}}\biggl(\mcal\{dF_p\}\biggr)\underbracket{\eval{D(\phi_2\circ\phi_1^{-1})}_{\phi_1(p)}}_{\text{invertible}}
    \]
\end{proof}



\end{document}