\usepackage[utf8]{inputenc}
\usepackage{amsmath}
\usepackage{amssymb}
\usepackage{amsfonts}
\usepackage{amsthm}
\usepackage{enumitem}
\usepackage{titlesec}

%Some headers to structure our assignments
\newtheorem*{wts}{WTS}
% \newtheorem*{answer}{Answer} removed
\newtheorem{theorem}{Theorem}[section]
\newtheorem{corollary}{Corollary}[theorem]
\newtheorem{lemma}[theorem]{Lemma}
\newtheorem*{remark}{Remark}
\newcommand{\problem}[1]{\section*{Problem #1}} %question command


\usepackage{pifont}
\usepackage{mathtools}

%for putting chapter numbers after the title
%\titleformat{\section}[hang]{\normalfont\Large\bfseries}{Chapter }{0em}{}{}

%Useful for defining functions and variables. (not centered properly?)
\newcommand{\defined}{\coloneqq}

%Some common sets
\newcommand{\real}{\mathbb{R}}
\newcommand{\nat}{\mathbb{N}}
\newcommand{\integer}{\mathbb{Z}}

\newcommand{\tx}[1]{\text{#1}}
\setlength{\parindent}{0pt} %default no indent
%New environment for labelling (a) (b) subproblems
\newenvironment{enumalpha}{\begin{enumerate}[label=(\alph*)]}{\end{enumerate}}

%Custom Operators, Real and Imaginary Parts
\renewcommand{\Re}{\operatorname{Re}}
\renewcommand{\Im}{\operatorname{Im}} % How do we use operators?

%Custom Operators, Domain, Range, Codomain, identity function
\newcommand{\range}{\operatorname{range}}
\newcommand{\dom}{\operatorname{dom}}
\newcommand{\codom}{\operatorname{codom}}
\newcommand{\id}[1]{\operatorname{id}_{#1}} %use with \id_{\real} as the identity function the reals.
\newcommand{\sgn}{\operatorname{sgn}}
%always put this at last

%shortcut for norms
\newcommand{\norm}[1]{\lVert {#1} \rVert}
\newcommand{\bignorm}[1]{\left\lVert {#1} \right\rVert}

%topology
\newcommand{\acc}{\operatorname{acc}} %accumultation points
\newcommand{\abrackets}[1]{\langle {#1} \rangle} %angled brackets like <x> for nets
\newcommand{\nb}[1]{\mathcal{N}_B({#1})} %N(x) is a neighbourhood of x
\newcommand{\Tau}{\mathrm{T}} %big tau for toplogical space
\newcommand{\cl}[1]{\overline{#1}} % closure

\usepackage{subfiles}
