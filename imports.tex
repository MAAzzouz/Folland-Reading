\usepackage[utf8]{inputenc}
\usepackage{amsmath}
\usepackage{amssymb}
\usepackage{amsfonts}
\usepackage{amsthm}
\usepackage{enumitem}
\usepackage{titlesec}

%Changing the font?
%\usepackage{fourier}
\usepackage{mlmodern}


%Some headers to structure our assignments
\newtheorem*{wts}{WTS}
% \newtheorem*{answer}{Answer} removed
\newtheorem{theorem}{Theorem}[section]
\newtheorem{corollary}{Corollary}[theorem]
\newtheorem{lemma}[theorem]{Lemma}
\newtheorem*{remark}{Remark}
\newcommand{\problem}[1]{\section*{Problem #1}} %question command


\usepackage{pifont}
\usepackage{mathtools}

%for putting chapter numbers after the title
%\titleformat{\section}[hang]{\normalfont\Large\bfseries}{Chapter }{0em}{}{}

%Useful for defining functions and variables. (not centered properly?)
\newcommand{\defined}{\coloneqq}

%Some common sets
\newcommand{\real}{\mathbb{R}}
\newcommand{\nat}{\mathbb{N}}
\newcommand{\integer}{\mathbb{Z}}

\newcommand{\tx}[1]{\text{#1}}
\setlength{\parindent}{0pt} %default no indent
%New environment for labelling (a) (b) subproblems
\newenvironment{enumalpha}{\begin{enumerate}[label=(\alph*)]}{\end{enumerate}}

%Custom Operators, Real and Imaginary Parts
\renewcommand{\Re}{\operatorname{Re}}
\renewcommand{\Im}{\operatorname{Im}} % How do we use operators?

%Custom Operators, Domain, Range, Codomain, identity function
\newcommand{\range}{\operatorname{range}}
\newcommand{\dom}{\operatorname{dom}}
\newcommand{\codom}{\operatorname{codom}}
\newcommand{\id}[1]{\operatorname{id}_{#1}} %use with \id_{\real} as the identity function the reals.
\newcommand{\sgn}{\operatorname{sgn}}
%always put this at last

%shortcut for norms
\newcommand{\norm}[1]{\lVert {#1} \rVert}
\newcommand{\bignorm}[1]{\left\lVert {#1} \right\rVert}

%topology
\newcommand{\acc}{\operatorname{acc}} % Set of Accumulation Points
\newcommand{\abrackets}[1]{\langle {#1} \rangle} % Angled Brackets for Nets
\newcommand{\nb}[1]{\mathcal{N}_B({#1})} %N_b(x) is a neighbourhood of x
\newcommand{\Tau}{\mathcal{T}} % Big Tau Denoting Topological Space
\newcommand{\cl}[1]{\overline{#1}} % Closure
\newcommand{\clc}[1]{(\overline{#1})^c} %  Complement of the Closure
\newcommand{\Epsilon}{\mathcal{E}} % Base for a topology, or genearting set for an Algebra, Sigma Algebra, Topology, etc.

%pi maps for toplogy
\newcommand{\pmap}[2]{\pi_{#1}({#2})}
\newcommand{\pnv}[2]{\pi_{#1}^{-1}({#2})}

% Spaces of Functions on a Topological Space
\newcommand{\bc}[1]{\operatorname{BC}({#1})}
\newcommand{\cc}[1]{\operatorname{C}_c({#1})}
\newcommand{\cnot}[1]{\operatorname{C}_0({#1})}
\newcommand{\supp}[1]{\operatorname{supp}{(#1)}}
\usepackage[]{hyperref}
\hypersetup{
    pdftitle={Your title here},
    pdfauthor={Your name here},
    pdfsubject={Your subject here},
    pdfkeywords={keyword1, keyword2},
    bookmarksnumbered=true,     
    bookmarksopen=true,         
    bookmarksopenlevel=1,       
    colorlinks=true,            
    pdfstartview=Fit,           
    pdfpagemode=UseOutlines,    % this is the option you were lookin for
    pdfpagelayout=TwoPageRight
}
\usepackage{subfiles}
